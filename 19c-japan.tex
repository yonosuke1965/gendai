
\chapter{日本の西洋思想受容}


\section{明治}


\subsection{福沢諭吉}


\subsection{中江兆民}


\subsection{立憲君主制}



\subsection{貧富の差}

貧民の生活の紹介。

河上肇『貧乏物語』(1916)。

\subsection{産業化と労働者}

工場法、労働者運動。

\subsection{社会主義}

片山潜、幸徳秋水らが社会民主党を設立。
人類同胞主義、軍備全廃、階級全廃、土地・資本・交通機関の公有化、分配の公平、平等な参政権、教育の無償化。即日解散を命じられる。

1906年、日本社会党設立。議会主義と直接行動論が対立。






\section{大正デモクラシー}

民主化。




\section{ナショナリズム}

北一輝とか。


\section{戦後}

新憲法。









%%% Local Variables:
%%% mode: japanese-latex
%%% TeX-master: "main_social_thought"
%%% coding: utf-8
%%% End:
