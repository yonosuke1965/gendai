\ifx\mybook\undefined
\documentclass[autodetect-engine,dvipdfmx-if-dvi,ja=standard]{bxjsarticle} \usepackage{mystyle}
\author{江口聡}
%\date{}
\title{勉強ティプス}
\if0 %----------------------------------------------------------------

\fi  %----------------------------------------------------------------
\begin{document}
\maketitle
\else\chapter{}
\fi




\section*{勉強ティプス}

\begin{itemize}

\item 年代はだいたいで覚える。細かいのは無理。〜世紀初頭、前半、なかば、後半、末、程度のおぼえかたでよい。

\item \emph{高校世界史の教科書と資料集}を手元に置いておく。年表を自分で作ってみるのもよい。高校で世界史を十分履修していない人は、まず中学校の「歴史」を確認しておこう。
\item 歴史の本はどんどん読むこと。高校で勉強されられた政治史中心の歴史よりは、もの(たとえば胡椒や砂糖)の歴史、文化(恋愛や結婚制度、帳簿、武器・戦争技術など)の歴史などがおもしろいだろう。
\item 人の名前は覚えにくい。肖像画や写真といっしょにすると覚えやすい。アルファベットでの綴り字も一応確認しておくと恥ずかしい読み間違いが減る(おぼえる必要はない)。人名、出来事等は、さっさとWikipediaなどで確認する癖をつけたい。Wikipediaもどんどん使う。
\item 河出書房新社の「ふくろうの本」シリーズ、新潮社の「とんぼの本」シリーズのような図版を中心にしたヴィジュアル本が読みやすいし印象に残りやすい。\emph{岩波ジュニア新書}も大学1回生で読むのにぴったりなので図書館を利用する。『砂糖の歴史』『フランス革命』など名著。

\item NHKの『映像の世紀』に代表される歴史と科学のテレビ番組もどんどん見よう。

\item 大学での勉強の仕方、ノートの取り方など研究すること。

\end{itemize}


\section*{文献}

\begin{itemize}
\item 専修大学出版企画委員会 (2018) 『新・知のツールボックス:新入生のための学び方サポートブック』、専修大学出版局
\item 吉原恵子 (2017) 『スタディスキルズ・トレーニング』、実教出版
\item 
\end{itemize}




\nocite{筒井美紀14:大学選び}


\ifx\mybook\undefined
\addcontentsline{toc}{chapter}{\bibname}
\bibliographystyle{jecon}
\bibliography{bib}


\end{document} %----------------------------------------------------------------
\fi






%%% Local Variables:
%%% mode: japanese-latex
%%% TeX-master: t
%%% coding: utf-8
%%% End:
