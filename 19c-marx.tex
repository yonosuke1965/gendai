\documentclass[]{jsarticle}
\usepackage{okumacro,plext}
\usepackage{natbib}
\usepackage{url}
\usepackage{txfonts}
\usepackage[utf8]{inputenc}
\usepackage[T1]{fontenc}
%\usepackage{my_resume}
%\usepackage{graphicx,wrapfig}
%\usepackage[greek,english]{babel}
%\usepackage{teubner}
%\usepackage[dvipdfm,bookmarkstype=toc=true,pdfauthor={江口聡, EGUCHI Satoshi}, pdftitle={}, pdfsubject={},pdfkeywords={},bookmarks=false, bookmarksopen=false,colorlinks=true,urlcolor=blue,linkcolor=black,citecolor=black,linktocpage=true]{hyperref}
  \AtBeginDvi{\special{pdf:tounicode EUC-UCS2}}% platex-utf8 でも OK
\author{江口聡}
%\date{}
\title{マルクス主義}
\if0 %----------------------------------------------------------------

\fi  %----------------------------------------------------------------
\begin{document}
\maketitle

% \section{ヘーゲル}
% \begin{itemize}
% \item 中央集権的な国家において、市民の利害関係は一致、愛情により統合される。
% \end{itemize}

\section{ブルジョワ道徳批判}

\begin{itemize}
\item 既存の宗教、法律、道徳などは単なる歴史的・社会的立場に制約された「イデオロギー」。
\item 一般的な道徳は、ブルジョワ(中産)階級の利益になるよう構成されている。
\end{itemize}


\section{史的唯物論}

\begin{itemize}
\item 歴史は生産様式の変化によって区分される。
\item 生産力は歴史とととも増大する。
\item 社会革命は生産力の増大によって要求される生産関係の変化の反映。
\item 階級と階級闘争。
\end{itemize}




\section{共産主義}

\begin{quote}
  共産主義社会の高次の段階では、個人が分業に奴隷的に隷属することが消滅する。・・・労働が生活の手段であるだけでなく、生の第一の必要となる。個人の全面的発達とともに生産力も発達し、すべての協同的な富がいっそう豊かに溢れでるようになる。こうした後にはじめて、ブルジョワ的権利という狭い地平は踏み越えられ、社会は旗に書き込む{\――}各人からは能力に応じて、各人には必要に応じて!
\end{quote}


生産資源の共有。
「共産主義理論は一つのフレーズ、私有財産の廃止というフレーズに要約することができる」



\section{搾取}

\begin{itemize}
\item 資本家による労働力の\emph{搾取}が根本問題。搾取は他者の不公正な利用。
\item 資本家は労働者から、労働の対価として労働者に支払われる賃金以上の価値(剰余`価値)を生
  産物としてひきだす。
\item 労働だけが価値を創出するのであり、資本家は生産物の価値の一部をうけとる。またれゆえ労働者は創出した価値以下の価値しかうけとらない。それゆえ、労働者は資本家に搾取されている。
\item 労働者は生産財を所有しておらず、資本家のために働くよう\emph{強制}されている。

\item 「労働はあらゆる富の源泉である、と経済学者たちはいっている。しかし労働は無限になおそれ以上のものである。それは人間生活のいちばん根本の条件なのであり、しかもある意味では、労働が人間そのものを創造したといわねばならない。」

\end{itemize}

\section{疎外}

\begin{itemize}
\item 労働疎外。


\item 「労働はもっとも労働者自身のものでありながら、もっとも労働者が疎外されているものである」
\item 労働は人間のもっとも重要な能力。しかし賃労働によって、労働者の労働力はたんなる商品になり、資本家の統制のもとにおかれる。さらに、資本主義社会では労働者は労働において頭を使うことがなく、それゆえ本質的な満足を得られない。

  \begin{quote}
    労働者は、彼が富をより生産すればするほど、彼の生産力の力と範囲とが増大すればするほど、それだけますます貧しくなる。労働者は商品をよりおおくつくればつくるほど、それだけますます彼はより安価な商品となる。事物世界の価値増大にぴったり比例して、人間世界の価値低下がひどくなる。(『経済学・哲学草稿』)
  \end{quote}


\item 生産手段を共有することによって、労働者は自分の労働生活の編成方法について発言でき、本質的な満足を得ることが可能になる。

\end{itemize}


『ゴータ綱領批判』。能力に応じて働き、必要に応じて取る。



\section{プルードン}

「私的所有は盗みだ!」





\nocite{高晃公95:マルクス}
\nocite{大川正彦04:マルクス}
\nocite{kymlicka90:_contem_polit_philos}
\nocite{marx1848:_manif_kommun_partei:水田}
\nocite{marx1848:_commun_manif}
\bibliographystyle{eguchi}  
\bibliography{bib}


\end{document} %----------------------------------------------------------------







%%% Local Variables:
%%% mode: japanese-latex
%%% TeX-master: t
%%% coding: utf-8
%%% End:
