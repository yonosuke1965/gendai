\documentclass[uplatex,dvipdfmx]{jsarticle} \usepackage{mystyle}%\author{} %date{}
\title{アメリカ独立革命・フランス革命・保守主義}
\if0 %----------------------------------------------------------------

\fi  %----------------------------------------------------------------
\begin{document}
\maketitle


\section{アメリカ独立}

トマス・ペイン(1737-1809)は『コモン・センス』『人間の権利』などでイギリス政府を批判し、民主制と代議制によれば大国でも共和政は可能だと主張し、王制ではなく共和制を支持。

のちに第3代米国大統領になるトマス・ジェファソンは中央集権的な政治に反対。連邦制による分権的な共和国を支持。三権分立を重視。


ジョージ・ワシントン、ベンジャミン・フランクリン、トマス・ジェファーソン、ジョン・アダムズ、ら。

税制に対する不満。所有権の問題。「代表なければ課税なし」。




1776年、アメリカは独立を宣言。ジェファーソンらが独立宣言を起草。


\section{フランス革命}

1889年、財政破綻した王権を否定したフランス革命が成立。共和政を目指すが、ジャコバン党独裁の恐怖政治に陥る。

\begin{itemize}
\item 貴族・聖職者の特権、特に司法、税制の特権に対する不満。
\item 1788、1789年の穀物不作 → 穀物暴動 → 1789年7月14日バスティーユ監獄襲撃。
\item 1789年、封建制の廃止。農奴の解放。領主の裁判権の破棄。 → 1793年、封建的土地所有権の破棄。農民が自分の土地の主人になる。
\item 1789年、「人間と市民の諸権利の宣言」(人権宣言)。(1)人は生まれながらにして、自由と平等の権利をもつ、(2)政治的結合の目的は、人間の生来の奪うべからざる権利の維持。自由、財産、安全、圧政への抵抗の権利。(3)主権は国民にある。(4)すべての市民は自ら、あるいは代表者を通じて立法に参加しうる、(5)思想・言論の自由、(6)所有権の神聖。
\item 1874年、奴隷を解放。
\item 上流市民のジロンド派と中小市民のジャコバン派との対立。→ ジャコバン派独裁の恐怖政治。 → テルミドールの反動 → ナポレオン政府 → 帝政。
\end{itemize}


\section{資料}



\subsection{アメリカ独立宣言}

\begin{quote}\footnotesize{}
  人の営みにおいて、ある人民にとって、他の人民と結びつけてきた政治的な絆を解消し、自然の法や自然の神の法によってその資格を与えられている独立した、対等の地位を地上の各国のうちに得ることが必要となるとき、人類の意見をしかるべく尊重するならば、その人民をして分離へと駆り立てた原因を宣言することが必要とされるだろう。

  我らは以下の諸事実を自明なものと見なす。すべての人間は平等につくられている。創造主によって、生存、自由そして幸福の追求を含むある侵すべからざる権利を与えられている。これらの権利を確実なものとするために、人は政府という機関をもつ。その正当な権力は被統治者の同意に基づいている。いかなる形態であれ政府がこれらの目的にとって破壊的となるときには、それを改めまたは廃止し、新たな政府を設立し、人民にとってその安全と幸福をもたらすのに最もふさわしいと思える仕方でその政府の基礎を据え、その権力を組織することは、人民の権利である。確かに分別に従えば、長く根を下ろしてきた政府を一時の原因によって軽々に変えるべきでないということになるだろう。事実、あらゆる経験の示すところによれば、人類は害悪が忍びうるものである限り、慣れ親しんだ形を廃することによって非を正そうとするよりは、堪え忍ぼうとする傾向がある。しかし、常に変わらず同じ目標を追及しての権力乱用と権利侵害が度重なり、人民を絶対専制のもとに帰せしめようとする企図が明らかとなるとき、そのような政府をなげうち、自らの将来の安全を守る新たな備えをすることは、人民にとっての権利であり、義務である。{\――}これら植民地が堪え忍んできた苦難はそうした域に達しており、植民地をしてこれまでの統治形態の変更を目指すことを余儀なくさせる必要性もまたしかりである。今日のグレートブリテン国王の歴史は、繰り返された侮辱と権利侵害の歴史であり、その事例はすべてこれらの諸邦に絶対君主制を樹立することを直接の目的としている。それを証明すべく、偏見のない世界に向かって一連の事実を提示しよう。(後略)

\end{quote}



\subsection*{フランス人権宣言}
\label{french}
\begin{quote}\footnotesize{}
  国民議会として組織されたフランス人民の代表者たちは、人権の不知・忘却または蔑視が公共の不幸と政府の腐敗の諸原因にほかならないことにかんがみて、一の厳粛な宣言のなかで、人の譲渡不能かつ神聖な自然権を展示することを決意したが、その意図するところは、社会統一体のすべての構成員がたえずこれを目前において、不断にその権利と義務を想起するようにするため、立法権および執行権の諸行為が随時すべての政治制度の目的との比較を可能にされて、よりいっそう尊重されるため、市民の要求が以後単純かつ確実な諸原理を基礎に置くものとなって、常に憲法の維持およびすべての者の幸福に向うものとなるためである。――その結果として国民議会は、至高の存在の面前でかつその庇護のもとに、つぎのような人および市民の権利を承認し、かつ宣言する。

  \begin{enumerate}



  \item[第一条] 人は、自由かつ権利において平等なものとして出生し、かつ生存する。社会的差別は、共同の利益の上にのみ設けることができる。

  \item[第二条] あらゆる政治的団結の目的は、人の消滅することのない自然権を保全することである。これらの権利は、自由・所有権・安全および圧政への抵抗である。

  \item[第三条] あらゆる主権の原理は、本質的に国民に存する。いずれの団体、いずれの個人も、国民から明示的に発するものでない権威を行ないえない。

  \item[第四条]自由は、他人を害しないすべてをなしうることに存する。その結果各人の自然権の行使は、社会の他の構成員にこれら同種の権利の共有を確保すること以上の限界を持たない。これらの限界は、法によってのみ、規定することができる。
 
  \item[第五条]法は、社会に有害な行為でなければ、禁止する権利を持たない。法により禁止されないすべてのことは、妨げることができず、また何人も法の命じないことをなすように強制されることがない。

  \end{enumerate}
\end{quote}


\section{バークの保守主義}

イギリスのバークはフランス革命に衝撃を受け『フランス革命の省察』を出版。国家は社会契約などといったものではなく、慣習をつうじて成立する。

\begin{itemize}
\item フランス革命を批判。
\item 歴史の重視。自由は国民の歴史に根ざした政体でなければ保障できない。
\item フランスの人権宣言は伝統や歴史の重要性を否定している。形而上学的な抽象であるために国民を服従させる情緒的な力をもてない。(神への愛、王への畏敬、聖職者への敬意、優れた人々への尊敬etc)
\item 「人権に関するかれらの理論に夢中になりすぎて、人間本性をまったく忘れてしまった」「政治は人間の推論に合わされるべきではなく、人間本性に合わされるべきであり、理性は人間本性のたんなる一部であってけっして最大の部分ではない。」
\item 『先入見は危急の場合即座に適用される。それは予め人の心を知恵と徳との普遍の進路に引きつけ、その人が決断に際して逡巡して懐疑的となったり、困惑したり、そして不決断になったりはさせない。先入観は人の徳を彼の習慣たらしめ、一連の無関係な行為とはしない。」
% \citet{macintyre67:_short_histor_of_ethic}深谷p.309
\end{itemize}



\bibliographystyle{eguchi}
\bibliography{bib}


\end{document} %----------------------------------------------------------------







%%% Local Variables:
%%% mode: japanese-latex
%%% TeX-master: t
%%% coding: utf-8
%%% End:
