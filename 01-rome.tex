\documentclass[uplatex,dvipdfmx]{jsarticle} \usepackage{mystyle}%\author{} %date{}
\title{ヘレニズム期〜古代ローマ}
\if0 %----------------------------------------------------------------

\fi  %----------------------------------------------------------------
\begin{document}
\maketitle

% \chapter{古代ギリシア}\fi


\section{ヘレニズム・ローマ時代}

\begin{itemize}
% \item ギリシアのポリス社会の崩壊。
\item アレクサンドロス大王(アリストテレスが家庭教師をした)がマケドニアを拡大。大帝国に。
  
\item 帝王は王の王。帝国は大小さまざまな国をたばねた国・民族の集合体。

\item  ギリシア文化が各地に伝播。しかしマケドニア帝国はアレクサンダーの死とともに崩壊。→ アンティゴノス朝マケドニア、セレウコス朝シリア、プトレマイオス朝エジプトなどに分裂。
\item ローマが勢力を伸ばす。エルトリア人の王がいたが、追放して前6世紀共和政に。貴族中心の共和制 → 最初の成文法である十二表法の制定(451B.C.)。 → イタリア半島統一 → 民主共和制へ移行(紀元前3世紀) → 地中海世界統一 → 前1世紀の内乱→ ポンペイウス、カエサル、クラッススの3人が内乱を平定して三頭政治 → カエサル(シーザー)独裁 → 帝政(ローマ帝国)(27 B.C.) → 大帝国に → ネロなどの暴君が出る → 五賢帝時代。パクス・ロマーナ(ローマによる平和)。
  
\item 3世紀ごろから不安定に。 → 内乱・異民族侵入に対応するため軍事力を強化。→ 増税により都市・地方は疲弊。
  
\item 軍人皇帝時代 → 3世紀末ディオクレティアヌス帝のころから帝王を神格化 → 専制君主時代。
\item コンスタンティヌス帝がキリスト教を公認。
  
\item ゲルマン人の大移動→帝国を東ローマ帝国と西ローマ帝国に分割 → 西ローマ帝国は滅亡 (476)。
\end{itemize}


\subsection{ローマの法制度}


\begin{itemize}
\item 広大なローマ帝国の諸民族すべてに適用される\emph{万民法}。
\item ギリシアのストア派哲学の影響。コスモポリタニズム(世界市民主義)。人類共通の\emph{自然法}が存在するという発想。
\item \emph{キケロ} (B.C. 106-B.C.43)。自然法は万人の自然本性に内在し、万民法はその写し。 → 万民法も恣意的なものではない。
\item   
\end{itemize}


\subsection{エピクロス派}

\begin{itemize}
\item エピクロス(341B.C. -- 270 B.C.)
\item 「快楽主義」と呼ばれるが、各種の苦を避け、実際には穏かな快を求める。「心の平安」(アタラクシア)が目標。

\end{itemize}

\subsection{ストア派}
\begin{itemize}
\item ゼノン(335B.C. -- 263B.C.)が開祖。ノモスにしばられた一ポリス市民ではなく、世界市民(コスモポリタン)として自然の法則と理性にしたがって生きる。

\item エピクテートス (55--135)、ローマ皇帝マルクス・アウレリウス(121--180)など。

\item   禁欲主義。不動心(アパテイア)。自分の義務を果たせ。人間が体験する苦難は、それに心をわずらわせねば消滅すると考える。不幸や不安や苦痛はすべて心のなかにある。それに心を向けなければ、あたかも存在しないも同然である。人間は世界を支配することはできないが、自分の心を支配することはできる。

\item   ローマの思想家に影響、支配者階層に支持される。
\end{itemize}





\section{資料}



\subsection{エピクロス }

    \begin{itemize}
    \item 「 自然のもらす富は限られており、また容易に獲得することができる。しかし、むなしい憶見の追い求める富は、限りなく拡がる。」 %79

    \item 「正しい人は、最も平静な心境にある。これに反し、不正な人は極度の動揺に満ちている。」

    \item 「欲望のうち、或るものは自然的でかつ必須であり、或るものは自然的だが必須ではなく、他のものは自然的でも必須でもなく、むなしい憶見によって生まれたものである。」


    \item 「飢えないこと、渇かないこと、寒くないこと、これが肉体の要求である。これらを所有したいと望んで所有するに至れば、その人は、幸福にかけては、ゼウスとさえ競いうるであろう。」

    \item 「生の限界を理解している人は、欠乏による苦しみを除き去って全生涯を完全なものにするものが、いかに容易に獲得されうるかを知っている。それゆえに、かれは、その獲得のために競争を招くようなものごとをすこしも必要としない。」

    \item 「水とパンで暮しておれば、わたしは身体上の快に満ち満ちていられる。そしてわたしは、ぜいたくによる快を、快それ自体のゆえにではないが、それに随伴していやなことが起るがゆえに、唾棄する。

    \item 「隠れて生きよ」
    \end{itemize}




\subsection*{エピクテートス}



   \begin{itemize}
   \item もろもろの存在のうち、あるものは私たちの権内にあるけれども、あるものは私たちの権内にはない。意見や意欲や欲求や忌避、一言でいって、およそ私たちの活動であるものは、私たちの権内にあるけれども、肉体や財産や評判や公職、一言でいって、およそ私たちの活動でないものは、私たちの権内にない。

   \item そして私たちの権内にあるものは、本性上自由であり、妨げられず、じゃまされないものであるが、私たちの権内にないものは、もろい、隷属的な、妨げられる、他に属するものだ。


   \item そこでつぎのことを記憶しておくがいい。もし本性上隷属的なものを自由なものと思い、他人のものを自分のものと思うならば、きみはじゃまされ、悲しみ、不安にされ、また、神々や人びとを非難するだろう。だが、もしきみのものだけをきみのものであると思い、他人のものを、事実そうであるように、他人のものと思うならば、だれもきみにけっして強制はしないだろう。だれもきみを妨げないだろう。きみはだれをも非難せず、だれをもとがめることはないだろう。きみはなにひとついやいやながらすることはなく、だれもきみに害を加えず、きみは敵を持たないだろう、なぜなら、きみはなにも害を受けないだろうから。

   \item 出来事が、きみの好きなように起こることを求めぬがいい、むしろ出来事が起こるように起こることを望みたまえ。そうすれば、きみは落ち着いていられるだろう。
   \end{itemize}



\subsection*{マルウス・アウレリウス}



 [第5章の1] 明け方に起きにくいときには、つぎの思いを念頭に用意しておくがよい。「人間のつとめを果たすために私は起きるのだ。」自分がそのために生まれ、そのためにこの世にきた役目をしに行くのを、まだぶつぶついっているのか。それとも自分という人間は夜具の中にもぐりこんで身を温めているために創られたのか。「だってこのほうが心地よいもの。」では君は心地よい思いをするために生まれたのか、いったい全体君は物事を受身に経験するために生まれたのか。それとも行動するために生まれたのか。小さな草木や小鳥や蟻や蜘蛛や蜜蜂までがおのがつとめにいそしみ、それぞれ自己の分を果して宇宙の秩序を形作っているのを見ないか。

  しかるに君は自分のつとめをするのがいやなのか。自然にかなった君の仕事を果すために馳せ参じないのか。「しかし休息もしなくてはならない。」それは私もそう思う、しかし自然はこのことにも限度をおいた。同様に食べたり飲んだりすることにも限度をおいた。ところが君はその限度を越え、適度を過すのだ。しかし行動においてはそうではなく、できるだけのことをしていない。

  結局君は自分自身を愛していないのだ。もしそうでなかったらば君はきっと自己の(内なる)自然とその意志を愛したであろう。ほかの人は自分の技量を愛してこれに要する労力のために身をすりきらし、入浴も食事も忘れている。ところが君ときては、彫金師が彫金を、舞踏家が舞踏を、守銭奴が金を、見栄坊がつまらぬ名声を貴ぶほどのにも自己の自然を大切にしないのだ。上にいった人たちは熱中すると寝食を忘れて自分の仕事を捗らせようとする。しかるに君には社会公共に役立つ活動はこれよりも価値のないものに見え、これよりも熱心にやるに値しないもののように考えられるのか。


\section{さらに学習するために}

\begin{itemize}
\item ここにあげた以外にも様々な哲学者が活躍した。ディオゲネス・ラエルティオス『ギリシア哲学者列伝』(岩波文庫)がおもしろい。
\item ローマの歴史はおもしろいし、文学作品や映画などで題材としてとりあげられ、また故事として使われるので勉強しよう。塩野七生『ローマ人の物語』(新潮社)などでもよい。
\end{itemize}





 \nocite{エピクロス59:教説と手紙}
 \nocite{エピクテートス58:人生談議:岩波}
 \nocite{マルクスアウレリウス56:自省録}
 \nocite{水田洋06:社会思想小史}
 \nocite{よくわかる法哲学}
 \nocite{よくわかる哲学}

 
\ifx\mybook\undefined
\bibliographystyle{jecon}
\bibliography{bib}


\end{document} %----------------------------------------------------------------

\fi





%%% Local Variables:
%%% mode: japanese-latex
%%% TeX-master: t
%%% coding: utf-8
%%% End:
