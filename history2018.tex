\ifx\mybook\undefined
\documentclass[autodetect-engine,dvipdfmx-if-dvi,ja=standard]{jsarticle} \usepackage{mystyle}
\author{江口聡}
%\date{}
\title{世界史常識}
\if0 %----------------------------------------------------------------

\fi  %----------------------------------------------------------------
\begin{document}
\maketitle
\else\chapter{}\fi


丸暗記でかまわないので、世界史の常識中の常識は1回生前期で頭に入れておくこと。


\section{先史時代}

先史時代=歴史に残ってない時代。考古学的資料しか使えないので不正確でわからないことも多い。

\begin{itemize}
\item 人類はアフリカ起源。ホモ・エレクトゥス、原人。
  \item 70万年前のジャワ原人、60万年前の北京原人。火の使用。
  \item   20万年前、旧人。ネアンデルタール人。石器、衣服、埋葬。
  
\item 5万年前のクロマニヨン人が直接の祖先。現在の人類は14-20万年前に共通の祖先をもつ(アフリカ単一起源説)。ヨーロッパ人とアジア人は7万年±1万3千年ぐらいに枝分かれした。
\item 1万年前ごろに氷河時代の終り、8500B.C.ごろ新石器の使用開始。農耕の開始。
\item 3000B.C.〜1500B.C. エジブト文明、メソポタミア文明、インダス文明、黄河文明。文字、都市、国王、国家。

\end{itemize}

語句:猿人、アウストラロピテクス(400万年前)。ネアンデルタール人、クロマニヨン人、アルタミラ遺跡、ラスコー遺跡、


\section{古代文明}
\begin{itemize}
\item 灌漑による農耕。文字の使用。暦の作成。巨大建造物。
\item 各地で宗教の成立。多神教的なものが多い。仏教は無神論的・哲学的。儒教は祭祀的。

\end{itemize}



\section{古代ギリシア}

語句:集住(シノイキスモス)、ポリス、アゴラ(広場)、ヘレネス、バルバロイ、ホメロス、ペルシャ戦争、マラトンの戦い、ピタゴラス、ソクラテス、プラトン、アリストテレス、マケドニア、アレクサンダー大王、ヘレニズム文化。
パルテノン宮殿。アテネの直接民主制。重装歩兵。ドラコン法。ソロンの革命。僭主。陶片追放(オストラシズム)、ペリクレス。ペルシャのダレイオス一世、サラミスの海戦。ペリクレス。
ペロポネソス戦争。衆愚政治。デマゴーゴス。
アンティゴノス朝マケドニア、プトレマイオス朝エジプト、セレウコス朝シリア。



\section{古代ローマ}


\begin{itemize}
\item 共和制。貴族が主導。平民は重装歩兵として重要。成文法(十二表法)。
\item カルタゴとのポエニ戦争。
\item ポンペイウス、カエサル、クラッススの三頭政治。カエサル暗殺。
\item オウタヴィアヌス、アントニウス、レポドゥスの三頭政治。オクタヴィアヌスが勝利。→ 皇帝アウグストゥスに。帝政の開始。
  
\item パックス・ローマーナ(ローマの平和)。ローマ法 → ヨーロッパ近代の法律まで影響。
\item 
\end{itemize}

語句:ユリウス・カエサル(ジュリアス・シーザー)、クレオパトラ、オクタヴィアヌス(アウグストゥス)、イエス、ビザンツ帝国、コロッセウム。

語句:共和制、コンスル(執政官)、ディクタトル(独裁官)、元老院、共和制。
重装歩兵。護民官。平民会。十二表法。カルタゴとのポエニ戦争。第一次・第二次三頭政治。
元首政(プリンキパトゥス)、パクス・ロマーナ(ローマの平和)、ネロ帝、
ネルヴァ、トラヤヌス、ハドリアヌス、アントニヌス・ピウス、マルクス・アウレリウス・アントニヌス。
コンスタンティヌス帝、ミラノ勅令、ニケーア会議、ユリアヌス帝。


\subsection{キリスト教}

ローマ帝国下のユダヤでイエスが宗教活動、保守派に憎まれ死刑となる。弟子たちはイエスの復活と信じ、イエスは神の子であり神であるという信仰を成立させる。



\section{中世}

395 ローマ帝国の分裂。476 西ローマ帝国滅亡。486 フランク王国成立。
634 イスラム帝国成立。 870 フランク王国分裂。 1038 セルジューク・トルコ成立。1096 十字軍開始。


語句:ゲルマン人の大移動、4世紀のローマ帝国の分裂、フランク王国、十字軍、

\subsection{イスラーム}

メッカの証人ムハンマドが、唯一神アラーへの服従を解く。622年メディナへ移住、630年メッカ占領。アラビア半島統一。

11世紀セルジューク・トルコはエルサレルムを占領。ローマ教皇は聖地奪還を呼びかけ1096から200年間断続的に十字軍を発見。

語句: ムハンマド、アッラー、メディアナ、ヒジュラ、ムスリム、ウンマ、シャリーア(イスラム法)、クルアーン(コーラン)。カリフ。


\ifx\mybook\undefined
\addcontentsline{toc}{chapter}{\bibname}
\bibliographystyle{eguchi}
\bibliography{bib}


\end{document} %----------------------------------------------------------------
\fi






%%% Local Variables:
%%% mode: japanese-latex
%%% TeX-master: t
%%% coding: utf-8
%%% End:
