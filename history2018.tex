\ifx\mybook\undefined
\documentclass[twocolumn,autodetect-engine,dvipdfmx-if-dvi,ja=standard]{jsarticle} \usepackage{mystyle}
\author{江口聡}
%\date{}
\title{世界史常識}
\if0 %----------------------------------------------------------------

\fi  %----------------------------------------------------------------
\begin{document}
\maketitle
\else\chapter{}\fi


丸暗記でかまわないので、世界史の常識中の常識は1回生前期で頭に入れておくこと。


\section{先史時代}

先史時代=歴史に残ってない時代。考古学的資料しか使えないので不正確でわからないことも多い。

\begin{itemize}
\item 人類はアフリカ起源。ホモ・エレクトゥス、原人。
  \item 70万年前のジャワ原人、60万年前の北京原人。火の使用。
  \item   20万年前、旧人。ネアンデルタール人。石器、衣服、埋葬。
  
\item 5万年前のクロマニヨン人が直接の祖先。現在の人類は14-20万年前に共通の祖先をもつ(アフリカ単一起源説)。ヨーロッパ人とアジア人は7万年±1万3千年ぐらいに枝分かれした。
\item 1万年前ごろに氷河時代の終り、8500B.C.ごろ新石器の使用開始。農耕の開始。
\item 3000B.C.〜1500B.C. エジブト文明、メソポタミア文明、インダス文明、黄河文明。文字、都市、国王、国家。

\end{itemize}

語句:猿人、アウストラロピテクス(400万年前)。ネアンデルタール人、クロマニヨン人、アルタミラ遺跡、ラスコー遺跡、





\section{古代文明}
\begin{itemize}
\item 灌漑による農耕。文字の使用。暦の作成。巨大建造物。
\item 各地で宗教の成立。多神教的なものが多い。仏教は無神論的・哲学的。儒教は祭祀的。

\end{itemize}



\section{古代ギリシア}

語句:集住(シノイキスモス)、ポリス、アゴラ(広場)、ヘレネス、バルバロイ、ホメロス、ペルシャ戦争、マラトンの戦い、ピタゴラス、ソクラテス、プラトン、アリストテレス、マケドニア、アレクサンダー大王、ヘレニズム文化。
パルテノン宮殿。アテネの直接民主制。重装歩兵。ドラコン法。ソロンの革命。僭主。陶片追放(オストラシズム)、ペリクレス。ペルシャのダレイオス一世、サラミスの海戦。ペリクレス。
ペロポネソス戦争。衆愚政治。デマゴーゴス。
アンティゴノス朝マケドニア、プトレマイオス朝エジプト、セレウコス朝シリア。


\begin{enumerate}
\item - 前8世紀ごろ集落が連合。アクロポリス(城山)とそのふもとにおかれたアゴラ(広場)を中心に集住。都市(ポリス)をつくる。地中海・黒海の沿岸に植民市を建設。交易活動。アルファベットを使う。
\item イオニア系の都市国家アテナイ。貴族政 → 民主政。
\item ドーリア系の都市国家スパルタ。貨幣の使用禁止、交易の禁止、兵士としてのきびしい教育などの「リュクルゴスの制」。
\item 古代ギリシア人は自分たちをヘレネス(英雄ヘレンの子孫)、異民族をバルバロイ(聞きぐるしい言葉を話す者)を呼び区別。オリンポス12神。
\item   アテネ市民とは、自分で武具を購入して重装歩兵となる者。
\item 最初の成文法ドラコン法(前7世紀後半)。ソロンの改革(前6世紀諸島)、財産政治。
\item 前6世紀中頃、ペイシストラトスが僭主(法によらない支配者)となり、中小農民を保護、独裁政治。
\item クレイステネスの改革。将来僭主になりそうなものを陶片追放(オストラキスモス)。
\item アケメネス朝ペルシャとのペルシャ戦争で勝利 → ペリクレス時代。民主政の全盛期。
\item 直接民主政。参政権は市民の成年男子のみ、重要事項は民会で決定、官職は抽選。奴隷もかなりの数存在。
\item スパルタとのペロポネソス戦争。扇動政治家(デマゴーグ)による衆愚政治。→ アテネ敗北。
\item スパルタの覇権も長くはなく、ポリス間抗争が続き、のちにマケドニアの支配下に入る。
\end{enumerate}
ヘレニズム期
\begin{enumerate}
\item - アレクサンドロス大王。アリストテレスを家庭教師にしていた。
\item ペルシアを滅ぼし、インダス川まで到達。
\item ギリシア文化+オリエント文化→ ヘレニズム文化。
\item マケドニア帝国の分裂→ アンティゴノス朝マケドニア、セレウコス朝シリア、プロレマイオス朝エジプト。

\end{enumerate}
\section{古代ローマ}


\begin{itemize}
\item 共和制。貴族が主導。平民は重装歩兵として重要。成文法(十二表法)。
\item カルタゴとのポエニ戦争。
\item ポンペイウス、カエサル、クラッススの三頭政治。カエサル暗殺。
\item オウタヴィアヌス、アントニウス、レポドゥスの三頭政治。オクタヴィアヌスが勝利。→ 皇帝アウグストゥスに。帝政の開始。
  
\item パックス・ローマーナ(ローマの平和)。ローマ法 → ヨーロッパ近代の法律まで影響。
\item 
\end{itemize}

語句:ユリウス・カエサル(ジュリアス・シーザー)、クレオパトラ、オクタヴィアヌス(アウグストゥス)、イエス、ビザンツ帝国、コロッセウム。

語句:共和制、コンスル(執政官)、ディクタトル(独裁官)、元老院、共和制。
重装歩兵。護民官。平民会。十二表法。カルタゴとのポエニ戦争。第一次・第二次三頭政治。
元首政(プリンキパトゥス)、パクス・ロマーナ(ローマの平和)、ネロ帝、
ネルヴァ、トラヤヌス、ハドリアヌス、アントニヌス・ピウス、マルクス・アウレリウス・アントニヌス。
コンスタンティヌス帝、ミラノ勅令、ニケーア会議、ユリアヌス帝。


古代ローマ
\begin{enumerate}
\item - ラテン人が都市国家ローマを建設。前6世紀末に共和制に。
\item 前3世紀前半にイタリア半島全体を支配。
\item 貴族(パトリキ)による共和制。元老院が最高決定機関。執政官(コンスル)を置く。非常時には独裁官(ディクタトル)を置く。平民は重装歩兵として活躍、次第に力を持ち貴族政治に不満。
\item 前5世紀前半、平民から選ばれる護民官を置き、元老院やコンスルの決定に拒否権を行使。
\item 前5世紀、十二表法。ローマ最初の成文法。
\item カルタゴとのポエニ戦争。カルタゴの名将ハンニバルに苦戦するが勝利。
\item 閥族派と平民派の対立。
\item グラックス兄弟の改革。大土地所有者の土地を没収して無産市民に分配しようとするが失敗。
\item 剣闘士スパルタクスの反乱などによって混乱。
\item 第1回三頭政治、→ 平民派のカエサル(シーザー)が台頭。カエサルが終身独裁官となり改革 → 暗殺される。
\item 第2回三頭政治。→ カエサルの養子オクタウィアヌスが台頭、アントニウスとエジプトのクレオパトラ連合軍を破り地中海世界を統一(前30年)。→ 皇帝アウグストゥスとなる。ローマ帝国成立。
\item ローマの平和(パクス・ロマーナ)。五賢帝の時代には領土も拡大。
\item 3世紀には帝国の全自由人にローマ市民権を与える。

\end{enumerate}
\section{ユダヤ民族}

\begin{enumerate}
\item - 前13世紀モーセがエジプトからヘブライ(イスラエル)人をひきいて脱出(出エジプト)。
\item パレスチナ地方は古代にはカナーン(カナン)と呼ばれていた。
\item 前10世紀ごろ、ダヴィデ王の時代にヘブライ人はカナーン人からイェルサレムを奪い都とした。
\item 前1000年ごろダビデの子ソロモン王の時代に繁栄。エジプト王の娘と結婚、エチオピア女王(シバの女王)と交流。
\item イスラエル王国とユダ王国に分裂。ユダ王国は新バビロニアのネブカドネザル2世により滅亡、バビロンへ強制移住させられ強制労働させられる(バビロン捕囚、前586〜538年)。前4世紀にはセレウコス朝シリアの支配下。
\item ユダヤ教では唯一神ヤハウェ(ヤーヴェ)のみを信仰し、ユダヤ人だけが救われるという選民思想を柱とした。
\item 前1世紀にはローマの支配下で自治国となる。ユダヤ王ヘロデ。
\item 1世紀、2世紀にそれぞれローマに対して大規模な抵抗運動を起こす。
\item 135年、ユダヤ人はイェルサレムへの立ち入りを禁止される。→ 世界各地への離散(ディアスポラ)。

\end{enumerate}
\section{キリスト教}

ローマ帝国下のユダヤでイエスが宗教活動、保守派に憎まれ死刑となる。弟子たちはイエスの復活と信じ、イエスは神の子であり神であるという信仰を成立させる。



\section{中世}

395 ローマ帝国の分裂。476 西ローマ帝国滅亡。486 フランク王国成立。
634 イスラム帝国成立。 870 フランク王国分裂。 1038 セルジューク・トルコ成立。1096 十字軍開始。


語句:ゲルマン人の大移動、4世紀のローマ帝国の分裂、フランク王国、十字軍、

\section{イスラーム}

メッカの証人ムハンマドが、唯一神アラーへの服従を解く。622年メディナへ移住、630年メッカ占領。アラビア半島統一。

11世紀セルジューク・トルコはエルサレルムを占領。ローマ教皇は聖地奪還を呼びかけ1096から200年間断続的に十字軍を発見。

語句: ムハンマド、アッラー、メディアナ、ヒジュラ、ムスリム、ウンマ、シャリーア(イスラム法)、クルアーン(コーラン)。カリフ。





\section{大航海時代}


\begin{enumerate}
\item 
\item ポルトガルのエンリケ航海王子が援助した航海者たちはアフリカ西岸のヴェルデ岬に到達した。
\item 1488年、ポルトガルの航海者がアフリカ南端の喜望峰に到達した。
\item コロンブスはスペインの援助を受けて、1492年にアメリカ海域に到達した。
\item ポルトガルのヴァスコ・ダ・ガマが1498年にインドに到達した。
\item ポルトガルは1510年にインドのゴア、1511年にマラッカを占領した。さらに1517年には中国の明と通称を開き、1557年にマカオに居住権を獲得した。また、1543年に日本の種子島に漂着した。
\item マゼランは1519年に香辛料の産地モルッカ諸島をめざして西まわりの大航海に出発し、1521年に現在のフィリピンで死亡した。
\item アメリカ大陸からトウモロコシ、ジャガイモ、タバコ、トマト、さつまいも、唐辛子などがヨーロッパにもたらされた。
\item アメリカ大陸のアステカ帝国は1521年にコルテスによって、インカ帝国は1533年にピサロによって滅ぼされた。

\end{enumerate}

\section{ルネサンス}


\begin{enumerate}
\item \item ルネサンスが最初に起こった都市はフィレンツェであり、金融業者のメディチ家がルネサンス文化の保護者として知られる。
\item フィレンツェの政治学者マキアヴェリは『君主論』を著して現実主義的な政治手法の重要性を説いた。
\end{enumerate}

\section{発明と自然科学}


\begin{enumerate}
\item
\item 火薬、羅針盤、活版印刷術の三つが三大発明と呼ばれる。
\item 活版印刷はグーテンベルクによって発明され、知識の普及に役立った。
\end{enumerate}


\section{宗教革命}


\begin{enumerate}
\item - メディチ家出身の教皇レオ十世はサン・ピエトロ大聖堂の改修費用のために免罪符(贖宥状)を発行した。
\item 1517年ドイツのルターは免罪符(贖宥状)を批判した「95箇条の論題」を発表した。彼は聖書だけが唯一の権威であるとしてローマ教皇と対立した。
\item ドイツのカール五世はルター派を弾圧したが、1555年のアウクスブルクの宗教和議で妥協した。
\item スイスでツヴィングリがチューリヒで宗教改革運動を開始した。
\item 同じくローマ教皇を批判したカルヴァンはジュネーブで布教を行なった。
\item カルヴァンの教えは商工業者の間で広まった。カルヴァンの教説を信奉する人々はイングランドではピューリタン(清教徒)、フランスではユグノーと呼ばれた。
\item イギリスのヘンリ八世の離婚はローマ教皇に認めらなかったため、1534年にローマ教会から分離した。その後、エリザベス一世が1559年に英国教会を確立した。
\item ローマ教会でも、イグナティウス・ロヨラが1534年に設立したイエズス会などを中心に宗教改革に対抗する改革がおこなわれた。イエズス会宣教師のマテオ・リッチは中国で、フランシスコ・ザビエルは日本へ伝道した。

\end{enumerate}


\section{絶対王政}


\begin{enumerate}
\item   - 15世紀末からヨーロッパの国家は次第に中央集権化し、絶対王政を確立しはじめる。
\item 国王による支配を正当化するために君主権は神から授けられたとする王権神授説が唱えられた。
\item 絶対王政では、集権的支配のための官僚組織と、それまでの封建騎士団や傭兵にかわる常備軍が設置された。
\item 常備軍や官僚を財政的に維持するために、重商主義政策がとられた。
\end{enumerate}


\section{市民革命}


\begin{enumerate}
\item - イギリスのジェームズ一世は王権神授説を唱え専制政治をおこなった。
\item チャールズ一世の統治下1628年に議会は「権利の請願」を通過させたが、チャールズは無視し、またピューリタンを弾圧した。
\item 1642年に内乱(ピューリタン戦争)が勃発し、クロムウェルが王党派を破った(ピューリタン革命)。1649年チャールズ一世は処刑された。
\item チャールズ一世の処刑後、イギリスは共和政となる。クロムウェルは1653年に護国卿となり独裁体制を敷くが、1658年に病死し、チャールズ二世が国王となる(王政復古)。
\item 議会は王の専制に対して1679年に人身保護法を発布した。
\item チャールズ二世の息子ジェームズ二世は専制体制とカトリック復興を目指したため議会と対立し、1688年に国外に追放される。議会はオランダからウィリアム三世とメアリ二世を迎えた(名誉革命)。
\item 1689年に「権利の章典」が発布された。
\item 1721年、ジョージ一世が国王のときにウォルポールが議会の主権と責任内閣制、および「王は君臨すれども統治せず」の原則を確立した。

\end{enumerate}



\begin{enumerate}
\item 1607年、イギリスは最初の北米植民地ヴァージニアを建設した。
\item 1608年、フランスは最初の北米植民地ケベックを建設し、カナダを領有した。
\item 1755年に英仏の間で植民地戦争が起き、1763年のパリ条約で集結した。イギリスはフランスからミシシッピ以のルイジアナ、カナダ、ドミニカ、セネガル、スペインからフロリダを獲得した。またスペインはフランスからミシシッピ以西のルイジアナを獲得した。
\item 1740年、プロイセンのフリードリヒ二世は、オーストリア継承戦争でシュレジエンを奪った。彼は啓蒙思想家のヴォルテールらと交流し、「君主は国家の第一の下僕」と称した。
\item 1756年、オーストリアのマリア・テレジアは、宿敵フランスを同盟を結び、プロイセン王国との七年戦争を起こした。
\item ロシアのエカテリーナ二世は、啓蒙思想家ヴォルテールと交流し、啓蒙専制君主として知られる。
\item ルイ十四世のヴェルサイユ宮殿が代表する豪壮・華麗な芸術様式をバロックと呼ぶ。これに対して、フリードリヒ二世のサンスーシ宮殿に代表される優美・繊細な芸術様式をロココと呼ぶ。
\item イギリスのニュートンは万有引力の法則を発見し『プリンキピア・マテマチカ』著して古典力学を確立した。
\item イギリスのボイルは気体力学の出発点を築いた。
\item フランスのラヴォワジェは質量保存の法則を発見し、燃焼の理論を確立した。
\item イギリスのハーヴェーは血液循環の仕組みを発見した。
\item イギリスのジェンナーは種痘法を開発し、予防接種を始めた。
\item 18世紀に、大麦→クローバー→小麦→カブと4年周期で輪作するノーフォーク農法が開発された。また食糧増産を目的に、イギリス議会は開放農地や牧羊地を耕作地として囲い込む第二次エンクロージャーを奨励した。(農業革命)
\item 18世紀後半、イギリスは商業覇権を確立した。新大陸の綿花・砂糖・タバコ・コーヒーなどを欧州に、欧州の綿布・酒・雑貨をアフリカに、アフリカの奴隷を新大陸にもちかえる大西洋三角貿易が盛んになった。
\item 七年戦争後、イギリス王は植民地に対する課税を強化した。植民地側は「代表なくして課税なし」の原則を採択して反発した。
\item 1773年、イギリス政府が茶の独占販売権を東インド会社に与えると、ボストン茶会事件が起きた。
\item 1775年、レキシントンの闘いでイギリス正規軍と植民地側民兵が衝突し、アメリカ独立戦争がはじまった。同年ワシントンが植民地軍総司令官に就任した。
\item 1776年、大陸会議はアメリカ独立宣言を発表した。
\item 独立宣言はジェファーソンが起草し、フランクリンとジョン・アダムズが加筆修正した。
\item 旧体制(アンシャン・レジーム)では、第一身分と呼ばれる聖職者、第二身分と呼ばれる貴族は封建的特権をもち、第三身分の平民は貧しい生活を送っていた。
\item 1879年、ルイ十六世が憲法制定議会を武力制圧しようとしたことをきっかけにして、パリ市民はバスティーユ牢獄を襲撃した。
\item フランス人権宣言はラファイエットが起草したもので、基本的人権、国民主権、私有財産の不可侵などを主張しており、イギリスの啓蒙思想家ロックの影響を受けている。
\item (フランス革命のまわりは非常に多くの重要な出来事があるので、どうしても一回入門書を読んでおく必要がある。とりあえず遅塚忠躬 『フランス革命:歴史における劇薬』(岩波ジュニア新書、1997) がおすすめ。)





\end{enumerate}

おぼえておきたい世界史事項 (4)

\begin{enumerate}
\item - 1814-1815年、ウィーン会議が開催され、フランス革命以前の国家の状態に戻す正統主義が基本原則とされた。
\item 1848年、各国での革命によりウィーン体制が崩壊した(「諸国民の春」)。


\item ジェンナーは種痘法を開発した。
\item アークライトは水力紡績機を発明した。
\item カーライトは力織機を開発した。
\item ワットは蒸気機関を改良し、工業動力として使用できるようにした。
\item ダービー父子は18世紀前半にコークスを用いて鉄鉱石を溶解する製鉄法を開発した。
\item 1804年最初の軌道式蒸気機関車が開発され、スティーブンソンが開発した。
\item 最初の鉄道の開通は1825年。実用化されたのは1830年のイギリスのマンチェスター〜リヴァプール間であった。
\item 産業革命期、工場制手工業にかわって、工場制機械工業が確立し、イギリスは世界の工場と呼ばれるようになった。
\item ファラデーの電磁気学、リービヒの有機化学はすぐに産業に応用された。
\item X線の発明者はレントゲンである。
\item ダーウィンは『種の起源』で進化論を唱えた。
\item パストゥールは狂犬病の予防接種を発明した。
\item コッホは結核菌やコレラ菌を発見した。
\item ダイナマイトを発明したのはノーベルである。
\item 1883年、ドイツのダイムラーはガソリンを燃料とする内燃機関を発明した。
\item ドイツのディーゼルは重油・軽油を燃料に使うディーゼルエンジンを発明した。
\item 19世紀前半にアメリカのモールスは電信機を発明した。
\item 無線通信を発明したのはイタリア人技術者のマルコーニである。
\item 世界初の万国博覧会(ロンドン万国博覧会)は1851年に開催された。
\item 1870年代なかばから、武力で海外市場(植民地)を獲得・維持しようとする傾向が強まる。これを帝国主義という。
\item 19世紀後半以降の重化学工業の発達を第二次産業革命という。
\item 日露戦争中の1905年、ロシアで第一次ロシア革命が起きる。
\item 1914年、サライェヴォ事件をきっかけに第一次世界大戦が起きる。同盟国(ドイツ・オーストリア・オスマン帝国など)と連合国(フランス、イギリス、ロシア、日本)などに分かれて闘う。戦争は長期化し、毒ガス、戦車などの新兵器が使われる。
\item 1917年、ロシアでレーニンに指導されるソヴィエト政府が成立。ロマノフ朝は滅亡。


\end{enumerate}


\ifx\mybook\undefined
\addcontentsline{toc}{chapter}{\bibname}

\bibliography{bib}






\end{document} %----------------------------------------------------------------
\fi






%%% Local Variables:
%%% mode: japanese-latex
%%% TeX-master: t
%%% coding: utf-8
%%% End:
