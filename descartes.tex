
\chapter{近代の科学}



\section{デカルト}

\begin{itemize}
\item ルネ・デカルト(1596--1650)
\item デカルト。「学問」(科学)の作りなおし。
\item 良識・精神を正しく使えば誰もが確実な真理・正しい答を得ることができるはず。
\end{itemize}

\begin{quote}
    良識はこの世で最も公平に配分されているものである。というのは、だれもかれもそれを十分に与えられていると思っていて、他のすべてのことでは満足させることのはなはだむずかしい人々でさえも、良識については、自分のもっている以上を望まぬのがつねだからである。・・・したがって、われわれの意見がまちまちであるのは、われわれのうちのある者が他の者よりもより多く理性をもつから起こるのではなく、ただわれわれが自分の考えをいろいろちがった\ruby{途}{みち}によって導き、また考えているのが同一のことでない、ということから起こるのであると。というのは、よい精神をもつというだけでは十分ではないのであって、たいせつなことは精神をよく用いることだからである。
  \end{quote}

\begin{itemize}
\item \emph{方法的懐疑}。いろんなものが信じられないので、とにかく疑ってみる。
\item デカルトの「四つの規則」(明晰判明、分析、総合、枚挙)
\end{itemize}

\begin{quotation}
    第一は、私が明証的に真であると認めたうでなくてはいかなるものをも真理としては受け入れないこと。いいかえれば、注意深く即断と偏見とを避けること。そして、私がそれを疑ういかなる理由ももたないほど、明晰かつ判明に、私の精神に現われるもの以外の何ものをも、私の判断のうちにとり入れないこと。

    第二、私が吟味する問題のおのおのを、できる限り多くの、しかもその問題を最もよく解くために必要なだけの数の、小部分に分かつこと。

    第三、私の思想を順序に従って導くこと。最も単純で最も認識しやすいものからはじめて、少しずつ、いわば階段を踏んで、最も複雑なものの認識にまでのぼってゆき、かつ自然のままでは前後の順序をもたぬものの間にさえも順序を想定して進むこと。

    最後には、何ものをも見落とすことがなかったと確信するほどに、完全な枚挙と、全体にわたる通覧とを、あらゆる場合に行なうこと。(デカルト、『方法序説』(1637)、野田又夫訳、中公文庫、1974、p.27)
  \end{quotation}
  







%%% Local Variables:
%%% mode: japanese-latex
%%% TeX-master: t
%%% coding: utf-8
%%% End:
