\documentclass[uplatex,dvipdfmx]{jsarticle}
\usepackage{okumacro,plext}
\usepackage{natbib}
\usepackage{url}
\usepackage{txfonts}
\usepackage[utf8]{inputenc}
\usepackage[T1]{fontenc}
\usepackage{otf}
%\usepackage{my_resume}
%\usepackage{graphicx,wrapfig}
%\usepackage[greek,english]{babel}
%\usepackage{teubner}
%\usepackage[dvipdfm,bookmarkstype=toc=true,pdfauthor={江口聡, EGUCHI Satoshi}, pdftitle={}, pdfsubject={},pdfkeywords={},bookmarks=false, bookmarksopen=false,colorlinks=true,urlcolor=blue,linkcolor=black,citecolor=black,linktocpage=true]{hyperref}
  \AtBeginDvi{\special{pdf:tounicode EUC-UCS2}}% platex-utf8 でも OK
\author{江口聡}
%\date{}
\title{現代社会入門I 小テスト}


% \def\anaumei#1{\xkskip%
% \framebox[1.5cm]{\bf (\rensuji\label{#1})}\xkskip}
% \def\anaakiref#1{{\bf (\ref{#1})}}

\usepackage{renban}



% \newcommand{\sentakusi}[4]{
% \hspace{.3zw}
% \emph{ア}\hspace{1zw} #1 \hspace{2zw} \emph{イ} \hspace{1zw}#2 \hspace{2zw}\emph{ウ}\hspace{1zw} #3 \hspace{2zw}\emph{エ}\hspace{1zw} #4

% }


\newcounter{qnumber}\setcounter{qnumber}{1}
\def\anaume{\hspace{.5zw}\framebox[1.5cm]{\bf \theqnumber}\hspace{.5zw}\stepcounter{qnumber}}

\newcounter{anumber}\setcounter{anumber}{1}
\newcommand\sentakusi[1]{{{{\bf \Alph{anumber}}~#1}\hspace{1zw}\\} \stepcounter{anumber}}

% \sentakusi{}{}{}{}



\if0 %----------------------------------------------------------------

\fi  %----------------------------------------------------------------
\begin{document}
\maketitle

\subsection*{問1 以下の文章を読み、空白にもっとも適切であるものの下の記号群から選べ }

\begin{enumerate}

\setlength{\parskip}{.9zw}
\setlength{\itemsep}{.9zw}







  
% \item 共和政ローマでは、第1回三頭政治ののちに平民派の\空欄数字{}が台頭し終身独裁官となるが、暗殺される。

  
% \item のちに信者たちによって救世主とされるイエスが生まれたころのユダヤ王は\空欄数字{}である。
  
% \item 哲学者アリストテレスを教師にしていたマケドニアの\空欄数字{}は、ペルシアをほろぼしインダス川まで至る大帝国を築いた。
  
\item 共和政ローマでは、第1回三頭政治ののちに平民派の\空欄数字{}が台頭し終身独裁官となるが、暗殺される。



\item フィレンツェの\空欄数字{}は『君主論』を著して、君主は道徳にとらわれず、軍や嘘や裏切りなどあらゆる手段を使って権力の維持と拡大を目指すべきだと主張した。
  
\item ヨーロッパで活版印刷を発明したのは\空欄数字{}である。

\item サン・ピエトロ大聖堂の改修のため贖宥状を大量に発行した教皇レオ十世は\空欄数字{}の出身だった。
  
\item 17世紀イギリスの内乱で、チャールズ一世の処刑後の共和制で権力を握ったのは\空欄数字{}である。

  
\item 16世紀から18世紀にかけて絶対君主制の国家がとった経済政策を\空欄数字{}政策という。
  
\item 16世紀、フランス人のカルヴァンが活動したのは\空欄数字{}である。
  
\item 1721年、イギリスで\空欄数字{}が第一大蔵卿に就任し、事実上最初の「イギリス首相」として議会の支持を背景に政治をおこなった。

\item メキシコ高原にあったアステカ帝国を征服したのはスペインの\空欄数字{}である。

  
\item ルターやカルヴァンらの宗教改革に、カトリック教会側が対抗した活動は対抗宗教改革あるいはカトリック改革と呼ばれ、\空欄数字{}が創設したイエズス会の布教活動などが有名である。

  
\end{enumerate}  

\pagebreak{}

\begin{flushleft}
\emph{選択肢}
\setcounter{anumber}{1}


\sentakusi{アレクサンドロス}
\sentakusi{イグナチオ・デ・ロヨラ}
\sentakusi{カエサル}
\sentakusi{クロムウェル}
\sentakusi{グーテンベルク}
\sentakusi{コルテス}
\sentakusi{ジュネーヴ}
\sentakusi{スコットランド}
\sentakusi{カエサル(ガイウス・ユリウス・カエサル)}
\sentakusi{チェーザレ・ボルジア}
\sentakusi{チューリッヒ}
\sentakusi{ネロ}
\sentakusi{ハプスブルク家}
\sentakusi{ピザロ}
\sentakusi{フランシスコ・ザヴィエル}
\sentakusi{ブルボン家}
\sentakusi{マキアヴェリ}
\sentakusi{マテオ・リッチ}
\sentakusi{ヴェネツィア}
\sentakusi{重商主義}
\sentakusi{重農主義}

\end{flushleft}

% \subsection*{問2 (エキストラ問題、答えられれば最高+10点) }
% 「人権」という考え方は、世界史・思想史においてどのように成立したか、中学1年生に向けて2〜3分程度で説明するつもりで簡潔に述べよ。


\end{document} %----------------------------------------------------------------







%%% Local Variables:
%%% mode: japanese-latex
%%% TeX-master: t
%%% coding: utf-8
%%% End:
