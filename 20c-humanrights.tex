\ifx\mybook\undefined
\documentclass[uplatex,dvipdfmx]{jsarticle}\usepackage{mystyle}
\author{江口聡}
%\date{}
\title{ナショナリズム、帝国主義、ファシズム、そして人権}
\if0 %----------------------------------------------------------------

\fi  %----------------------------------------------------------------
\begin{document}
\maketitle

\else\chapter{ナショナリズム、帝国主義、ファシズム、そして人権}\fi


\section{奴隷解放}

\begin{itemize}
\item  フランス 1784。
\item アメリカ。リンカーン。南北戦争。 1864。
\end{itemize}





\section{第一次世界大戦}

オーストリア・ハンガリー帝国が崩壊。


\section{ソビエト連邦の成立}

1917年ロシア2月革命、同年10月革命。1922年樹立宣言。一党独裁。レーニン死去後スターリンが独裁。ネップ(新経済政策)と呼ばれる計画経済。コルホーズなどでの農業集団化。


\section{ファシズム}
\begin{itemize}
\item 中間層を基盤とした大衆運動。
\item 反動。権威主義。

\item ナショナリズム。
\end{itemize}


\nocite{水田洋91:社会思想史への招待,山脇直司92:ヨーロッパ社会思想史,水田洋06:社会思想小史}


\subsection{国際連盟}

アメリカ参加せず。20年代なかばから30年代にかけて中米諸国が脱退、ソ連は1934年から。日本、ドイツ、イタリアが脱退。ソ連を除名。


\section{第二次世界大戦}


\section{国際連合}

1945年設立。


\section{人権宣言}

エレノア・ルーズベルトらが起草。








\section{さらに学習するために}

\begin{itemize}
\item エレノアの本。
\end{itemize}


\ifx\mybook\undefined
\bibliographystyle{eguchi}  
\bibliography{bib,library}

\end{document} %----------------------------------------------------------------

\fi





%%% Local Variables:
%%% mode: japanese-latex
%%% TeX-master: t
%%% coding: utf-8
%%% End:
