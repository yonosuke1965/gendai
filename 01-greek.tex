\documentclass[uplatex,dvipdfmx]{jsarticle} \usepackage{mystyle}%\author{} %date{}
\title{原始国家〜古代ギリシア}
\if0 %----------------------------------------------------------------

\fi  %----------------------------------------------------------------
\begin{document}
\maketitle

% \chapter{古代ギリシア}\fi



\section{古代ギリシアのポリス社会}

\subsection{ポリス}

\begin{itemize}
\item ギリシア。紀元前8世紀ごろ、人々が集住して都市「\emph{ポリス}」ができる。
\item 古代ギリシアの\emph{ポリス}国家。城壁で囲まれた都市が一つの国家。
\item 「国家」とは? 現代的に考えると、(1) 土地、(2)人民、(3)主権をもっている人の集団が国家。「主権」とは他の国から独立に意思決定をおこなうこと。それ以上の決定組織がない。国民から税金を集めたり、戦争したりする決定力をもっているのが国。
\item 各ポリスによって制度・法律はさまざま。\emph{王政}の国(王が最終的な決定権をもつ)、\emph{共和制}(王が存在せず、国家元首を直接・間接に選出したり、複数の代表者を置いたりする)・\emph{民主制}(共和制のなかでも、国民全体の意志にもとづく政治をおこなう)の国など。
アテネのライバル国家のスパルタは王政(貴族制)。

\end{itemize}

\subsection{アテネの民主制}
\begin{itemize}
\item アテネの民主制が民主主義の原点として有名で、世界史的に重要とされている。
\item 奴隷と戦争の重要性。土地と奴隷を確保するための戦争が頻繁にある。騎馬を利用する貴族による戦争から、次第に富裕な平民が武具を用意して重装歩兵として活躍するようになる。平民が戦争参加の見返りに参政権を主張し貴族と対立。
\item 紀元前7世紀のドラコン法による「法の支配」(法治主義)(⇔ 人の支配(人治主義))。紀元前6世紀初頭ソロンの改革。血統ではなく財産額による参政権(財産政治)。
\item 非合法な独裁者である僭主(tyranny)の登場。法には反するが、平民の支持があった。→ 僭主政治の崩壊、僭主の出現を予防する陶片追放(オストラシズム)制度の採用。
\item 紀元前5世紀はアテネの黄金時代。
\item 人口20万程度の(当時の)大都市。奴隷もいた(20万人中の8万人ぐらいか)。
\item 「市民」(国民)とされるのは財産をもった成人男性。戦争になれば自分で装備して出兵する。中上流家庭の女性は家の外に出なかった。
\item 621B.C. ドラコン法(慣習法を成文化)。→ 僭主(独裁者)の登場 → 陶片追放(オストラシズム)制度で潜主の再登場を避けようとする。
\item ペルシア戦争(500 B.C. -- 449 B.C.)。
\item ペリクレス時代(443 B.C. -- 429 B.C.)。ペリクレスは民主制を称えた演説でも有名。直接民主制、官職は抽選。参政権は自由市民男子のみ。裁判も選出された裁判員の多数決。
\item アテネとスパルタとのペロポネソス戦争(431B.C. -- 404. B.C.)。
\item 扇動政治家(デマゴーク)が出現することもあった。 → 「デマ」の語源。
\item 市民の没落。傭兵の流行。
\item マケドニアが勢力を伸ばす。アレクサンドロス大王(336 B. C. -- 323 B.C.)がエジプト征服、ペルシアを滅ぼし、東方遠征。
\item アレクサンドロスの死後、アンティゴノス朝マケドニア、セレウコス朝シリア、プトレマイオス朝エジプトに分裂。アレクサンドロスからプトレマイオス朝エジプトの滅亡までが\emph{ヘレニズム時代}。
\end{itemize}


\section{ソフィストたち}

\begin{itemize}
\item ポリス社会では「弁論」が重要。投票で政治を決め、裁判も投票制なので弁論の技術が大事になる。
 → 弁論を教える職業的教師\emph{ソフィスト}(「知者」の意味)の登場。
\item プロタゴラスの「人間は万物の尺度」という言葉の代表されるように、\emph{相対主義}的な考え方を好む。
\item \emph{ノモス}(法、道徳)と\emph{ピュシス}(自然)を区別する。法や道徳などの社会のルール(ノモス)は人間が作りだしたものにすぎない。だから我々は弱肉強食の自然のルール(ピュシス)にしたがって、ズルできるときはするべきだ、といった主張がなされる。
\item → \emph{相対主義}。善悪、正・不正は社会の決め事にすぎず、社会によって違う。すべての社会に共通の「本当に正しいこと」などは存在せず、実は力の対立があるだけだけである。我々はよく生きようとすれば、知恵と力をつかって他人を支配しなければならない。

\end{itemize}




\section{ソクラテス}


\begin{itemize}
\item 元祖「哲学者」
\item 友人がデルフォイの神殿で「一番賢いのはソクラテスだ」という神託を受ける。「自分は賢い」と公言している人々(ソフィストや政治家たち)は本当に賢いのか?
\item 「汝自身を知れ」「ただ生きることではなくよく生きることが重要だ」「吟味されない生は生きるに値しない」などの言葉で知られる。
\item 知への愛 → 哲学 → 19世紀ごろまで学問はなんでも「哲学」。
\item ソフィストや政治家たちは、たとえば「善とは何か」「正義とは何か」を本当に知っているのか?

\item 議論・談論の重要性。
\end{itemize}

  \begin{quote}
    人間にとっては、徳その他のことについて、毎日談論するという、このことが、まさに最大の善きことなのであって、わたしがそれらについて、問答しながら、自分と他人を吟味しているのを、諸君は聞かれているわけであるが、これに反して、吟味のない生活は、人間の生きる生活ではないと、こう言っても、わたしがこう言うのを、諸君はなおさら信じないであろう。しかしそのことは、まさにわたしの言うとおりなのだ、諸君。ただそれを信じさせることが、容易ではないのです。(38a)
  \end{quote}
  \begin{itemize}

\item 問答法(対話法)。 → ソクラテス的方法 (Socratic method)。演説や教授ではなく、対話・問答によって真理を発見する。(しかしたいてい喧嘩別れ。はっきりした答はなかなか出てこない。)
\end{itemize}

  \begin{quote}
     わたしとは、どんな人間であるかといえば、もしわたしの言っていることに何か間違いでもあれば、こころよく反駁を受けるし、他方また、ひとの言っていることに何か本当でない点があれば、よろこんで反駁するような、とはいっても、反駁を受けることが、反駁することに比べて、少しも不愉快にならないような、そういう人間なのです。なぜなら、反駁を受けることの方が、より大きな善であるとわたしは考えているからです。(『ゴルギアス』 458A )
  \end{quote}
\begin{itemize}

\item 「無知の知」。知恵のある者(ソフィスト)ではなく、知恵を愛する者、知恵を求める者(phil 愛 + sophia 知恵 → philosophoi 哲学者\footnote{明治期に西周は当初philosophyを、哲を求める学問として希哲学と訳した。})。以降学者は自然科学者も含めてすべて「哲学者」を自称することになる。博士号 Ph.~D.は Philosophiae Doctor 「哲学博士」の略号。

\item 子どもに、その子どもにとって善いものを与えることと、子どもが欲するものを与えることはまったく違う。 → 「善い」ことは「本人が欲すること」とは違う。

\item 自由とは欲望のままに行動することではない。独裁者になって他人を殺すより、正しく生きて独裁者に殺される方が幸福でありよく生きている。

\item 健康、強さ、美しさ、富などは一般に有益であるが、時に害を与えることがある。それを\kenten{正しく} 使用するときに有益。節制、正義、勇気、理解力、寛大さなども知性がともなわなければ有害なことがある。

\item   → したがって徳は知。徳(aret\={e}, virtue)の基本の意味は「すぐれてある」こと、卓越性、能力。悪徳は無知による。

\item ソクラテスは新しい神を祭り若者に悪影響を与えるとして裁判にかけられ、死刑を宣告される。友人たちが脱獄をすすめるが、断わり毒を飲んで死ぬ。

\end{itemize}

\section{プラトン}

\begin{itemize}
\item ソクラテスの弟子の一人。ソクラテスの言行を書き残す(他にも政治家クセノポンや喜劇作家アリストパネスなどもソクラテスの言行を書き残している)。『ソクラテスの弁明』『クリトン』『ゴルギアス』など。
\item 学園アカデメイアを創設する。
\item 『国家』では自身の立場で政治を探究。正義とそれを実現する理想的な国家のあり方を考える。
\item 民主制(デモクラティア)を強烈に批判。ソクラテスを刑死させたように、(劣った)大衆の判断は当てにならない。
\item 人間には生まれつきの素質の上下がある。
\item 人間の魂が欲望と意志と理性に分けられるように、国家も生産者と防衛者と統治者に分けられる。
\item 統治者と防衛者は国家のために私的な欲望を捨てなければならない。→ 財産は持たない → 子どもは家族ではなく国家が養育する。
\item 統治者は子どものころから特別な教育を受ける必要がある。
\item 哲人王が統治する王政がベストの政体。民主制は必然的に腐敗する。
\end{itemize}


\section{アリストテレス}
\begin{itemize}
\item プラトンの弟子。アカデメイアで学んだのちに、学園リュケイオンを設立。散歩しながら議論したので逍遥学派と呼ばれる。
\item 現代で言えば文学、天文学、物理学、生物学など広範囲の学問の書物を残し、「万学の祖」と呼ばれる。→ アラビア語に翻訳され、中世にふたたびヨーロッパにもたらされる。近世のはじめまで自然科学についての権威とされていた。
\item 「人間は社会的動物である。」
\item 幸福=よく生きる≠単に楽しく暮らす。
\item よく生きる=人間の能力(徳)の発揮 = 自己実現。
\item 徳(アレテー)=卓越性。
\item 倫理的徳と知的徳。倫理的徳については「中庸」がベスト。勇気という徳は、臆病と蛮勇の中間、気前のよさという徳はケチと浪費の中間。
\item 正義(ディカイオシュネー)。正義にはいくつか意味がある。応報的正義。善には善を、悪には悪を与える。分配的正義 = 名誉や財産の正しく平等な分配。匡正的正義 = 平等な関係が不当に侵されたときに、悪に罰を加え、善に報奨を与える正義。
\item 友愛(ピリア)。ポリスは他者を承認し尊重しあう共同体(コイノニア)。

\item 国家体制
  \begin{itemize}
  \item × 単なる多数決による民主制。貧しいものの利益だけが追求される。
  \item × 僭主独裁制。少数のものの利益だけが追求される。
  \item 王政。王が法にしたがって市民全体の利益を追求する。
  \item 優秀者支配。エリートたちが市民全体の利益を追求。
  \item 国政(ポリーテイアー)。よき指導者のもとで市民全体の利益を追求する民主制(共和制)。
  \end{itemize}
\end{itemize}


\fi
\ifx\mybook\undefined
\bibliographystyle{eguchi}
\bibliography{bib}


\end{document} %----------------------------------------------------------------

\fi





%%% Local Variables:
%%% mode: japanese-latex
%%% TeX-master: t
%%% coding: utf-8
%%% End:
