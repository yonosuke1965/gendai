\ifx\mybook\undefined
\documentclass[uplatex,dvipdfmx]{jsarticle} \usepackage{mystyle}%\author{} %date{}
\title{ロマン主義、ヘーゲル主義、民族主義}
\if0 %----------------------------------------------------------------

\fi  %----------------------------------------------------------------
\begin{document}
\maketitle
\else\chapter{ロマン主義・ヘーゲル・民族主義}
\fi



\section{19世紀ロマン主義}

\begin{itemize}
\item 理性と普遍性を重視する啓蒙主義への反発。
\item 18世紀末から「ロマン主義」と呼ばれる思想・芸術運動が生じる。ドイツが中心。
\item 18世紀啓蒙主義の理性尊重への反発。ルソーらの人間的・自然的な感情を重視する立場からの影響。
\item ナポレオンの登場による英雄崇拝。
\item 芸術・宗教的感情の尊重。情熱的な恋愛の神聖視。
\item 文学ではゲーテ、シュレーゲル、ヘルダーリン、音楽ではベートーヴェンなどが主要人物。以降シューマン、ベルリオーズ、リスト、ワーグナーなど。
\item 個性、民族性、宗教的感情の重視。
\item 自然への注目。
\end{itemize}


\section{市民社会から国家主義へ}

\begin{itemize}
\item 18世紀末〜19世紀初頭にブルジョワ(有産階級)を中心にした「市民社会」がほぼ成立。資本主義。
\item 市民の政治的発言権。自由な競争。市民のなかでの貧富の差の拡大。
\item ホッブズ・ロック的な夜警国家から、より強力な権力をもった国家が求められるようになる。
\end{itemize}


\section{ナショナリズム}

フィヒテなど。





\section{ヘーゲル}

\begin{itemize}

\item ドイツからフランス革命を見たヘーゲル (1770--1831)は、人類の歴史を自由の実現する過程として捉えた。
個人はばらばらに存在するものではなく、具体的な社会制度(「人倫」)や組織のなかでこそ人間となる。

\item ヘーゲルの\emph{弁証法}という考え方は世界的に影響を与えた。
\item すべてのものは運動(変化)している。「正(テーゼ)」「反(アンチテーゼ)」の対立から「合(総合、ジンテーゼ)」が生み出される三段階を経てより高い段階に至る。

\item 歴史の発展も正・反・合の三段階を繰り返し、より大きな「自由」が実現される過程として理解できる。

\item \emph{人倫}の三段階。
  \begin{enumerate}
  \item \emph{家族}は自然的愛情に結ばれた共同体。男女\footnote{「男性は対外関係でたくましく活躍するもの、女性は受動的で主観的なもの」という対照がある。}の愛、その結果として子どもが生まれ、それは成長し独立した人格となって家族の外に出ていく。家族は共同体として存続するが、養育と教育によって子どもを自立させ、みずかから解体する。

\item \emph{市民社会}。独立したこどもは市民の一人として社会に参加する。市民社会は「欲望の体系」であり、各個人はみずからの欲望を追求するため労働に精をだす。しかし市民社会を成立させる市場の活動の結果、少数者への富の集中や多数者の窮乏化が起きる。

\item 中央集権化された\emph{国家}。個人の利益と全体の利益が一致。

  \end{enumerate}


% \section{ショーペンハウアー}


\subsection{発展する世界史}

\begin{itemize}
\item 歴史は一定の法則に従って発展しているという発想。
\item 古代は少数の人間のものだった自由が、次第に多くの人のものになってゆくという発展。
\item 「世界精神」。
\end{itemize}






\section{さらに学習するために}

宇野重規 (2013)『西洋政治思想史』、有斐閣。

ピーター・シンガー (1995)『ヘーゲル入門:精神の冒険』、島崎隆訳、青木書店。

\end{itemize}

\nocite{singer83:_hegel}

\ifx\mybook\undefined
\bibliographystyle{eguchi}
\bibliography{bib}
\end{document} %----------------------------------------------------------------
\fi






%%% Local Variables:
%%% mode: japanese-latex
%%% TeX-master: t
%%% coding: utf-8
%%% End:
