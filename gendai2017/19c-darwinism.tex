\documentclass[dvipdfmx]{jsarticle}
\usepackage{okumacro,plext}
\usepackage{natbib}
\usepackage{url}
\usepackage{txfonts}
\usepackage[utf8]{inputenc}
\usepackage[T1]{fontenc}
%\usepackage{my_resume}
%\usepackage{graphicx,wrapfig}
%\usepackage[greek,english]{babel}
%\usepackage{teubner}
%\usepackage[dvipdfm,bookmarkstype=toc=true,pdfauthor={江口聡, EGUCHI Satoshi}, pdftitle={}, pdfsubject={},pdfkeywords={},bookmarks=false, bookmarksopen=false,colorlinks=true,urlcolor=blue,linkcolor=black,citecolor=black,linktocpage=true]{hyperref}
  \AtBeginDvi{\special{pdf:tounicode EUC-UCS2}}% platex-utf8 でも OK
\author{江口聡}
%\date{}
\title{19世紀の自然科学と社会}
\if0 %----------------------------------------------------------------

\fi  %----------------------------------------------------------------
\begin{document}
\maketitle


\section{ダーウィン以前}

\begin{enumerate}
 \item キリスト教的理解=「神が作った種」。デザイン論。時計を見れば時計職
       人がそれを作ったことがわかるように、生物のようによくデザインされた
       ものを見ればそれを創造した者がいることがわかる。人間以外の生物は人間のために造られた。

\item 18世紀までの世界観。
\item デザイン論。人間も他の自然物も神がデザインした→動植物は人間が使
  う「ため」にある。
\item 存在の連鎖。大気→水→土→樹木→昆虫→爬虫類→魚類→両足獣→サル
  →人間
\item 生物の多様性。なぜこんなに多様な生物が存在するのか?
\item 博物学。リンネ(1707--1778)の分類学。生物の世界に秩序を見つけようとする。

\end{enumerate}



\section{ダーウィンの進化論}

\begin{itemize}
\item ラマルク(1744--1829)の進化論。「生物はすべて前進する」。「獲得形質の遺伝」をもとにした用不用説。 ← 否定されてる。
\item ダーウィン(1809--1882)。ガラパゴス諸島の生物の多様性の観察。フィンチという小鳥の研究。
  フィンチの種類は多様で、その嘴は、その食べ物によって違う。ほんの少しの環境の違いに生物が
  適応している。園芸家の品種改良の観察。
\item 進化論の柱
  \begin{enumerate}
  \item 遺伝。子供は親の性質に似る。
  \item 変異。生物の個体には、同じ種に属していてもさまざまな変異が見られる。
  \item 「生存のための奮闘」(struggle for existence 生存競争、生存闘争)
  \item 自然淘汰(自然選択)。生物は多産で、生き残れるより多く生む。→ 競争。
  \item 変異のなかには、生存や繁殖に影響を与えるものがある。→有利な変異の保存。
  \end{enumerate}
\item 人間と動物の共通性。動物にも心や知性がある。
\end{itemize}



\section{よくある誤解}

\begin{enumerate}

 \item ×動物は種の保存のために生殖する
 \item ×低級な動物から高等な動物に進化してきた(→すべての生物は進化の結果)
 \item ×自然淘汰はなんらかの目的にむかっている(→淘汰そのものは無目的)
 \item ×進化は進歩である(→単なる変化にすぎない)
 \item ×人間はチンパンジーから進化してきた(→共通の祖先から進化してきた)

\end{enumerate}


\section{社会進化論}
\begin{enumerate}
\item コント(1798--1857。実証主義社会学。人類の知的発展は神学的→形而上学的→実証的の三段階をたどる。
\item スペンサー(1820--1903)。万物は単純なものから複雑なものに進化する。
\item 最適者生存(survival of the fittest)。→ 弱肉強食、優勝劣敗 → 自由放任(レッセ・フェール)が最適な社会をつくる。

\item 社会進化論→ ナショナリズム。国家間、人種間、民族間の闘争を「進化論」から正当化。「適者生存」→欧米各国、特にナチス。→欧米の移民政策に影響。→ナチスドイツの民族浄化政策。
\item ← ハクスリー(1825--1895)は進化の法則に逆らうような倫理が必要と主張。
\begin{quote}
  治癒の見込みのない病人、老悴した者、虚弱な心身あるいは欠陥のある心身を有する者、過剰な新生児などは、ちょうど園芸家が欠陥のある植物や過剰な植物を間引きし飼育家が望ましくな家畜を屠殺するように、取り除かれることになろう。強くて健康な者、それもこの統治者の目的に最も適応するような子孫をつくるために注意深く縁組みされた者だけが一族を残すことを許されることになろう。(Th. ハクスリー)
\end{quote}

\end{enumerate}

\section{精神医学と優生学}

\begin{itemize}
\item フランスの精神医学者ピネル(1745--1826)。閉鎖病棟で鎖につながれている「狂人」を解放。病院の改善。
\item ガルの骨相学。
\item プリチャード「道徳的障害」の概念。
\item イタリアの医師ロンブローゾの犯罪人類学。犯罪者の身体を測定。生来性犯罪者の概念。犯罪者は隔世遺伝(先祖返り)。 → 人種差別的だと批判されるが、アイディアは犯罪学に発展。
\item ダーウィンの従兄弟ゴルトン(1822--1911)。統計学の基礎(回帰、相関など)。人類遺伝学 → 優生学。人類の遺伝的改良。→ 各国に影響。

\end{itemize}


% \bibliographystyle{eguchi}  
% \bibliography{bib}


\end{document} %----------------------------------------------------------------







%%% Local Variables:
%%% mode: japanese-latex
%%% TeX-master: t
%%% coding: utf-8
%%% End:
