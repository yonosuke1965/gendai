\ifx\mybook\undefined
\documentclass[uplatex,dvipdfmx]{jsarticle} \usepackage{mystyle}%\author{} %date{}
\title{}
\if0 %----------------------------------------------------------------

\fi  %----------------------------------------------------------------
\begin{document}
\maketitle




\else\chapter{西洋思想の導入}\fi



\section{明治}


\subsection{福沢諭吉}

啓蒙思想家。『西洋事情』『文明論之概略』。『学問のすゝめ』。実学を推奨。スピーチとディスカッションの訓練の必要性を説く。

\subsection{中江兆民}

ルソーを紹介。『三酔人経綸問答』。

\subsection{立憲君主制}

大日本帝国憲法(明治憲法)。1889。伊藤博文らがドイツ憲法を手本に起草。天皇が元首。議会によらずに行使することのできる大権をもつ。


\subsection{貧富の差}

貧民の生活の紹介。

河上肇『貧乏物語』(1916)。

\subsection{産業化と労働者}

工場法、労働者運動。

\subsection{社会主義}

片山潜、幸徳秋水らが社会民主党を設立。
人類同胞主義、軍備全廃、階級全廃、土地・資本・交通機関の公有化、分配の公平、平等な参政権、教育の無償化。即日解散を命じられる。

1906年、日本社会党設立。議会主義と直接行動論が対立。






\section{大正デモクラシー}

民主化。

女性運動。



\section{ナショナリズム}

北一輝とか。

国粋主義。

京都学派。

\section{戦後}

新憲法。








\ifx\mybook\undefined
\bibliographystyle{eguchi}
\bibliography{bib}
\end{document} %----------------------------------------------------------------
\fi






%%% Local Variables:
%%% mode: japanese-latex
%%% TeX-master: t
%%% coding: utf-8
%%% End:
