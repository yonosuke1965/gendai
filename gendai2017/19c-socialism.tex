\ifx\mybook\undefined
\documentclass[uplatex,dvipdfmx]{jsarticle} \usepackage{mystyle}%\author{} %date{}
\title{}
\if0 %----------------------------------------------------------------

\fi  %----------------------------------------------------------------
\begin{document}
\maketitle
\else\chapter{社会主義}\fi


\begin{itemize}

\item 18世紀末の都市化・工業化 → 貧富の差の拡大。
\item 資本主義の問題。貧富の差の拡大。
\item 資本家と労働者の階級対立。
\item オーエンの共産村。
\item サン・シモンの統制経済。
\item フーリエの理想的共同体の計画。
\item 平等の重視。 → 自由競争の否定、制限。生産手段の共有。
\item サン=シモンやフーリエ(仏)。 オーウェン(英)
\item → 小さな共同体での財産・生産手段の共有。

\end{itemize}


\section{初期社会主義}



\emph{ウィリアム・ゴドウィン} (1756-1836)『政治的正義』(1793)。無政府主義。富の平等な分配を要求。一夫一婦制の否定。


\emph{アンリ・ド・サンシモン}は貴族階級の出身だが、理想社会は貴族ではなく産業人(生産者)が政治的指導権を握るべきだと主張した。


\emph{シャルル・フーリエ}(1772-1837)は、各人の食欲や性欲といった本能的欲望の自由な充足が社会の目標であるとした。
そのため一夫一婦の結婚制度を否定する。400〜2000人程度の規模の共産村での生活を構想した。

成功した工場主である\emph{ロバート・オウエン}(1771-1848)は自分の工場で5歳の子供が1日13時間働いているのを見てショックを受ける。功利主義者ベンサムから影響を受け、児童労働を禁止するため工場法の成立のために尽力した。


『新社会論』。環境決定論。貧困層は道徳的にもあまりよくない状態にあるが、しかしこれは教育や環境の結果にすぎない。環境が労働者の資質を決定するので、労働者の生活・労働条件を決定する経営者に責任があると考えた。教育と環境を改善することによって性格形成・性格改善をめざす。また子どものための学校を工場に併設し、幼稚園の原型をつくった。


\begin{quotation}
子供にとって、子供の親にとって、社会にとってより良いことは、子供が12歳になり、教育を終え、要求される労働と作業に体が耐えられるようになるまで、労働を開始してはならないということである……。

人格の形成に関する国家計画は、どのような個人の立場にも関係なく、教育の近代的発展を含めなければならない。そして、帝国の国民であるすべての子供を除外してはならない。これを実現しないものは、排除すべき俯瞰用途不誠実の行動となり、社会における権利侵害となる……
% \footnote[p.247から孫引き。]{\citet{ishay04:_histor_human_right}}
。
\end{quotation}



\section{マルクス主義}

\begin{enumerate}

\item マルクス(1818--1883)
\item 資本主義社会批判。
\item 資本家階級と労働者階級の対立。
\item 労働者は労働力を\emph{搾取}されている。 → \emph{人間疎外}・労働疎外
\item \emph{唯物史観}。歴史は人類の経済活動の発展として見られる。生産力が向上すると従来の社会関係に矛盾が生じ、社会改革が必要になる。 → 歴史は階級闘争の歴史。
\item 共産主義の成立によって歴史は完成する。
\item 暴力革命も肯定。
\item → レーニン(1870--1924)。\emph{プロレタリアート独裁}。
\end{enumerate}

\subsection{ブルジョワ道徳批判}

\begin{itemize}
\item 既存の宗教、法律、道徳などは単なる歴史的・社会的立場に制約された「イデオロギー」。
\item 一般的な道徳は、ブルジョワ(中産)階級の利益になるよう構成されている。
\end{itemize}


\subsection{史的唯物論}

\begin{itemize}
\item 歴史は生産様式の変化によって区分される。
\item 生産力は歴史とととも増大する。
\item 社会革命は生産力の増大によって要求される生産関係の変化の反映。
\item 階級と階級闘争。
\end{itemize}




\subsection{共産主義}

\begin{quote}
  共産主義社会の高次の段階では、個人が分業に奴隷的に隷属することが消滅する。・・・労働が生活の手段であるだけでなく、生の第一の必要となる。個人の全面的発達とともに生産力も発達し、すべての協同的な富がいっそう豊かに溢れでるようになる。こうした後にはじめて、ブルジョワ的権利という狭い地平は踏み越えられ、社会は旗に書き込む{\――}各人からは能力に応じて、各人には必要に応じて!
\end{quote}


生産資源の共有。
「共産主義理論は一つのフレーズ、私有財産の廃止というフレーズに要約することができる」



\subsection{搾取}

\begin{itemize}
\item 資本家による労働力の\emph{搾取}が根本問題。搾取は他者の不公正な利用。
\item 資本家は労働者から、労働の対価として労働者に支払われる賃金以上の価値(剰余`価値)を生
  産物としてひきだす。
\item 労働だけが価値を創出するのであり、資本家は生産物の価値の一部をうけとる。またれゆえ労働者は創出した価値以下の価値しかうけとらない。それゆえ、労働者は資本家に搾取されている。
\item 労働者は生産財を所有しておらず、資本家のために働くよう\emph{強制}されている。

\item 「労働はあらゆる富の源泉である、と経済学者たちはいっている。しかし労働は無限になおそれ以上のものである。それは人間生活のいちばん根本の条件なのであり、しかもある意味では、労働が人間そのものを創造したといわねばならない。」

\end{itemize}

\subsection{疎外}

\begin{itemize}
\item 労働疎外。


\item 「労働はもっとも労働者自身のものでありながら、もっとも労働者が疎外されているものである」
\item 労働は人間のもっとも重要な能力。しかし賃労働によって、労働者の労働力はたんなる商品になり、資本家の統制のもとにおかれる。さらに、資本主義社会では労働者は労働において頭を使うことがなく、それゆえ本質的な満足を得られない。

  \begin{quote}
    労働者は、彼が富をより生産すればするほど、彼の生産力の力と範囲とが増大すればするほど、それだけますます貧しくなる。労働者は商品をよりおおくつくればつくるほど、それだけますます彼はより安価な商品となる。事物世界の価値増大にぴったり比例して、人間世界の価値低下がひどくなる。(『経済学・哲学草稿』)
  \end{quote}


\item 生産手段を共有することによって、労働者は自分の労働生活の編成方法について発言でき、本質的な満足を得ることが可能になる。

\end{itemize}



『ゴータ綱領批判』。能力に応じて働き、必要に応じて取る。



\section{プルードン}

「私的所有は盗みだ!」


\nocite{高晃公95:マルクス}
\nocite{大川正彦04:マルクス}
\nocite{kymlicka90:_contem_polit_philos}
\nocite{marx1848:_manif_kommun_partei:水田}
\nocite{marx1848:_commun_manif}


\section{修正社会主義、民主社会主義}

\begin{enumerate}
\item ベルンシュタインの修正社会主義。暴力革命を否定、議会主義による平和革命。

\item 民主社会主義。イギリスのフェビアン協会など。功利主義の影響。漸進的な社会改革による民主主義社会の実現を目指す。社会保障制度、産業資本の社会的管理、自由の保護。
\end{enumerate}


\ifx\mybook\undefined

\bibliographystyle{eguchi}
\bibliography{bib}


\end{document} %----------------------------------------------------------------
\fi




\section{マルクス主義}
\begin{itemize}
\item マルクスとエンゲルス(Friedrich Engels, 1830-1895)。
\end{itemize}

\subsection{ブルジョワ道徳批判}

\begin{itemize}
\item 既存の宗教、法律、道徳などは単なる歴史的・社会的立場に制約された「イデオロギー」。
\item 一般的な道徳は、ブルジョワ(中産)階級の利益になるよう構成されている。
\end{itemize}


\subsection{史的唯物論}

\begin{itemize}
\item 歴史は生産様式の変化によって区分される。
\item 生産力は歴史とととも増大する。
\item 社会革命は生産力の増大によって要求される生産関係の変化の反映。
\item 階級と階級闘争。
\end{itemize}




\subsection{搾取}

\begin{itemize}
\item 資本家による労働力の\emph{搾取}が根本問題。搾取は他者の不公正な利用。
\item 資本家は労働者から、労働の対価として労働者に支払われる賃金以上の価値(剰余`価値)を生
  産物としてひきだす。
  \item 労働だけが価値を創出するのであり、資本家は生産物の価値の一部をうけとる。またれゆえ労働者は創出した価値以下の価値しかうけとらない。それゆえ、労働者は資本家に搾取されている。
\item 労働者は生産財を所有しておらず、資本家のために働くよう\emph{強制}されている。

\item 「労働はあらゆる富の源泉である、と経済学者たちはいっている。しかし労働は無限になおそれ以上のものである。それは人間生活のいちばん根本の条件なのであり、しかもある意味では、労働が人間そのものを創造したといわねばならない。」

\end{itemize}

\section{疎外}

\begin{itemize}
\item 労働疎外。


\item 「労働はもっとも労働者自身のものでありながら、もっとも労働者が疎外されているものである」
\item 労働は人間のもっとも重要な能力。しかし賃労働によって、労働者の労働力はたんなる商品になり、資本家の統制のもとにおかれる。さらに、資本主義社会では労働者は労働において頭を使うことがなく、それゆえ本質的な満足を得られない。

  \begin{quote}
    労働者は、彼が富をより生産すればするほど、彼の生産力の力と範囲とが増大すればするほど、それだけますます貧しくなる。労働者は商品をよりおおくつくればつくるほど、それだけますます彼はより安価な商品となる。事物世界の価値増大にぴったり比例して、人間世界の価値低下がひどくなる。(『経済学・哲学草稿』)
  \end{quote}


\item 生産手段を共有することによって、労働者は自分の労働生活の編成方法について発言でき、本質的な満足を得ることが可能になる。

\end{itemize}

\subsection{共産主義}

\begin{quote}
  共産主義社会の高次の段階では、個人が分業に奴隷的に隷属することが消滅する。・・・労働が生活の手段であるだけでなく、生の第一の必要となる。個人の全面的発達とともに生産力も発達し、すべての協同的な富がいっそう豊かに溢れでるようになる。こうした後にはじめて、ブルジョワ的権利という狭い地平は踏み越えられ、社会は旗に書き込む{\――}各人からは能力に応じて、各人には必要に応じて!
\end{quote}

\begin{itemize}
\item 生産資源の共有。「共産主義理論は一つのフレーズ、私有財産の廃止というフレーズに要約することができる」
\item 暴力革命の必要性。
\item プロレタリア独裁。
\end{itemize}





\subsection{マルクス以後}

\begin{itemize}
\item 社会主義者の国際組織第一インターナショナル。(1866)
\item 無政府主義。バクーニンやクロポトキンなど。
\item 第二インターナショナル(1889)。
\item 暴力革命やプロレタリア独裁を否定する修正マルクス主義(ベルンシュタイン)→ 社会民主主義。穏健な改良主義。
\item 社会主義。フェビアン協会(1884)。社会改良、福祉厚生の向上。
\item ロシアでのソヴィエト革命(1917)。レーニン。
\item 毛沢東による中国共産党独裁(1949)。
\end{itemize}






%%% Local Variables:
%%% mode: japanese-latex
%%% TeX-master: t
%%% coding: utf-8
%%% End:
