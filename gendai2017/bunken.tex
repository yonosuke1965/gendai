\documentclass[dvipdfmx]{jsarticle}
\usepackage{okumacro,plext}
\usepackage{natbib}
\usepackage{url}
\usepackage{txfonts}
\usepackage[utf8]{inputenc}
\usepackage[T1]{fontenc}
\usepackage{otf}
%\usepackage{my_resume}
%\usepackage{graphicx,wrapfig}
%\usepackage[greek,english]{babel}
%\usepackage{teubner}
%\usepackage[dvipdfm,bookmarkstype=toc=true,pdfauthor={江口聡, EGUCHI Satoshi}, pdftitle={}, pdfsubject={},pdfkeywords={},bookmarks=false, bookmarksopen=false,colorlinks=true,urlcolor=blue,linkcolor=black,citecolor=black,linktocpage=true]{hyperref}
  \AtBeginDvi{\special{pdf:tounicode EUC-UCS2}}% platex-utf8 でも OK
% \author{江口聡}
%\date{}
\title{現代社会入門I 参考文献リスト}
\if0 %----------------------------------------------------------------

\fi  %----------------------------------------------------------------
\begin{document}
\maketitle



ホッブズ

\begin{itemize}
\item 田中浩(1998)『ホッブズ』、研究社出版。ホッブズの評伝。時代背景など非常にわかりやすい。
\item 田中浩(2006)『ホッブズ』、清水書院。同じ著者による評伝。安い。
\end{itemize}


ロック

\begin{itemize}
\item 野田又夫 (1985)『ロック』、講談社。
\item 松下圭一 (1987) 『ロック「市民政府論」を読む』、岩波書店。昭和の偉い先生による一般読者市民向け解説。
\item 浜林正夫 (1996) 『ロック』、研究社出版。この「イギリス思想叢書」シリーズはどれもよい。

\end{itemize}

ルソー

\begin{itemize}
\item 作田啓一 (1980)『ジャン‐ジャック・ルソー:市民と個人』、人文書院。
\item 桑瀬章二郎 (2010)『ルソーを学ぶ人のために』、人文書院。定評ある「学ぶ人のために」シリーズの一冊。現代の研究動向がわかる。(渡)
\item 仲正昌樹 (2011) 『今こそルソーを読みなおす』、日本放送協会出版。わかりやすい良書。
\item 重田園江 (2013) 『社会契約説:ホッブズ、ヒューム、ルソー、ロールズ』、ちくま新書。若手研究者による「社会契約」という考え方についての優れた入門書。ロールズは20世紀後半に非常に影響力のあった政治哲学者。(江)
\end{itemize}



モンテスキュー

\begin{itemize}
\item 福鎌忠恕(1975)『モンテスキュー:生涯と思想』、酒井書店。全三冊の大作だが、時代背景の中でモンテスキューの思想形成を活写して読ませる。(渡)

\end{itemize}

スミス

\begin{itemize}
\item 高島善哉 (1968) 『アダム・スミス』岩波書店〔岩波新書(青版)674〕
\item 山崎怜 (2005) 『アダム・スミス』、研究社出版。
\item 堂目卓生 (2008) 『アダム・スミス 『道徳感情論』と『国富論』の世界』、中央公論新社〔中公新書1936〕。新旧の定評ある新書。あわせて読めば日本の社会思想の変化も見える。(渡)
\end{itemize}

革命の時代

\begin{itemize}
\item 阿川尚之 (2013) 『憲法で読むアメリカ史』、ちくま学芸文庫。はじめての成文憲法をもった国アメリカで、奴隷制度や言論の自由、その他憲法をめぐってどんな戦いがおこなわれたか、現代に至るまでおもしろく読める。(江)
\item 河野健二(1959)『フランス革命小史』岩波書店。新書の中の古典。古本屋に行こう。(渡)
\item 池田理代子『ベルサイユのばら』、集英社。名作少女マンガなので必ず読みましょう。いろんな版がある。(江)
\end{itemize}

バーク

\begin{itemize}
\item C. B. マクファースン、谷川昌幸訳(1988)『バーク――資本主義と保守主義』お茶の水書房。バークについての数少ない概説書。活動の多面性と思想の変遷がよくわかる。(渡)

\end{itemize}

カント
\begin{itemize}
\item 
ハンナ・アーレント、浜田義文監訳(1987)『カント政治哲学の講義』法政大学出版局
\item 浜田義文編(1989)『カント読本』法政大学出版局。「学ぶ人のために」シリーズと並んで定評ある「読本」シリーズの一冊。20世紀の思想家による講義とあわせてどうぞ。(渡)
\end{itemize}


平等の要求

\begin{itemize}
\item 水田珠江(1973)『女性解放思想の歩み』岩波書店〔岩波新書(青版)871〕。古典。(江・渡)
\item 辻村みよ子 (1997) 『女性と人権:歴史と理論から学ぶ』、日本評論社。名著。女性解放運動について1冊読むならとりあえずこれ。(江)
\item 奥田暁子・支倉寿子・秋山洋子 (2003)『図説フェミニズム思想史:明日にむかって学ぶ歴史』、ミネルヴァ書房。3冊読むと日本のフェミニズム関係の学問の進歩がわかる。(江)
\end{itemize}

トクヴィル
\begin{itemize}
\item 富永茂樹(2010)『トクヴィル 現代へのまなざし』岩波書店。ハイブロウな新書。トクヴィルという人物の複雑さがよく伝わる。(渡)
\end{itemize}

功利主義
\begin{itemize}
\item 永井義雄 (2003) 『ベンサム』、研究社出版。基本書。(江)
\item フィリップ・スコフィールド、『ベンサム:功利主義入門』、川名雄一郎・小畑俊太郎訳、慶應義塾大学出版会。最新のベンサム研究だが読みやすくおもしろい。(江)
\item 小泉仰 (1997) 『J. S. ミル』、研究社出版。基本書。(江)
\item 児玉聡(2012)『功利主義入門―はじめての倫理学』筑摩書房〔ちくま新書967〕。ゴドウィンや19世紀の公衆衛生政策などを含めて、高校生「ジェレ美」が功利主義を学んでいく。(江・渡)
\end{itemize}

ダーウィン
\begin{itemize}
\item 内井惣七(2009)『ダーウィンの思想――人間と動物のあいだ』岩波書店。エッジのたった新書。ダーウィン思想のどこか革新的なのかがよくわかる。(渡)
\item ピエール・ダルモン (1992) 『医者と殺人者:ロンブローゾと生来性犯罪者伝説』、新評論。19世紀後半の生物学の発展などを受けて犯罪などに対する考え方も変り、「犯罪学」が成立する。そうした新しい思想はさまざまな問題を含んでいた。ロンブローゾはそうした立役者の一人。(江)
\end{itemize}

デュルケーム
\begin{itemize}
\item 作田啓一(1983)『デュルケーム』講談社。「人類の知的遺産」シリーズの一冊。著者は戦後を代表する社会学者のひとり。(渡)
\end{itemize}

マルクス/ヴェーバー

\begin{itemize}
\item 大塚久雄(1966)『社会科学の方法―ヴェーバーとマルクス―』。(渡)
\item 山之内靖(1997)『マックス・ヴェーバー入門』岩波書店。それぞれの時代を代表する入門書。解釈の変化を知るうえでも併読が望ましい。(渡)
\item 仲正昌樹 (2014) 『マックス・ウェーバーを読む』。ウェーバーの主著数冊のポイントをわかりやすく説明している。この先生の思想史ものはどれもよい。(江)
\end{itemize}

福沢

\begin{itemize}
\item 丸山真男(1986)『「文明論之概略」を読む』岩波書店。古典。福沢と『文明論之概略』をすみずみまで味わうための本。新書で三冊。(渡)
\item 堀場清子(編) (1991) 『『青鞜』女性解放論集』、岩波書店(岩波文庫)。大正期の女性解放の動き。女性論者どうしの論争がおもしろい。折井美耶子(編)『資料 性と愛をめぐる論争』、ドメス出版、も同様。(江)
\end{itemize}

ボーヴォワール

\begin{itemize}
\item  アリス・シュヴァルツァー、福井美津子訳(1994)『ボーヴォワールは語る 『第二の性』その後』平凡社。インタビューの記録。「その後」がとても大事。(渡)
\end{itemize}




\end{document} %----------------------------------------------------------------







%%% Local Variables:
%%% mode: japanese-latex
%%% TeX-master: t
%%% coding: utf-8
%%% End:
