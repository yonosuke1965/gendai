\ifx\mybook\undefined
\documentclass[uplatex,dvipdfmx]{jsarticle} \usepackage{mystyle}%\author{} %date{}
\title{産業革命、経済学の成立}
\if0 %----------------------------------------------------------------

\fi  %----------------------------------------------------------------
\begin{document}
\maketitle

\else\chapter{17世紀〜19世紀の科学}\fi


\section{17世紀〜19世紀の科学}

% \begin{tabular}[c]{l l}
%   1789 & ラヴォアジェの『化学原論』 \\
%   1790 & メートル法 \\
%   1796 & 種痘法(ジェンナー) \\
%   1799 & ヴォルタの電池 \\
% \end{tabular}

ガリレオ(1564-1643)が振り子の等時性やら躯体の法則を発見、望遠鏡による天体観測、地動説の証明。ケプラー(1571-1630)が惑星運行の法則を発見。
ハーヴェー(1578-1657)が血液の循環を発見(1628)。

1662年、イギリスで王立協会(ロイヤル・ソサエティー)が発足。1666年、フランスの科学アカデミーが発足、1699年に王立となる。


ホイヘンス(1629-1695)の振り子時計を発明、光の波動理論。

ニュートン(1642-1727)が万有引力の法則を発見(1687)。

フランクリン(1706-1790)が、雷は電気であると発見、避雷針を発明。リンネ(1707-1778)が植物分類学を確立。

ビュフォン(1707-1788)の博物誌。

ラヴォアジェ(1743-1794)が燃焼理論を確立、質量不変の法則を発見、近代化学を創始。

ヴォルタ(1745-1827)が電池を発明(1799)、電気学を創始。

ジェンナー(1749-1823)が種痘を発明(1796)。

キュヴィエ(1769-1832)の動物分類学、解剖学。

1790年にはフランスでメートル法が制定される。


\section{第一次産業革命}


% \begin{tabular}[c]{l l}
%   1709 & コークス燃料による製鉄法 \\
%   1733 & 飛び梭 \\
%   1769 & 水力紡績機 \\
%   1769 & ワットの蒸気機関 \\
%   1784 & 反射炉による製鉄法 \\
%   1785 & 力織機 \\
%   1807 & 蒸気船 \\
%   1814 & 蒸気機関車 \\
% \end{tabular}

鉄は大きな機械、あるいは精密な機械をつくるために必要だが、従来、製鉄のためには木炭が必要で、木炭を作るための樹木の不足が問題になっていたが、1709年に石炭由来のコークス燃料による製鉄法が開発され木炭不足の心配が解消された。また炭鉱開発が進む。さらに1784年に反射炉による製鉄法が開発されると鉄鋼材の大量生産が可能になった。

紡績の方面では1733年の飛び梭の発明が作業効率を発展させた。さらに1769年には水力紡績機が開発された。1785年には自動織機(力織機)が発明される。
1769年のワットが蒸気機関を改良し、1807年に蒸気船、1814年に蒸気機関車が発明される。








 18世紀、科学と技術の進歩による産業革命。綿工業などの機械化。蒸気機関の発明。機械化 → 機械を製造する機械工業、機械の原料を作る鉄鋼業、動力となる石炭を生産する石炭業などの発展。19世紀になると蒸気機関車、蒸気船などの発明。→ 交通革命。

   資本主義の成立。資本 = もとで。工場を作り経営するにはお金がかかる。資本家(金持ち)が資本を投下。労働者を雇い製品を生産し、利潤を得る。私有財産と資本主義にもとづいた産業革命と商業の発展により、国の富の急速な拡大と貧富の差の拡大がもたらされた。







\section{アダム・スミス}


\begin{itemize}

\item スコットランド啓蒙主義の中心人物のアダム・スミス(1723-1790)が、富とそれをめぐる人々の活動を考える経済学を成立させる。

\item 文明社会では互恵性が重要。自分の利益が相手の利益につうじることを示すことによって協力してもらうことができる。
  
\item 人間は他人との取引、交換、交易を好む傾向がある。
  
\item 利己的な個人が自分の利益を最大化しようとすると公共の利益になる。(マンデヴィルも昔同じことを言ってた)

\item   分業は効率がよい。人々が分業して別のものを生産し、交換するとお互いに得になる。

\item 人間は自分の利益を目標に行動する。しかし、各人が自由競争とフェアプレイという一定の制約のもとで、自分の利益をもとめ行動すると、全体としての利益に通じる。「個人は自分の自由になる資本の最も有益な使い方を見つけようと努力している。社会の利益は考えない。しかし、自分の利益への努力は、必然的に、社会にとってもっとも有益な使い道を選ばせることになる。」各人が自分の利益をもとめながら、「見えざる手\footnote{スミスは「「神の」見えざる手」は使っていない}」に導かれて全体の利益を促進することになる。「われわれが夕食を期待できるのは、肉屋、醸造屋、パン屋などの博愛精神のためではなく、かれらが自分自身の利益を重視しているからなのである。」

\item   取引・貿易はお互いの利益になる。重商主義が求めた政府による保護や規制は不要であるばかりか国家の富の形成にとって邪魔。政府の役割は、他の国家からの暴力や侵略からの国家の防衛と、社会の構成員を他の構成員による不正や抑圧から守ること。公共の利益につながる公共事業と公共機関を設立・維持する必要がある。

\item レッセフェール


\end{itemize}

\section{リカード}

スミスにつづいてデヴィッド・リカード(1772-1823)などが古典派経済学を完成。自由貿易を支持。比較優位という考え方がおもしろい。


\section{マルサス}


トマス・マルサス(1766-1834)の『人口論』は貧困の原因は、人口の増加が農業などの生産の増加より速いかであると主張、人口抑制手段(避妊など)を訴えた。





\ifx\mybook\undefined
\bibliographystyle{eguchi}
\bibliography{bib}
\end{document} %----------------------------------------------------------------
\fi




%%% Local Variables:
%%% mode: japanese-latex
%%% TeX-master: t
%%% coding: utf-8
%%% End:
