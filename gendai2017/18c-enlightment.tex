\ifx\mybook\undefined
\documentclass[uplatex,dvipdfmx]{jsarticle} \usepackage{mystyle}%\author{} %date{}
\title{啓蒙思想}
\if0 %----------------------------------------------------------------
\fi  %----------------------------------------------------------------
\begin{document}
\maketitle
\else\chapter{啓蒙主義}\fi

\label{cha:enlightment}



18世紀ヨーロッパは啓蒙時代、「啓蒙の世紀」「理性の世紀」と呼ばれる。「啓蒙」とは「蒙」(くらい)ものを「啓く」ことで、無知な人々に正しい知識を与え、合理的な考え方をするように導くことを指す。英語ではenlightment、ドイツ語ではErkl{\"a}rung 、フランス語ではilluminationとよばれ、どれも「明るくする」「光で照らす」のような意味を含んでいる。

理性による社会生活の進歩と改善を目指す。主流だったキリスト教の権威に対して批判的であり、また専制政治を批判。

サロンに啓蒙知識人が集まって議論する。ポンパドゥール夫人(1721-1764)が有名。


% \section{イギリス啓蒙}



% \subsection{マンデヴィル}



% \subsection{ハチスン}





\section{フランス啓蒙主義}

\emph{ベール}(Pierre Bayle, 1647-1706)は、神学的な歴史観や伝説に対して懐疑的な態度をとり、資料にもとづく実証的な歴史学を成立させた。

\emph{モンテスキュー}(Charles-Louis de Montesquieu, 1689-1755)は専制政治を批判し、『法の精神』で国王の専制を防ぐ手立てとして、立憲君主制と三権(立法、行政、司法)分立を提唱した。


フランス啓蒙主義の中心人物\emph{ヴォルテール}\footnote{ヴォルテールの名はペンネーム。本名はFra\c{c}ois-Marie Arouet。} (Voltaire, 1694-30)。の宗教(キリスト教)批判。大量の文章を書き、信教の自由、言論の自由、政教分離などを訴えた。「君の言うことには反対がだ、君がそれを言う権利は死んでも守る」という発言が有名。


\emph{ディドロ}や\emph{ダランベール}が中心になって人類の知識を百科事典の形で集結させようとする試みである『百科全書』が発行された。


\emph{エルヴェシウス}(Claude-Adien Helvetius, 1715-71)は快苦の原理によって人間の行動を説明し、魂は肉体の一部であって、肉体が死ねば魂も消滅すると主張。宗教者から非難される。

\emph{コンドルセ} (Marie-Jean-Antoine Nicolas de Caritat Condorcet1743-1794)。社会数学。投票行動を分析。文明の発展。狩猟→牧畜→農業→商業。「世論の法廷」



\nocite{hunt07:_inven_human_right}




 \emph{ベッカリーア}(Cesare Bonesana Beccaria, 1738-1794)は『犯罪と刑罰』で拷問、残虐な刑罰、刑罰の見世物化に反対。









\section{ルソー}


\subsection{『人間不平等起源論』}



  \begin{itemize}

\item 『人間不平等起源論』(1754)。

\item 「社会の基礎を検討した哲学者たちは皆自然状態まで遡る必要を感じたが、誰一人としてそこに到達しなかった」

\item ホッブズのような自然状態が戦争状態であるという見方はフィクション。

\item 自然人は孤立して生活し、樫の木で空腹を満たし、小川で渇きを癒し、樫の木の下で寝る。自己充足。他者を必要としない。

\item 自然人は孤立しているので社会的不平等は感じない。自由と平等。正・不正、善悪の概念もな
  し。
\item 「もっぱら自己に配慮する」自己愛(amour de soi)のみ。

\item ただし動物と共通に「同胞の苦しみを見るのを避ける生来の嫌悪感から幸福を追求する情熱を緩和する原理、憐愍 piti\'{e}をもつ」ので他人に危害を加えたりもしない。憐愍は自己愛のひとつ。

\item 自然状態では、人びとは自己愛や自尊心に加えて憐憫の情ももつ。→ 相互依存のない自足的幸福の状態。

 \item 自己愛は憐愍は、他者と比較し競争を生む悪しき「自尊心」(amour propre)とは別。

\item 自然状態ではまだ理性や知性や知識が発達していないので、欲望は単純。闘争はおこらない。

\begin{quote}
  「森の中を迷い歩き、生活技術もなく、ことばもなく、住居もなく、戦争も同盟もなく、同胞を少しも必要としないが、また彼らに危害を加えることも少しも望まず、おそらくは同胞のだれかを個人的に覚えていることすらなく、・・・わずかな情念に従うだけで、自分だけでことが足り、この情念に固有の感情と知識の光しかもっていなかった」
\end{quote}

\item 他人との接触による言語の使用や知性の発達によって、道徳感情がめばえはじめる。「自己完成能力」をもつようになる。→ 無垢な自然状態の幸福の破綻。


\item \emph{私的所有}のはじまり。「ある土地に囲いをして「これはおれのものだ」と言うことを最初に思いつき、それを信じてしまうほど単純な人びとを見つけた人こそ政治社会の真の創立者だった」

\item → 政治的不平等の起源。

\item 「富める者は、隣人を自分の目的に導くためのもっともらしい理由を容易に発明した。「弱い者たちを抑圧から守り野心を抑え、所有物を保証するために団結しよう。」粗野でおだてに乗りやすく強欲な誰もが、自分の自由を保証できると思って、自分の鉄鎖の前に駆けつけたのである。」

\item 知性の発達、欲求・欲望の増大が社会状態への発達に大きな要因となる。
\item   → 国家の成立。憐憫の情が失なわれる。国家間の戦争。

\item 富や力の不平等は自然法によって是認されるものではない。「数多くの飢えた人びとが必要なものにもこと欠くのに一握りの人たちが余分なものに満ちあふれている」のは不正。

\item 自由とは依存したり隷属状態にないこと。自立、自足していること。他の人間や欲望に支配されず\kenten{自己が自己の主人}であること。

\item 自然状態から社会状態に移行したときに、人間は「事物への依存」から「人間への依存」へ移行した。われわれは「他者の意見や判断から自己の生存感情を得る」ようになってしまっている。

  \end{itemize}

\subsection{『エミール』}

\begin{enumerate}
\item 教育論。人間を自然に成長させる、自律的で道徳的なひとにする。子どもを自然に還す。

\item 悪しき社会に住んでいる我々は自分が自然性を失なった奇形的存在になっていることに気づかない。偏見、権威、必要、慣例、習俗、制度などが自然をおしつぶつす。学問、芸術、奢侈などは自由と平等の喪失という不幸を覆い隠している。

\item 子どもの発達に応じた教育。子どもの人格と自由を尊重。 → 近代の教育論に巨大な影響。

\item 自己愛(amour de soi)と自尊心(amour propre)の区別。自己愛が悪しき自尊心に変質するのを防ぐ。

\item 情念の平衡情念の沈黙の教育。自由は自己支配(欲望と力を想像力で釣り合わせた状態)。「感覚が快いか不快か、次に自己と事物の間に適合性が認められるか、最後に理性がわれわれに与える幸福または完全性の観念にもとづきいかに判断を下すか」の三段階。「さまざまな感覚のよく規制された使用から生まれて、事物の外観と結合し事物の本性を教える」。

\item 脱自己利害。「配慮の対象が直接自分にかかわることが少なければ少ないだけ、個人的利害の感情からくる錯覚も恐れる必要がなくなる。この利害が一般化されればされるほど、ますます公正になる。この人類への愛がわれわれにおいて正義への愛に他ならない」

\end{enumerate}


\subsection{『社会契約論』}


  \begin{itemize}


  \item 『社会契約論』(1762)。「人間は生まれながらにして自由であるが、しかしいたるところで鉄鎖につながれている。ある者は他人の主人であると信じているが、事実は彼ら以上に奴隷なのだ。どうしてこの変化が生じたのか、私にはわからない。この変化を何が正当化するのか、といえば、この問題なら解くことができると思う。」

\item 自然人の平等と自由をそこなわないままに社会状態に移行させるにはどうしたらよいか?

\item 自由(⇔隷属)の不可譲性。「自己の自由の放棄は、人間の資格、人間の権利そして義務をも放棄することである。すべてを放棄したものに補償はありえない。こうした放棄は人間の本性と相容れるものではなく、意志からあらゆる自由を奪うのは、行為からあらゆる道徳性を奪うことである。」

\item 「各人がすべての人と結びつきながら、しかも自分自身にしか服従せず、以前と同様に自由のままである」ことはどうすれば可能か?

\item → \emph{自治} (autonomy、自律、自己決定自己支配)の理想。→ 政体としては共和制・民主制がベスト。

\item \emph{一般意志} volont\'{e} generale。「人民が十分な情報において討議するとき、市民が相互的になんの打ち合せもなければ、わずかの差が多く集まって一般意志が生じ、結果は常によくものであろう」

\item 「個別意志は本性において自己優先に傾き、一般意志は平等に傾く。」

\item ただし、政治的決定は個々人の利益を求める「個別意志」の総和としての「全体意志」ではなく、一個の共同体として全体の利益を求める「一般意志」によらねばならない。

\item 「人民が十分な情報において討議するとき、市民が相互間になんの打ち合わせもなければ、わずかの差が多く集まって一般意志が生じ、結果は常によきものであろう」「一般意志は誤ることがない」

\item → フランス革命に強い影響。

\end{itemize}
% \section{ルソー}

% \begin{itemize}

% \item ジャン=ジャック・ルソー (Jean-Jacques Rousseau, 1712--1778)



% \item 『人間不平等起源論』(1754)。

% \item 自然状態では、人びとは自己愛や自尊心に加えて憐憫の情ももつ。→ 相互依存のない自足的幸福の状態。


% \item 私的所有のはじまり。「ある土地に囲いをして「これはおれのものだ」と言うことを最初に思いつき、それを信じてしまうほど単純な人びとを見つけた人こそ政治社会の真の創立者だった」

% \item 「富める者は、隣人を自分の目的に導くためのもっともらしい理由を容易に発明した。「弱い
%   者たちを抑圧から守り野心を抑え、所有物を保証するために団結しよう。」粗野でおだてに乗りや
%   すく強欲な誰もが、自分の自由を保証できると思って、自分の鉄鎖の前に駆けつけたのである。」

% \item   →国家の成立。憐憫の情が失なわれる。国家間の戦争。

% \item 『社会契約説』(1762)。「人間は自由なものとして生まれた。しかもいたるところで鎖につな
%   がれている。自分た他人の主人と思っているようなものも、実はその人びと以上に奴隷なのだ。ど
%   うしてこの変化が生じたのか。何がそれを正当なものとしうるのか。私はこの問題は解きうると信
%   じる。」

% \item 自由(⇔隷属)の不可譲性。「自己の自由の放棄は、人間の資格、人間の権利そして義務をも
%   放棄することである。」

% \item 「各人がすべての人と結びつきながら、しかも自分自身にしか服従しない」ことはどうすれば可能か? 

% \item → \emph{自治} (autonomy、自律、自己決定自己支配)の理想。→ 政体としては共和制・民主制がベスト。

% \item ただし、政治的決定は個々人の利益を求める「個別意志」の総和としての「全体意志」ではな
%   く、一個の共同体として全体の利益を求める「一般意志」によらねばならない。「人民が十分な情
%   報において討議するとき、市民が相互間になんの打ち合わせもなければ、わずかの差が多く集まっ
%   て一般意志が生じ、結果は常によきものであろう」「一般意志は誤ることがない」


% \item → フランス革命に強い影響。

% \item 『エミール』(1762)。教育論の名著。自然的感情・自発性の重視。

% \end{itemize}



\section{カント}

ドイツ啓蒙主義。フリードリッヒ大王の啓蒙政策。

\subsection{「啓蒙とは何か」}

\begin{quotation}
  
  啓蒙とは、人間がみずから課した子ども状態から抜け出ることである。子ども状態とは、他人の指導なしには自分の悟性(理解力)を用いる能力がないことである。このような子ども状態の原因が悟性の欠如にではなく、他人の指導がなくとも自分の悟性を用いる決意と勇気の欠如にあるなら、子ども状態の責任は本人にある。「\emph{Sapere Aude!} あえて知ろうとせよ!/あえて賢くあれ!」「自分自身の\ruby{悟性}{あたま}を使う勇気を持て!」こそ啓蒙のモットーである。・・・未成年でいることは、たしかに気楽である。わたしのかわりの悟性をもつ本、わたしのかわりの良心をもつ牧師、わたしにかわって食事を気づかってくれる医者などがあれば、わたしはあえて自分の力を使う必要などない。支払うお金さえわたしにあれば、自分で考える必要などない。誰か他のひとがわたしのかわりに厄介なことを考えてくれるだろう。親切にも人々の後見を買って出ている保護者たちが、大多数の人々(これには女性全部が含まれる)が、成熟することを、難しいとだけでなく危険でもあると思いこむように仕組んでいるのである。まず自分の家畜たちを愚鈍にてなずけ、次にそうして従順になった家畜たちが、つなぎとめられている補助車なしには一歩も歩き出せないようにした上で、万が一、彼らが一人で歩きだそうとなどすれば、この保護者たちは危険だぞと怯えさせる。ところがこんな危険というものはたいしたものではない。なんどか転んでしまえば、けっきょくはちゃんと歩き方を学ぶことができるのだから。しかしこういった戒めでさえ、ひとびとを気おくれさせ、それ以上のことを試みなくさせるに十分なのである。(カント『啓蒙とは何か』)
\end{quotation}



\begin{itemize}


\item 国家の立法、執行、司法の権力が癒着して国民をあざむくところに専制が生じる。これは(1)他人の恣意的な支配からの自由、(2)法のまえでの平等、(3)独立した国民としての投票権の三つの資格をうばう国民にとってきわめて危険な統治方法。→特定の幸福概念を国民に押しつける、多数の専制による少数意見の抹殺などの権利侵害が起こる。

\item →「国民が自分と自分の同胞に関して決定できないことは、主権者もそれに関して決定できない」国民が自分の服する法律を自分自身で決定する以外にない。
 
\item →純粋共和制が政体として望ましい。三権分立、代表議会制。

\item 国民一人一人の自由と平等を守るためには、国民はみずからを啓蒙し、公的な言論や代議制を通して積極的に政治に参与する必要がある。
\end{itemize}

\subsection{『永遠平和のために』}

カントは『永遠平和のために』(1795) \citep{kant1795:_zum_ewigen_fried}で国際法、国際公民法を構想する。

 \begin{itemize}
\item    国家間の永遠平和のための予備条項
  \begin{enumerate}
  \item 将来の戦争の種をひそかに保留して締結された平和条約は、決して平和条約とみなされてはならない。


  \item 独立しているいかなる国家(小国であろうと大国であろうと、この場合問題ではない)も、継承、交換、買収、または贈与によって、ほかの国家がこれを取得できるということがあってはならない。


  \item 常備軍は、時と共に全廃されなければならない。


  \item 国家の対外紛争に関しては、いかなる国債も発行されてはならない。


  \item いかなる国家も、ほかの国家の体制や統治に、暴力をもって干渉してはならない。


  \item いかなる国家も、他国との戦争において、将来の平和時における相互間の信頼を不可能にしてしまうような行為をしてはならない。

  \end{enumerate}


\item 国家間の永遠平和のための確定条項

  \begin{enumerate}
\item  各国家における市民的体制は、共和的でなければならない。
\item  国際法は自由な諸国家の連合制度に基礎をおくべきである
\item 世界市民法は、普遍的な友好をもたらす諸条件に制限されなければならない。
  \end{enumerate}
 \end{itemize}

 カントの考える「共和制国家」は代表制をとり、立法府と行政府が分離している国家。一般選挙は必ずしも求められない。共和制は一般に、他の政体より平和的であると考えた。



 \section{アダム・スミス}

 \begin{itemize}
 \item アダム・スミス (1723-1790)経済学。お金、富をそれをめぐる人々の活動を考える。『国富論』(『諸国民の富』)商業社会。他人と物と取り引き・交換・交易するという人間の性質。分業して別のものを生産し、交換するとお互いに得になる。文明社会は互恵性が重要。自分の利益が相手の利益になることを示すことによって協力してもらう。人々は自分の利益を目標に行動する。しかし各人が自由に自分の利益を求めて行動すると、全体としての利益に通じる。「個人は自分の自由になる資本の最も有益な使い方を見つけようと努力している。社会の利益は考えない。しかし、自分の利益への努力は、必然的に、社会にとってもっとも有益な使い道を選ばせることになる。」各人が自分の利益をもとめながら、見えざる手に導かれて全体の利益を促進することになる。自由競争とフェアプレイが必要条件。政府による保護や既成(重商主義)は不要であるばかりか邪魔。

政府の役割
他の独立国家からの暴力や侵略からの防衛。
社会の構成員を他の構成員による不正や抑圧から守る。→ 厳正な司法制度の必要。
特定の公共事業と公共機関を設立・維持する。
 \end{itemize}





\ifx\mybook\undefined
\bibliographystyle{eguchi}
\bibliography{bib}
\end{document} %----------------------------------------------------------------
\fi






%%% Local Variables:
%%% mode: japanese-latex
%%% TeX-master: t
%%% coding: utf-8
%%% End:
