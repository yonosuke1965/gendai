\ifx\mybook\undefined
\documentclass[uplatex,dvipdfmx]{jsarticle}
\usepackage{okumacro,plext}
\usepackage{natbib}
\usepackage{url}
\usepackage{txfonts}
\usepackage[utf8]{inputenc}
\usepackage[T1]{fontenc}
\usepackage{otf}
%\usepackage{my_resume}
%\usepackage{graphicx,wrapfig}
%\usepackage[greek,english]{babel}
%\usepackage{teubner}
%\usepackage[dvipdfm,bookmarkstype=toc=true,pdfauthor={江口聡, EGUCHI Satoshi}, pdftitle={}, pdfsubject={},pdfkeywords={},bookmarks=false, bookmarksopen=false,colorlinks=true,urlcolor=blue,linkcolor=black,citecolor=black,linktocpage=true]{hyperref}
  \AtBeginDvi{\special{pdf:tounicode EUC-UCS2}}% platex-utf8 でも OK
\author{江口聡}
%\date{}
\title{なぜ社会思想を学ぶか}
\if0 %----------------------------------------------------------------

\fi  %----------------------------------------------------------------
\begin{document}
\maketitle

\else

\chapter{なぜ社会思想を学ぶか}

\fi


若い人々には気づきにくいことだが、人間の社会は大きく変化してきた。そしてこれからも社会は変わる!

大学生・市民としての基礎教養。大学4年間での学習・研究のため、このテキスト程度のことは理解しておくこと。

公務員試験、教職、入社試験 → このテキスト程度で「人文」教養の6、7割とれることを目指す。

授業ではだいたい古代ギリシアから20世紀初頭までの西洋思想を扱う。


\section{問題意識}


「自由」「平等」「正義」「人権」「人民主権」「民主主義」「生命尊重」「選挙権」「福祉国家」といったさまざまな概念とその由来。
国家はなんのためにあるのか?

社会のルールはどのようにして決まるのか? どのようにして決める\kenten{べき}か。

社会的動物としての人間とはどのような存在か。

反省する動物としての人間。人間は自己と社会について考える存在。

将来を構想するもの・環境を変えるものとしての人間。どういう世界が望ましいか?

  歴史の重要性。歴史のある時点から、人間は、人間とはどんなものであるか、どうあるべきか、社会の成り立ちはどんなものであるか、どうあるべきか、といったことを考えはじめる。

  社会状況が思想を生み、誰かの思想が社会状況を変革する。

  新しい発想や考え方が出てきたときに、\kenten{それまでに存在しなかったもの}を考えてみる。

方針としては、断片でもいいので、なるべく思想家たちの生の言葉を味わって欲しい。




\section*{勉強ティプス}


 人の名前は覚えにくい。肖像画や写真といっしょにすると覚えやすい。綴り字も一応確認しておくと恥ずかしい読み間違いが減る。人名、出来事等はWikipediaなどで確認する癖をつけたい。
 年代はだいたいで覚える。細かいのは無理。
 世界史の教科書を手元に置いておく。
 年表を自分で作ってみるのもよい。





\section{おすすめ本}
\begin{itemize}
\item 水田洋、『社会思想小史』、ミネルヴァ書房、2006。
\item リン・ハント、『人権を創造する』、岩波書店、2011。
\item ミシェリン・イシェイ、『人権の歴史』、明石書店、2008。

\end{itemize}

\ifx\mybook\undefined


% \bibliographystyle{eguchi}  
% \bibliography{bib,library}


\end{document} %----------------------------------------------------------------




\fi


%%% Local Variables:
%%% mode: japanese-latex
%%% TeX-master: t
%%% coding: utf-8
%%% End:
