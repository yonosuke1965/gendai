\ifx\mybook\undefined
\documentclass[uplatex,dvipdfmx]{jsarticle} \usepackage{mystyle}%\author{} %date{}
\title{}
\if0 %----------------------------------------------------------------

\fi  %----------------------------------------------------------------
\begin{document}
\maketitle


\else\chapter{ユダヤ・キリスト教・イスラーム}

\fi
\section{ユダヤ教}


\subsection{ユダヤ民族の宗教}


\begin{itemize}
\item 当時の他の多くの地域では\emph{多神教}が普通。ユダヤ教の特異さ = \emph{一神教}。
  神は一人だけ。人格をもった神(人格神)。
\item 律法=神との契約。民族の祖アブラハムが神と最初の契約を結ぶ。
\item モーセに戒律(十戒)が与えられる。
\item 預言者(神の言葉を聞いた者)と呼ばれる宗教・政治的リーダー。
\item 戒律の重視。→ 戒律を守るのが「ユダヤ人」。選民思想。
\item 旧約聖書\footnote{ただし「旧約」「新約」という呼び方はキリスト教からのもの。}は歴史書であると同時に神とユダヤ民族の契約の書。

\end{itemize}

\section{イエス}


\begin{itemize}

\item イエス\footnote{学術的な文脈では、「キリスト」とは呼ばないことが多い。}の革新的思想。
「山上の垂訓」は必ず読んでおくこと。
\item 黄金律(the Golden Rule)「人にしてもらいたいと思うことは何でも、あなたがたも人にしなさい。」(マタイ12:7)
\item 新約聖書。イエスとその弟子たちの言行録と手紙。イエス自身が新しい宗教を確立したわけではない。
パウロ中心に教義を完成。イエスによって神との契約が成就されたと解釈。
\item 「神の王国」という思想。神が王として世界を支配する。
\item パウロ。「イエスはキリストであり、人類の罪を贖った。」→ 「キリスト教」の成立。原罪、贖罪、復活、三位一体などの教義。
\item 隣人愛。
\item 性的禁欲。
\item 「神の国」という思想。神が王として世界を支配する。
\item ローマ帝国で迫害される(当時のローマは多神教)。殉教。
\item → 4世紀はじめに公認 → 後半に国教に。

\end{itemize}


\section{中世キリスト教}

\begin{itemize}
\item 古代ギリシア思想と融合。
\item 4-5世紀。アウグスティヌス。「原罪」の強調。
\item 理性と信仰。対立?  → ずっと問題。信仰の方が上位にあるという考え方が主流。
\item 禁欲的。
\item 西洋国家と宗教。ローマ教皇が皇帝・国王を任命。
\item ローマ教皇の権威は12世紀ごろ頂点。十字軍派遣。
\end{itemize}





\section{イスラーム}


\begin{itemize}

\item ムハンマドが最後の預言者。(神ではない。信仰の対象はあくまで唯一神。)
\item クルアーン(コーラン)は神からの啓示。
\item 最近は「イスラム教」とは呼ばない。イスラームは単なる宗教ではなく、社会制度・政治経済まで含めた広い概念。
\item \emph{六信}。アッラー、啓典、預言者、天使、最後の審判、天命。

\item 唯一神信仰。ユダヤ教やキリスト教と共通。絶対帰依。
\item モーセやイエスも預言者の一人。(イエスは預言者であって神ではない)
\item 五行。\ruby{信仰告白}{シャハーダ}、\ruby{礼拝}{サラート}、\ruby{喜捨}{ザカート}、\ruby{断食}{サウム}、\ruby{巡礼}{ハッジ}。
\item ウンマ=ムスリムの信仰共同体。
\item 飲酒禁止。豚肉禁忌。異教徒が殺した動物も×。利子をとることの禁止。
\item 政教一致。
\item 結婚に際しての明示的契約。一夫多妻を許容。チャドルの使用。左手は不浄。
\item 商業の肯定。(キリスト教は否定的)
\item カリフの座をめぐっての争いで分裂。世界史の教科書でシーア派の由来など確認しておくこと。
\item 8世紀、イスラーム帝国。学術も発展。
\end{itemize}



\section{ローマ帝国とキリスト教}

\begin{itemize}
\item キリスト教はローマ帝国で4世紀はじめに公認。4世紀後半には国教化。
\item 4世紀後半のゲルマン民族の大移動 → ローマ帝国の領土縮小。
\item 4世紀後半、西ローマ帝国滅亡。ゲルマン民族のフランク王国成立 → キリスト教に改宗。
\item 9世紀初頭、カール大帝が西ローマ帝国皇帝に戴冠。
\end{itemize}







\section{封建社会}

\begin{itemize}
\item ローマの恩貸制度。家臣の奉仕を義務に、土地の使用権を貸し与える。
\item ゲルマン国家の従士制。貴族に仕えることによって衣食や武器を給与される。
\item → 中世封建領主。諸侯や騎士は独立した領主として土地を支配・経営。契約によって他の領主に仕える。
\item 諸侯・騎士どうしの関係。家臣は主君に戦争その他での忠節を近い、主君は家臣に土地を与える。
\item 土地には農奴(不自由農民、賦役や貢租を求められる)が付属する。
\item 教会権力と世俗権力の両立。世俗的権力を神の名によって正当化・権威づけ。皇帝や国王は異教徒の侵入からキリスト教世界を防衛する。
\item 次第に世俗権力と教会権力は対立するようになる。
\item 11世紀からの十字軍 → 世俗(国王)権力の増大
\item → 中世封建社会は次第に動揺。
\end{itemize}



\section{キリスト教の国家論}

\subsection{アウグスティヌス}

\begin{itemize}
\item ヒッポのアウグスティヌス(354-430)。キリスト教の教義の土台を据えた「教父」の一人。『神の国』など。
\item 人間はアダムとイブから遺伝している「原罪」を背負った罪人。イエス・キリストが罪を許し\ruby{贖}{あがな}う。
\item 『神の国』では「地の国」と「神の国」との対立が描かれる。神の国は人類の歴史が終るときに成就する。世俗国家(地の国)は神の国の理念に反してはならず、教会を見習うべき。
\item → キリスト教の権威が世俗の権威(皇帝権など)を正当化するという思想。

\end{itemize}


\subsection{トマス・アクィナス}
\begin{itemize}
\item トマス・アクィナス(1225-74)。
\item 自然法思想。ローマ時代から存在していた考え方で、人間が定めた法(実定法)に先行して、自然の法が存在しているという思想。
\item トマスは神の永遠の法が人間に示されたときに自然法と呼ばれると考える。人間が定める実定法は、神の永遠の法にしたがって共通善を目指さねばならない。
\end{itemize}



\if0

\section{旧約・新訳聖書から}



\subsection*{『出エジプト記』第20章から(モーセの十戒)}

神はこれらすべての言葉を告げられた。「わたしは主、あなたの神、あなたをエジプトの国、奴隷の家から導き出した神である。あなたには、わたしをおいてほかに神があってはならない。あなたはいかなる像も造ってはならない。上は天にあり、下は地にあり、また地の下の水の中にある、いかなるものの形も造ってはならない。あなたはそれらに向かってひれ伏したり、それらに仕えたりしてはならない。わたしは主、あなたの神。わたしは熱情の神である。わたしを否む者には、父祖の罪を子孫に三代、四代までも問うが、わたしを愛し、わたしの戒めを守る者には、幾千代にも及ぶ慈しみを与える。あなたの神、主の名をみだりに唱えてはならない。みだりにその名を唱える者を主は罰せずにはおかれない。安息日を心に留め、これを聖別せよ。六日の間働いて、何であれあなたの仕事をし、七日目は、あなたの神、主の安息日であるから、いかなる仕事もしてはならない。あなたも、息子も、娘も、男女の奴隷も、家畜も、あなたの町の門の中に寄留する人々も同様である。六日の間に主は天と地と海とそこにあるすべてのものを造り、七日目に休まれたから、主は安息日を祝福して聖別されたのである。あなたの父母を敬え。そうすればあなたは、あなたの神、主が与えられる土地に長く生きることができる。殺してはならない。姦淫してはならない。盗んではならない。隣人に関して偽証してはならない。隣人の家を欲してはならない。隣人の妻、男女の奴隷、牛、ろばなど隣人のものを一切欲してはならない。」



\subsection*{マタイによる福音書 5章から}

イエスはこの群衆を見て、山に登られた。腰を下ろされると、弟子たちが近くに寄って来た。
そこで、イエスは口を開き、教えられた。


\subsubsection*{姦淫してはならない}



「あなたがたも聞いているとおり、『姦淫するな』と命じられている。しかし、わたしは言っておく。みだらな思いで他人の妻を見る者はだれでも、既に心の中でその女を犯したのである。もし、右の目があなたをつまずかせるなら、えぐり出して捨ててしまいなさい。体の一部がなくなっても、全身が地獄に投げ込まれない方がましである。もし、右の手があなたをつまずかせるなら、切り取って捨ててしまいなさい。体の一部がなくなっても、全身が地獄に落ちない方がましである。」

\subsubsection*{復讐してはならない}

「あなたがたも聞いているとおり、『目には目を、歯には歯を』と命じられている。しかし、わたしは言っておく。悪人に手向かってはならない。だれかがあなたの右の頬を打つなら、左の頬をも向けなさい。あなたを訴えて下着を取ろうとする者には、上着をも取らせなさい。 だれかが、一ミリオン行くように強いるなら、一緒に二ミリオン行きなさい。求める者には与えなさい。あなたから借りようとする者に、背を向けてはならない。」

\subsubsection*{敵を愛しなさい}

「あなたがたも聞いているとおり、『隣人を愛し、敵を憎め』と命じられている。しかし、わたしは言っておく。敵を愛し、自分を迫害する者のために祈りなさい。あなたがたの天の父の子となるためである。父は悪人にも善人にも太陽を昇らせ、正しい者にも正しくない者にも雨を降らせてくださるからである。自分を愛してくれる人を愛したところで、あなたがたにどんな報いがあろうか。徴税人でも、同じことをしているではないか。自分の兄弟にだけ挨拶したところで、どんな優れたことをしたことになろうか。異邦人でさえ、同じことをしているではないか。だから、あなたがたの天の父が完全であられるように、あなたがたも完全な者となりなさい。」


\subsubsection*{施しをするときには}

「見てもらおうとして、人の前で善行をしないように注意しなさい。さもないと、あなたがたの天の父のもとで報いをいただけないことになる。だから、あなたは施しをするときには、偽善者たちが人からほめられようと会堂や街角でするように、自分の前でラッパを吹き鳴らしてはならない。はっきりあなたがたに言っておく。彼らは既に報いを受けている。施しをするときは、右の手のすることを左の手に知らせてはならない。あなたの施しを人目につかせないためである。そうすれば、隠れたことを見ておられる父が、あなたに報いてくださる。」


\subsubsection*{神と富}



「だれも、二人の主人に仕えることはできない。一方を憎んで他方を愛するか、一方に親しんで他方を軽んじるか、どちらかである。あなたがたは、神と富とに仕えることはできない。」

\subsubsection*{思い悩むな}

「だから、言っておく。自分の命のことで何を食べようか何を飲もうかと、また自分の体のことで何を着ようかと思い悩むな。命は食べ物よりも大切であり、体は衣服よりも大切ではないか。空の鳥をよく見なさい。種も蒔かず、刈り入れもせず、倉に納めもしない。だが、あなたがたの天の父は鳥を養ってくださる。あなたがたは、鳥よりも価値あるものではないか。あなたがたのうちだれが、思い悩んだからといって、寿命をわずかでも延ばすことができようか。なぜ、衣服のことで思い悩むのか。野の花がどのように育つのか、注意して見なさい。働きもせず、紡ぎもしない。しかし、言っておく。栄華を極めたソロモンでさえ、この花の一つほどにも着飾ってはいなかった。今日は生えていて、明日は炉に投げ込まれる野の草でさえ、神はこのように装ってくださる。まして、あなたがたにはなおさらのことではないか、信仰の薄い者たちよ。だから、『何を食べようか』『何を飲もうか』『何を着ようか』と言って、思い悩むな。それはみな、異邦人が切に求めているものだ。あなたがたの天の父は、これらのものがみなあなたがたに必要なことをご存じである。何よりもまず、神の国と神の義を求めなさい。そうすれば、これらのものはみな加えて与えられる。だから、明日のことまで思い悩むな。明日のことは明日自らが思い悩む。その日の苦労は、その日だけで十分である。」

\subsubsection*{人を裁くな}

「人を裁くな。あなたがたも裁かれないようにするためである。あなたがたは、自分の裁く裁きで裁かれ、自分の量る秤で量り与えられる。あなたは、兄弟の目にあるおが屑は見えるのに、なぜ自分の目の中の丸太に気づかないのか。兄弟に向かって、『あなたの目からおが屑を取らせてください』と、どうして言えようか。自分の目に丸太があるではないか。偽善者よ、まず自分の目から丸太を取り除け。そうすれば、はっきり見えるようになって、兄弟の目からおが屑を取り除くことができる。神聖なものを犬に与えてはならず、また、真珠を豚に投げてはならない。それを足で踏みにじり、向き直ってあなたがたにかみついてくるだろう。」


\subsubsection*{金持ちの青年(マタイ19章16-26)}

\label{sec:1916-26}

さて、一人の男がイエスに近寄って来て言った。「先生、永遠の命を得るには、どんな善いことをすればよいのでしょうか。」イエスは言われた。「なぜ、善いことについて、わたしに尋ねるのか。善い方はおひとりである。もし命を得たいのなら、掟を守りなさい。」男が「どの掟ですか」と尋ねると、イエスは言われた。「『殺すな、姦淫するな、盗むな、偽証するな、父母を敬え、また、隣人を自分のように愛しなさい。』」そこで、この青年は言った。「そういうことはみな守ってきました。まだ何か欠けているでしょうか。」イエスは言われた。「もし完全になりたいのなら、行って持ち物を売り払い、貧しい人々に施しなさい。そうすれば、天に富を積むことになる。それから、わたしに従いなさい。」青年はこの言葉を聞き、悲しみながら立ち去った。たくさんの財産を持っていたからである。イエスは弟子たちに言われた。「はっきり言っておく。金持ちが天の国に入るのは難しい。重ねて言うが、金持ちが神の国に入るよりも、らくだが針の穴を通る方がまだ易しい。」

弟子たちはこれを聞いて非常に驚き、「それでは、だれが救われるのだろうか」と言った。イエスは彼らを見つめて、「それは人間にできることではないが、神は何でもできる」と言われた。すると、ペトロがイエスに言った。「このとおり、わたしたちは何もかも捨ててあなたに従って参りました。では、わたしたちは何をいただけるのでしょうか。」イエスは一同に言われた。「はっきり言っておく。新しい世界になり、人の子が栄光の座に座るとき、あなたがたも、わたしに従って来たのだから、十二の座に座ってイスラエルの十二部族を治めることになる。わたしの名のために、家、兄弟、姉妹、父、母、子供、畑を捨てた者は皆、その百倍もの報いを受け、永遠の命を受け継ぐ。口語訳を見るしかし、先にいる多くの者が後になり、後にいる多くの者が先になる。」

\subsubsection*{善いサマリア人 (ルカによる福音書 10:25-37)}


すると、ある律法の専門家が立ち上がり、イエスを試そうとして言った。「先生、何をしたら、永遠の命を受け継ぐことができるでしょうか。」イエスが、「律法には何と書いてあるか。あなたはそれをどう読んでいるか」と言われると、彼は答えた。「『心を尽くし、精神を尽くし、力を尽くし、思いを尽くして、あなたの神である主を愛しなさい、また、隣人を自分のように愛しなさい』とあります。」イエスは言われた。「正しい答えだ。それを実行しなさい。そうすれば命が得られる。」しかし、彼は自分を正当化しようとして、「では、わたしの隣人とはだれですか」と言った。イエスはお答えになった。「ある人がエルサレムからエリコへ下って行く途中、追いはぎに襲われた。追いはぎはその人の服をはぎ取り、殴りつけ、半殺しにしたまま立ち去った。ある祭司がたまたまその道を下って来たが、その人を見ると、道の向こう側を通って行った。同じように、レビ人もその場所にやって来たが、その人を見ると、道の向こう側を通って行った。ところが、旅をしていたあるサマリア人は、そばに来ると、その人を見て憐れに思い、近寄って傷に油とぶどう酒を注ぎ、包帯をして、自分のろばに乗せ、宿屋に連れて行って介抱した。そして、翌日になると、デナリオン銀貨二枚を取り出し、宿屋の主人に渡して言った。『この人を介抱してください。費用がもっとかかったら、帰りがけに払います。』さて、あなたはこの三人の中で、だれが追いはぎに襲われた人の隣人になったと思うか。」律法の専門家は言った。「その人を助けた人です。」そこで、イエスは言われた。「行って、あなたも同じようにしなさい。」


\subsubsection*{ヨハネによる福音書 8:3-11}


そこへ、律法学者たちやファリサイ派の人々が、姦通の現場で捕らえられた女を連れて来て、真ん中に立たせ、イエスに言った。「先生、この女は姦通をしているときに捕まりました。こういう女は石で打ち殺せと、モーセは律法の中で命じています。ところで、あなたはどうお考えになりますか。」イエスを試して、訴える口実を得るために、こう言ったのである。イエスはかがみ込み、指で地面に何か書き始められた。しかし、彼らがしつこく問い続けるので、イエスは身を起こして言われた。「あなたたちの中で罪を犯したことのない者が、まず、この女に石を投げなさい。」そしてまた、身をかがめて地面に書き続けられた。これを聞いた者は、年長者から始まって、一人また一人と、立ち去ってしまい、イエスひとりと、真ん中にいた女が残った。イエスは、身を起こして言われた。「婦人よ、あの人たちはどこにいるのか。だれもあなたを罪に定めなかったのか。」女が、「主よ、だれも」と言うと、イエスは言われた。「わたしもあなたを罪に定めない。行きなさい。これからは、もう罪を犯してはならない。」〕





\section{さらに学習するために}


\fi



\ifx\mybook\undefined

\bibliographystyle{eguchi}
\bibliography{bib}


\end{document} %----------------------------------------------------------------

\fi





%%% Local Variables:
%%% mode: japanese-latex
%%% TeX-master: t
%%% coding: utf-8
%%% End:
