
\chapter{現代の議論}

\section{ハイエク}


\section{バーリン}






\section{福祉リベラリズム}


\subsection{公正としての正義}
\begin{itemize}
\item ジョン・ロールズの『正義論』(1971)
\item 人間にとってなにが「善」であるかは公共的な立場からは解決不可能。→ なにが当人にとって
「善」であるかは当人の判断にまかさられねばならない。

\item 法や社会制度を考える上では「善」ではなく「正しさ」を問題にするべき。
(「善」に対する「正」の優位)
\item 正義こそが社会制度の第一の徳目。
\item 社会とは各人がそれぞれの善を追求する「相互利益を求める協働事業」。
\item 利害の不一致がある状況で、社会的な便益を適性に分配するにはどうしたらよいか?原理が必要。
\item 自由で平等な個人からなる「原初状態」を仮想してみる。各人は合理的だが、自分自身の属性を知らないと仮定(無知のヴェール)。
\item 無知のヴェール。自分の現実の社会的階層、社会的地位、性別、資産、精神的・肉体的能力を
  知らないとする。
\end{itemize}

\subsection{正義の二原理}

\begin{itemize}

  \item 各人は、すべての人びとにとっての同様な自由な体系と両立しうる最大限の自由への平等な権利を持たなければならない。
  \item 社会的・経済的不平等は、次の二つの条件を満たすように配置されなければならない。
\begin{enumerate}
    \item 最も恵まれない人びとの最大限の利益となるように
    \item 公正な機会の均等という条件の下ですべての人に開かれた職務や地位にのみともなうように。
    \end{enumerate}

  \begin{quote}
    福祉国家的資本主義は、財産や技能の当初の分配を所与の実質的な不平等として容認し、その上
    で事後的に所得を再配分しようとする。これにたいして財産所有の民主主義は、財産や技能の当
    初の分配における平等を求めるが、所得の再配分措置にたいしては、それほど重きをおかない。
  \end{quote}

\item 功利主義者は所有権に二次的な価値しか認めない。あくまで効用(人びとの幸福)の最大化が目標。

\end{itemize}


\section{リバタリアニズム(自由優先主義)}

\begin{itemize}

\item ロバート・ノージックの『アナーキー・国家・ユートピア』(1974)
\item ノージックのような自由優先主義(リバタリアン)によれば、暴力や詐欺のような不正な手段
  によらずに、正当に獲得した財産であれば、自由に使ってもよいとされる。私的所有権はほぼ絶
  対。
  \item 人びとは自分自身を所有する。
  \item 世界は原初的には誰のものでもない。
  \item 世界にはたらきかけ、自分自身の労働によって獲得したものはその人のものである。
  \item 他者の状態を悪くしないかぎり、世界の均等ではない取り分にたいする絶対的権利の獲得が可能である。
%   \item 世界の均等ではない取り分にたいする絶対的権利の獲得は比較的容易である。
   \item 一度、人びとが私有財産を占有すれば、資本や労働の自由市場が道徳的に要請される。

   

\end{itemize}


\section{共同体主義}




\nocite{nozick74:_anarc_state_and_utopia}
\nocite{有福孝岳99:エチカ,坂井明宏07:現代倫理学}
\nocite{rawls71:_theor_of_justic}
\nocite{川本隆史95:現代倫理学の冒険}
