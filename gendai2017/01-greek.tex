\ifx\mybook\undefined
\documentclass[uplatex,dvipdfmx]{jsarticle} \usepackage{mystyle}%\author{} %date{}
\title{原始国家〜古代ギリシア}
\if0 %----------------------------------------------------------------

\fi  %----------------------------------------------------------------
\begin{document}
\maketitle
\else

\chapter{原始社会}


\fi






\section{原始社会から国家へ}

\begin{enumerate}

\item 狩猟採集生活。原始共産社会。親族・血族を中心にした集団生活。部族・氏族社会。
\item 遊牧・農耕の開始 → 土地の私有。
\item → 余剰食料 → 貧富の差の発生。
\item → 分業のはじまり → 原始国家。
\item 王や貴族階級の成立。
\item 他氏族を征服し奴隷化する。奴隷制。
\item 支配従属関係が発生する。
\end{enumerate}


\section{古代文明}

\begin{itemize}
\item 紀元前3000年ごろからメソポタミアで最初の都市文明。
\item 治水・灌漑など。\emph{神権政治}。王、神官、役人、戦士など階級社会。
\item 紀元前18世紀ごろハンムラビ王。
\item 文字が発明されると、法律や慣習が成文化される。紀元前1700年ごろに『ハンムラビ法典』が成立。同害刑法「目には目を」などが有名。
\item 紀元前3000年ごろエジプトではファラオによる統一国家。やはり専制的神権政治。
\end{itemize}


\ifx\mybook\undefined
\else
\chapter{古代ギリシア}\fi



\section{古代ギリシアのポリス社会}

\subsection{ポリス}

\begin{itemize}
\item ギリシア。紀元前8世紀ごろ、人々が集住して都市「\emph{ポリス}」ができる。
\item 古代ギリシアの\emph{ポリス}国家。城壁で囲まれた都市が一つの国家。
\item 「国家」とは? 現代的に考えると、(1) 土地、(2)人民、(3)主権をもっている人の集団が国家。「主権」とは他の国から独立に意思決定をおこなうこと。それ以上の決定組織がない。国民から税金を集めたり、戦争したりする決定力をもっているのが国。
\item 各ポリスによって制度・法律はさまざま。\emph{王政}の国(王が最終的な決定権をもつ)、\emph{共和制}(王が存在せず、国家元首を直接・間接に選出したり、複数の代表者を置いたりする)・\emph{民主制}(共和制のなかでも、国民全体の意志にもとづく政治をおこなう)の国など。
アテネのライバル国家のスパルタは王政(貴族制)。

\end{itemize}

\subsection{アテネの民主制}
\begin{itemize}
\item アテネの民主制が有名で世界史的に重要とされている。
\item 紀元前5世紀はアテネの黄金時代。
\item 人口20万程度の(当時の)大都市。奴隷もいた(20万人中の8万人ぐらいか)。
\item 「市民」(国民)とされるのは財産をもった成人男性。戦争になれば自分で装備して出兵する。中上流家庭の女性は家の外に出なかった。
\item 621B.C. ドラコン法(慣習法を成文化)。
→ 僭主(独裁者)の登場 → 陶片追放(オストラシズム)制度で潜主の再登場を避けようとする。
\item ペリクレス時代(443 B.C.〜 429 B.C.)。ペリクレスは民主制を称えた演説でも有名。
\item 直接民主制、官職は抽選。参政権は自由市民男子のみ。
\item 裁判も選出された裁判員の多数決。
\item 扇動政治家(デマゴーク)が出現することもあった。 → 「デマ」の語源。
\end{itemize}


\section{ソフィストたち}

\begin{itemize}
\item ポリス社会での「弁論」の重要性。投票で政治を決め、裁判も投票制なので弁論の技術が大事になる。
\item → 職業的教師\emph{ソフィスト}(=知者)の登場。弁論術(詭弁術)を教える。
\item プロタゴラス「人間は万物の尺度」。
\item \emph{ノモス}(法、道徳)と\emph{ピュシス}(自然)の区別。法や道徳などの社会のルール
(ノモス)は人間が作りだしたものにすぎない。(→ だから我々は弱肉強食の自然のルール(ピュシス)にしたがって、ズルできるときはするべきだ)といった主張がなされる。

\item → \emph{相対主義}。善悪、正・不正は社会によって異なるのだ。「本当に正しいこと」などは存在せず、力の対立があるだけだけ。我々は知恵と力をつかって他人を支配しなければならない。


\end{itemize}




\section{ソクラテス}


\begin{itemize}
\item 「汝自身を知れ」「ただ生きることではなくよく生きることが重要だ」
\item 知への愛 → 哲学 → 19世紀ごろまで学問はなんでも「哲学」。
\item 友人がデルフォイの神殿で「一番賢いのはソクラテスだ」という神託を受ける。「自分は賢い」と公言している人々(ソフィストや政治家たち)は本当に賢いのか?
たとえば「善とは何か」「正義とは何か」を本当に知っているのか?



\item 「吟味されない生は生きるに値しない」(『弁明』、38A))



  \begin{quote}
    人間にとっては、徳その他のことについて、毎日談論するという、このことが、まさに最大の善きことなのであって、わたしがそれらについて、問答しながら、自分と他人を吟味しているのを、諸君は聞かれているわけであるが、これに反して、吟味のない生活は、人間の生きる生活ではないと、こう言っても、わたしがこう言うのを、諸君はなおさら信じないであろう。しかしそのことは、まさにわたしの言うとおりなのだ、諸君。ただそれを信じさせることが、容易ではないのです。(38a)
  \end{quote}

\item 問答法(対話法)。 → ソクラテス的方法 Socratic method。対話・問答によって真理を発見する。(しかしたいてい喧嘩別れ。はっきりした答はなかなか出てこない。)

  \begin{quote}
     わたしとは、どんな人間であるかといえば、もしわたしの言っていることに何か間違いでもあれば、こころよく反駁を受けるし、他方また、ひとの言っていることに何か本当でない点があれば、よろこんで反駁するような、とはいっても、反駁を受けることが、反駁することに比べて、少しも不愉快にならないような、そういう人間なのです。なぜなら、反駁を受けることの方が、より大きな善であるとわたしは考えているからです。(『ゴルギアス』 458A )
  \end{quote}


\item 「無知の知」。知恵のある者(ソフィスト)ではなく、知恵を愛する者、知恵を求める者(phil 愛 + sophia 知恵 → philosophoi 哲学者\footnote{明治期に西周は当初philosophyを、哲を求める学問として希哲学と訳した。})。以降学者は自然科学者も含めてすべて「哲学者」を自称することになる。博士号 Ph.~D.は Philosophiae Doctor 「哲学博士」の略号。

\item 子どもに、その子どもにとって善いものを与えることと、子どもが欲するものを与えることはまったく違う。 → 「善い」ことは「本人が欲すること」とは違う。

\item 自由とは欲望のままに行動することではない。独裁者になって他人を殺すより、正しく生きて独裁者に殺される方が幸福でありよく生きている。

\item (87C以下)健康、強さ、美しさ、富などは一般に有益であるが、時に害を与えることがある。それを\kenten{正しく} 使用するときに有益。節制、正義、勇気、理解力、寛大さなども知性がともなわなければ有害なことがある。

\item   → したがって徳は知。徳(aret\={e}, virtue)の基本の意味は「すぐれてある」こと、卓越性、能力。悪徳は無知による。

\item ソクラテスは新しい神を祭り若者に悪影響を与えるとして裁判にかけられ、死刑を宣告される。友人たちが脱獄をすすめるが、断わり毒を飲んで死ぬ。

\end{itemize}

\section{プラトン}

\begin{itemize}
\item ソクラテスの弟子の一人。ソクラテスの言行を書き残す(他にも政治家クセノポンや喜劇作家アリストパネスなどもソクラテスの言行を書き残している)。『ソクラテスの弁明』『クリトン』『ゴルギアス』など。
\item 学園アカデメイアを創設する。
\item 『国家』では自身の立場で政治を探究。正義とそれを実現する理想的な国家のあり方を考える。
\item 民主制(デモクラティア)を強烈に批判。ソクラテスを刑死させたように、(劣った)大衆の判断は当てにならない。
\item 人間には生まれつきの素質の上下がある。
\item 人間の魂が欲望と意志と理性に分けられるように、国家も生産者と防衛者と統治者に分けられる。
\item 統治者と防衛者は国家のために私的な欲望を捨てなければならない。→ 財産は持たない → 子どもは家族ではなく国家が養育する。
\item 統治者は子どものころから特別な教育を受ける必要がある。
\item 哲人王が統治する王政がベストの政体。民主制は必然的に腐敗する。
\end{itemize}


\section{アリストテレス}
\begin{itemize}
\item プラトンの弟子。アカデメイアで学んだのちに、学園リュケイオンを設立。散歩しながら議論したので逍遥学派と呼ばれる。
\item 現代で言えば文学、天文学、物理学、生物学など広範囲の学問の書物を残し、「万学の祖」と呼ばれる。→ アラビア語に翻訳され、中世にふたたびヨーロッパにもたらされる。近世のはじめまで自然科学についての権威とされていた。
\item 「人間は社会的動物である。」
\item 幸福=よく生きる≠単に楽しく暮らす。
\item よく生きる=人間の能力(徳)の発揮 = 自己実現。
\item 徳(アレテー)=卓越性。
\item 倫理的徳と知的徳。倫理的徳については「中庸」がベスト。勇気という徳は、臆病と蛮勇の中間、気前のよさという徳はケチと浪費の中間。
\item 正義(ディカイオシュネー)。正義にはいくつか意味がある。応報的正義。善には善を、悪には悪を与える。分配的正義 = 名誉や財産の正しく平等な分配。匡正的正義 = 平等な関係が不当に侵されたときに、悪に罰を加え、善に報奨を与える正義。
\item 友愛(ピリア)。ポリスは他者を承認し尊重しあう共同体(コイノニア)。

\item 国家体制
  \begin{itemize}
  \item × 単なる多数決による民主制。貧しいものの利益だけが追求される。
  \item × 僭主独裁制。少数のものの利益だけが追求される。
  \item 王政。王が法にしたがって市民全体の利益を追求する。
  \item 優秀者支配。エリートたちが市民全体の利益を追求。
  \item 国政(ポリーテイアー)。よき指導者のもとで市民全体の利益を追求する民主制(共和制)。
  \end{itemize}
\end{itemize}


\section{ヘレニズム・ローマ時代}

\begin{itemize}
\item ポリスの崩壊。
\item アレクサンドロス大王(アリストテレスが家庭教師をした)がマケドニアを拡大。大帝国に。帝王は王の王。帝国は大小さまざまな国をたばねた国の集合体。
\item  ギリシア文化が各地に伝播。しかしマケドニア帝国はアレクサンダーの死とともに崩壊。
\item ローマ。貴族共和制 → 最初の成文法である十二表法の制定(451B.C.)。 → イタリア半島統一 → 民主共和制へ移行(紀元前3世紀) → 地中海世界統一 → 内乱→ 三頭政治 → カエサル(シーザー)独裁 → 帝政(ローマ帝国)(27 B.C.) → 大帝国に → ネロなどの暴君が出る。
\end{itemize}
\if0

\subsection{エピクロス派}

\begin{itemize}
\item エピクロス(341B.C. -- 270 B.C.)
\item 「快楽主義」と呼ばれるが、各種の苦を避け、実際には穏かな快を求める。「心の平安」(アタラクシア)が目標。

\end{itemize}

\subsection{ストア派}
\begin{itemize}
\item ゼノン(335B.C. -- 263B.C.)が開祖。ノモスにしばられた一ポリス市民ではなく、世界市民(コスモポリタン)として自然の法則と理性にしたがって生きる。

\item エピクテートス (55--135)、ローマ皇帝マルクス・アウレリウス(121--180)など。

\item   禁欲主義。不動心(アパテイア)。自分の義務を果たせ。人間が体験する苦難は、それに心をわずらわせねば消滅すると考える。不幸や不安や苦痛はすべて心のなかにある。それに心を向けなければ、あたかも存在しないも同然である。人間は世界を支配することはできないが、自分の心を支配することはできる。

\item   ローマの思想家に影響、支配者階層に支持される。
\end{itemize}





\section{資料}



\subsection{エピクロス }

    \begin{itemize}
    \item 「 自然のもらす富は限られており、また容易に獲得することができる。しかし、むなしい憶見の追い求める富は、限りなく拡がる。」 %79

    \item 「正しい人は、最も平静な心境にある。これに反し、不正な人は極度の動揺に満ちている。」

    \item 「欲望のうち、或るものは自然的でかつ必須であり、或るものは自然的だが必須ではなく、他のものは自然的でも必須でもなく、むなしい憶見によって生まれたものである。」


    \item 「飢えないこと、渇かないこと、寒くないこと、これが肉体の要求である。これらを所有したいと望んで所有するに至れば、その人は、幸福にかけては、ゼウスとさえ競いうるであろう。」

    \item 「生の限界を理解している人は、欠乏による苦しみを除き去って全生涯を完全なものにするものが、いかに容易に獲得されうるかを知っている。それゆえに、かれは、その獲得のために競争を招くようなものごとをすこしも必要としない。」

    \item 「水とパンで暮しておれば、わたしは身体上の快に満ち満ちていられる。そしてわたしは、ぜいたくによる快を、快それ自体のゆえにではないが、それに随伴していやなことが起るがゆえに、唾棄する。

    \item 「隠れて生きよ」
    \end{itemize}




\subsection*{エピクテートス}



   \begin{itemize}
   \item もろもろの存在のうち、あるものは私たちの権内にあるけれども、あるものは私たちの権内にはない。意見や意欲や欲求や忌避、一言でいって、およそ私たちの活動であるものは、私たちの権内にあるけれども、肉体や財産や評判や公職、一言でいって、およそ私たちの活動でないものは、私たちの権内にない。

   \item そして私たちの権内にあるものは、本性上自由であり、妨げられず、じゃまされないものであるが、私たちの権内にないものは、もろい、隷属的な、妨げられる、他に属するものだ。


   \item そこでつぎのことを記憶しておくがいい。もし本性上隷属的なものを自由なものと思い、他人のものを自分のものと思うならば、きみはじゃまされ、悲しみ、不安にされ、また、神々や人びとを非難するだろう。だが、もしきみのものだけをきみのものであると思い、他人のものを、事実そうであるように、他人のものと思うならば、だれもきみにけっして強制はしないだろう。だれもきみを妨げないだろう。きみはだれをも非難せず、だれをもとがめることはないだろう。きみはなにひとついやいやながらすることはなく、だれもきみに害を加えず、きみは敵を持たないだろう、なぜなら、きみはなにも害を受けないだろうから。

   \item 出来事が、きみの好きなように起こることを求めぬがいい、むしろ出来事が起こるように起こることを望みたまえ。そうすれば、きみは落ち着いていられるだろう。
   \end{itemize}



\subsection*{マルウス・アウレリウス}



 [第5章の1] 明け方に起きにくいときには、つぎの思いを念頭に用意しておくがよい。「人間のつとめを果たすために私は起きるのだ。」自分がそのために生まれ、そのためにこの世にきた役目をしに行くのを、まだぶつぶついっているのか。それとも自分という人間は夜具の中にもぐりこんで身を温めているために創られたのか。「だってこのほうが心地よいもの。」では君は心地よい思いをするために生まれたのか、いったい全体君は物事を受身に経験するために生まれたのか。それとも行動するために生まれたのか。小さな草木や小鳥や蟻や蜘蛛や蜜蜂までがおのがつとめにいそしみ、それぞれ自己の分を果して宇宙の秩序を形作っているのを見ないか。

  しかるに君は自分のつとめをするのがいやなのか。自然にかなった君の仕事を果すために馳せ参じないのか。「しかし休息もしなくてはならない。」それは私もそう思う、しかし自然はこのことにも限度をおいた。同様に食べたり飲んだりすることにも限度をおいた。ところが君はその限度を越え、適度を過すのだ。しかし行動においてはそうではなく、できるだけのことをしていない。

  結局君は自分自身を愛していないのだ。もしそうでなかったらば君はきっと自己の(内なる)自然とその意志を愛したであろう。ほかの人は自分の技量を愛してこれに要する労力のために身をすりきらし、入浴も食事も忘れている。ところが君ときては、彫金師が彫金を、舞踏家が舞踏を、守銭奴が金を、見栄坊がつまらぬ名声を貴ぶほどのにも自己の自然を大切にしないのだ。上にいった人たちは熱中すると寝食を忘れて自分の仕事を捗らせようとする。しかるに君には社会公共に役立つ活動はこれよりも価値のないものに見え、これよりも熱心にやるに値しないもののように考えられるのか。


\section{さらに学習するために}

\begin{itemize}
\item ここにあげた以外にも様々な哲学者が活躍した。ディオゲネス・ラエルティオス『ギリシア哲学者列伝』(岩波文庫)がおもしろい。
\item ローマの歴史はおもしろいし、文学作品や映画などで題材としてとりあげられ、また故事として使われるので勉強しよう。塩野七生『ローマ人の物語』(新潮社)などでもよい。
\end{itemize}





 \nocite{エピクロス59:教説と手紙}
 \nocite{エピクテートス58:人生談議:岩波}
 \nocite{マルクスアウレリウス56:自省録}
 \nocite{水田洋06:社会思想小史}

\fi
\ifx\mybook\undefined
\bibliographystyle{eguchi}
\bibliography{bib}


\end{document} %----------------------------------------------------------------

\fi





%%% Local Variables:
%%% mode: japanese-latex
%%% TeX-master: t
%%% coding: utf-8
%%% End:
