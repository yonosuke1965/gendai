\ifx\mybook\undefined
\documentclass[autodetect-engine,dvipdfmx-if-dvi,ja=standard]{bxjsarticle} \usepackage{mystyle}
\usepackage{wrapfig}
\author{江口聡}
%\date{}
\title{「ビッグ」ヒストリー}
\if0 %----------------------------------------------------------------

\fi  %----------------------------------------------------------------
\begin{document}
\maketitle
\else\chapter{}
\chapter{原始社会}
\fi

\section{タイムスケール}


我々が知っている歴史は、宇宙の歴史、地球の歴史のほんの一部。


\begin{table}
\caption{とても大きな時間}  
  \centering
  \begin{tabular}{l l}
    宇宙の創成 & 136億年前 \\
    地球の形成 & 46億年前 \\
    多細胞生物の発生 & 36億年前 \\
    哺乳類 & 2億2500万年前 \\
    霊長類 & 1億年〜7000万年前 \\
    恐竜の絶滅 & 6500万年前 \\
  \end{tabular}
\end{table}





\section{現生人類}

\begin{itemize}
\item 原生人類ホモ・サピエンスは25万年ぐらい前にアフリカ東岸に発生か。(アフリカ単一起源説が優勢)
\item 7〜5万年前にアフリカ大陸から外に進出(出アフリカ)。
\item 4〜5万年前にはオーストラリア、3万年前にはシベリア、1万3000年前には北アメリカ大陸、1万年前には南アメリカ最南端に到達。
\item   各地で大型哺乳類を絶滅させた。火入れ農耕で各地の生態系を変えた。
\item ミトコンドリアのDNAの研究から、各地に住む人々がどのていど昔に祖先を共有していたか推測が進んでいる。
\item 血縁を基本にした少人数の集団(最大200人程度))で生活していたと推測される。
\end{itemize}

\if0

\begin{table}
\caption{人類}
  \centering
  \begin{tabular}{l l l}

    500万年前 & 猿人 & アウストラロピテクス \\
    180万年前 & 原人 & ジャワ原人 \\
    50万円前 & 火の使用 & \\
    40万年前 & 旧人 & ネアンデルタール人 \\
    20〜30万年前 & 新人 & クロマニョン人 \\
    10〜5万年前 & & 出アフリカ \\
    9000年前 & & 農耕の開始 \\

                     
                       
    
  \end{tabular}
\end{table}


\fi

\section{狩猟採集生活}
\begin{itemize}
\item 人口増加は緩慢。高い幼児死亡率。
\item   親族・血族を中心にした集団生活。
\item 部族・氏族社会。血縁関係は重要。
\end{itemize}



\section{農耕生活の開始}

\begin{itemize}
  
\item サハラ砂漠(当時は砂漠ではなかった)での家畜の飼育とモロコシの栽培(紀元前9000〜8000年)。
\item 西アフリカでヤムイモの栽培(紀元前8000年前)。
\item 中部アメリカでカボチャの栽培(紀元前7000〜5000年)。
\item \ruby{肥沃}{ひよく}な三日月地帯で紀元前7000年前ごろから穀物の栽培。
\item → 人口の増加。
\end{itemize}




\section{先史時代}





\section{村落共同体から国家へ}

\begin{enumerate}

\item 狩猟採集生活。原始共産社会。親族・血族を中心にした集団生活。部族・氏族社会。
\item 放牧・農耕の開始 → 土地の私有の必要。
\item 余剰食料の貯蔵 →  農民以外の専門職の成立(祭祀、陶器工、兵士など) → 分業のはじまり → 貧富の差の発生 → 原始国家。
\item 王や貴族階級の成立。
\item 他氏族を征服し奴隷化する。奴隷制。 支配従属関係が発生する。
\end{enumerate}


\begin{table}[h]
\caption{社会の分類}
  \centering
  \begin{tabular}{ l l }
\hline{}
    バンド & 血縁集団社会。〜数十人まで、階層をもたない家族・親族関係 \\
    トライブ & 部族社会 。〜数百人、複数の血縁集団が集まった氏族社会 \\
    チーフダム & 首長社会。〜数千人、首長が意思決定。社会階層。社会分業。\\ 
    ステイト & 国家。専制的な権力者の支配が見られる。主権的な政治機構、官僚制度、法体系に基づく裁判。 \\
               \hline{}
                       
     
  \end{tabular}
\end{table}


\section{古代文明}

\begin{itemize}
\item 紀元前3000年ごろからメソポタミアで最初の都市文明。
\item 治水・灌漑など。\emph{神権政治}。王、神官、役人、戦士など階級社会。
\item 紀元前18世紀ごろハンムラビ王。
\item 文字が発明されると、法律や慣習が成文化される。紀元前1700年ごろに『ハンムラビ法典』が成立。同害刑法「目には目を」などが有名。
\item 紀元前3000年ごろエジプトではファラオによる統一国家。やはり専制的神権政治。
\end{itemize}




\section{読書案内}

近年、人類史は非常に人気がある。どんどん読もう。一般に、ポピュラーサイエンス(一般読者向け科学)は翻訳ものの方がレベルが高くおもしろい。

NHKなどのテレビ番組もチェック。

\begin{itemize}
\item 眞淳平『人類の歴史を変えた8つのできごと』
\item ハラリ『サビエンス全史』、
\item スティーブン・ミズン『氷河期以後』
  
\item アダム・ラザフォード『ゲノムが語る人類全史』、
\item マット・リドレー『繁栄』、
\item オッペンハイマー『人類の足跡10万年全史』、
\item ジャレド・ダイアモンド『昨日までの世界』『文明崩壊』『銃・病原菌・鉄』など。

\item NHKスペシャル『人類誕生』DVDなど。

\end{itemize}



\ifx\mybook\undefined
\addcontentsline{toc}{chapter}{\bibname}
%\bibliographystyle{jecon}
%\bibliography{bib}


\end{document} %----------------------------------------------------------------
\fi






%%% Local Variables:
%%% mode: japanese-latex
%%% TeX-master: t
%%% coding: utf-8
%%% End:
