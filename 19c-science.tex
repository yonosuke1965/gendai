\ifx\mybook\undefined
\documentclass[uplatex,dvipdfmx]{jsarticle}
\usepackage{okumacro,plext}
\usepackage{natbib}
\usepackage{url}
\usepackage{txfonts}
\usepackage[utf8]{inputenc}
\usepackage[T1]{fontenc}
\usepackage{otf}
%\usepackage{my_resume}
%\usepackage{graphicx,wrapfig}
%\usepackage[greek,english]{babel}
%\usepackage{teubner}
%\usepackage[dvipdfm,bookmarkstype=toc=true,pdfauthor={江口聡, EGUCHI Satoshi}, pdftitle={}, pdfsubject={},pdfkeywords={},bookmarks=false, bookmarksopen=false,colorlinks=true,urlcolor=blue,linkcolor=black,citecolor=black,linktocpage=true]{hyperref}
  \AtBeginDvi{\special{pdf:tounicode EUC-UCS2}}% platex-utf8 でも OK
\author{江口聡}
%\date{}
\title{19世紀の科学の発展}
\if0 %----------------------------------------------------------------

\fi  %----------------------------------------------------------------
\begin{document}
\maketitle
\else\chapter{19世紀の科学の発展}\fi



\label{cha:19}

19世紀は科学が開花した時代でもある。



\section{進化}



\subsection{ダーウィン以前}




\begin{enumerate}
 \item キリスト教的理解=「神が作った種」。デザイン論。時計を見れば時計職
       人がそれを作ったことがわかるように、生物のようによくデザインされたものを見ればそれを創造した者がいることがわかる。人間以外の生物は人間のために造られた。

\item 18世紀までの世界観。
\item デザイン論。人間も他の自然物も神がデザインした→動植物は人間が使
  う「ため」にある。
\item 存在の連鎖。大気→水→土→樹木→昆虫→爬虫類→魚類→両足獣→サル
  →人間
\item 生物の多様性。なぜこんなに多様な生物が存在するのか?
\item 博物学。
\item ビュフォン。
\item リンネ(1707--1778)の分類学。生物の世界に秩序を見つけようとする。


\end{enumerate}



\subsection{ダーウィンの進化論}

\begin{itemize}
\item ラマルク(1744--1829)の進化論。「生物はすべて前進する」。「獲得形質の遺伝」をもとにした用不用説。 ← 否定されてる。
\item ダーウィン(1809--1882)。ガラパゴス諸島の生物の多様性の観察。フィンチという小鳥の研究。
  フィンチの種類は多様で、その嘴は、その食べ物によって違う。ほんの少しの環境の違いに生物が適応している。園芸家の品種改良の観察。
\item 進化論の柱
  \begin{enumerate}
  \item 遺伝。子供は親の性質に似る。
  \item 変異。生物の個体には、同じ種に属していてもさまざまな変異が見られる。
  \item 「生存のための奮闘」(struggle for existence 生存競争、生存闘争)
  \item 自然淘汰(自然選択)。生物は多産で、生き残れるより多く生む。→ 競争。
  \item 変異のなかには、生存や繁殖に影響を与えるものがある。→有利な変異の保存。
  \end{enumerate}
\item 人間と動物の共通性。動物にも心や知性がある。
\end{itemize}




\subsection{よくある誤解}

\begin{enumerate}

 \item ×動物は種の保存のために生殖する
 \item ×低級な動物から高等な動物に進化してきた(→すべての生物は進化の結果)
 \item ×自然淘汰はなんらかの目的にむかっている(→淘汰そのものは無目的)
 \item ×進化は進歩である(→単なる変化にすぎない)
 \item ×人間はチンパンジーから進化してきた(→共通の祖先から進化してきた)

\end{enumerate}


\subsection{『人間の由来』}




\section{社会ダーウィン主義}

\begin{enumerate}
\item スペンサー(1820--1903)。万物は単純なものから複雑なものに進化する。
\item 最適者生存(survival of the fittest)。→ 弱肉強食、優勝劣敗 → 自由放任(レッセ・フェール)が最適な社会をつくる。

\item 社会進化論→ ナショナリズム。国家間、人種間、民族間の闘争を「進化論」から正当化。「適者生存」→欧米各国、特にナチス。→欧米の移民政策に影響。→ナチスドイツの民族浄化政策。

\item ← ハクスリー(1825--1895)は進化の法則に逆らうような倫理が必要と主張。
\begin{quote}
  治癒の見込みのない病人、老悴した者、虚弱な心身あるいは欠陥のある心身を有する者、過剰な新生児などは、ちょうど園芸家が欠陥のある植物や過剰な植物を間引きし飼育家が望ましくな家畜を屠殺するように、取り除かれることになろう。強くて健康な者、それもこの統治者の目的に最も適応するような子孫をつくるために注意深く縁組みされた者だけが一族を残すことを許されることになろう。(Th. ハクスリー)
\end{quote}

\end{enumerate}

\nocite{市野川容孝12:社会学}





\section{医学}



ゼンメルワイス。衛生。

パスツール。細菌の発見。

ナイチンゲール。統計学。


\section{精神医学}



フランスの精神医学者ピネル(1745--1826)。閉鎖病棟で鎖につながれている「狂人」を解放。病院の改善。

エミール・クレペリン(1856-1926)。精神医学の成立。精神病を早発性痴呆(統合失調症)と双極性障害(躁うつ病)に分類。

リヒャルト・クラフトエビング(1840-1902)。『性の病理学』。性的倒錯(変態性欲)の研究。

フロイト。ヒステリー症例の研究から「無意識」概念の提唱。「精神分析」という手法の開発。


\section{人体の測定、優生学}


ガルの骨相学。


\citet{darmon89:_medec_et_assas}

イタリアの医師ロンブローゾ(1836-1909)の「犯罪人類学」。『犯罪者論』犯罪者の身体を測定。生来性犯罪者の概念。
犯罪者は隔世遺伝(先祖返り)。
『天才論』。天才は生来性犯罪者や狂人と類似した特徴がある。
 → 人種差別的だと批判されるが、アイディアは現代の犯罪学にも影響。

 人体測定ブーム。
 \citet{darmon89:_medec_et_assas}




ダーウィンの従兄弟、フランシス・ゴルトン(1822--1911)。
統計学の基礎(回帰、相関など)。
指紋を統計的に処理し、犯罪捜査に使うことを提唱。
 人類遺伝学 → 優生学。人類の遺伝的改良。→ 各国に影響。




優生学的政策。
オーストラリア。1901年移民制限法。
米国。1907年断種法(インディアナ州)→30州。移民制限法 1924。白人と有色人種の混血を防ぐ。
ドイツ。1933年断種法。アルコール依存症患者、性犯罪者、精神障害者、遺伝病。
カナダ、ノルウェー、フィンランド、デンマーク、スイスetc.
マーガレット・サンガーらの産児制限運動とも関連。
日本でも第二次世界大戦前から各種の動き。戦後の優生保護法1948。
\citet{米本昌平00:優生学}など。


\section{科学・技術}



ワットの蒸気機関の発明 1769。スティーブンソンの蒸気機関車 1814。

1830年代に電信技術が実用化。1840年代にはモールス信号が一般化。

ファラデー(1791-1867)。電気分解、電磁誘導など。ベンゼン。イオン。

ボルタのボルタ電池。→ 電気めっき。

ノーベルのダイナマイト。→土木、戦争。

マクスウェル(1831-1879)。古典電磁気学の確立。電磁波(電波)の存在の予言。熱力学、統計学。
電磁波はヘルツによって発見される。 → マルコーニの無線電信 1895。

カール・ボッシュのアンモニア合成。→肥料革命。火薬の製造に必要な硝酸。



\section{心理学}



哲学 + 生理学 →
ヴィルヘルム・ヴントの実験心理学。1879年にライプチヒ大学に心理学実験室創設。
ウィリアム・ジェームズ。
20世紀初頭に学問分野として独立。
アルフレッド・ビネーの知能テスト。

\section{人類学}

生物学、地質学、地理学。
1857年、ネアンデルタール人の骨を発見。





\ifx\mybook\undefined

\bibliographystyle{eguchi}  
\bibliography{bib,library}


\end{document} %----------------------------------------------------------------
\fi






%%% Local Variables:
%%% mode: japanese-latex
%%% TeX-master: t
%%% coding: utf-8
%%% End:
