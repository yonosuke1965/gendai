\documentclass[uplatex,dvipdfmx]{jsarticle}
%\usepackage{anaume}
\usepackage{natbib}
\usepackage{okumacro}
\usepackage{plext}
\usepackage{url}
\usepackage{txfonts}
\usepackage[utf8]{inputenc}
\usepackage[T1]{fontenc}
%\usepackage{graphicx,wrapfig}
%\usepackage[greek,english]{babel}
%\usepackage{teubner}
\usepackage[dvipdfm,
%bookmarkstype=toc=true,
pdfauthor={江口聡, EGUCHI Satoshi},
pdftitle={},
pdfsubject={},
pdfkeywords={},
bookmarks=false,                 % ←
bookmarksopen=false,
colorlinks=true,
urlcolor=blue,
linkcolor=black,
citecolor=black,
linktocpage=true
]{hyperref}
  \AtBeginDvi{\special{pdf:tounicode EUC-UCS2}}% platex-utf8 でも OK
\author{江口聡}
\date{2017年6月16日}
 \title{現代社会入門I 小テスト}
% \title{現代社会入門I 小テスト再試験}
\begin{document}
\maketitle

\newcounter{qnumber}\setcounter{qnumber}{1}
\def\anaume{\hspace{.5zw}\framebox[1.5cm]{\bf \theqnumber}\hspace{.5zw}\stepcounter{qnumber}}

\newcounter{anumber}\setcounter{anumber}{1}
\newcommand\sentakusi[1]{{{{\bf \alph{anumber}}~#1}\hspace{1zw}} \stepcounter{anumber}}


\section*{問1}

以下の空欄に入るもっとも適切な語句を下の選択肢から選び、記号を解答用紙に書け。


\begin{enumerate}



 \item プラトンは理想的な国家は民主制ではなく{\anaume{}}が支配するべきであると考えた。また紀元前4世紀ごろアテネ郊外に学園アカデメイアを運営した。

\item 古代ローマ帝国は広大な領域での平和を確保したため、その次代はのちの歴史家たちから、\anaume{}と呼ばれた。
  
\item 元パリサイ派のユダヤ教徒で、キリスト教を迫害したが、のち回心してキリスト教徒となり、イエスの死刑は人類の贖罪のためであるという教義の確立に貢献した人物が\anaume{}である。

\item 15〜6世紀フィレンツェの\anaume{}は、その著書『君主論』で、君主は人々の幸福ではなく自分の領土である国家の保持と拡大のために自分の力量や手腕を使用しなければならなず、その際にいかなる権謀術策を用いてもかまわないと説いた。

\item 現在「夢の国」の意味で使われる「ユートピア」は\anaume{}の作品からとられた。
  
\item イギリスのカルヴァン主義者たちは特に\anaume{}と呼ばれ、質素で禁欲的な生活を重んじた。

% \item ポーランド出身の\anaume{}は『天球の回転について』で地動説を唱え、従来のキリスト教的宇宙観をくつがえし、自然科学の発達に貢献した。

% \item イタリアの\anaume{}は『天文対話』などを出版し天体の運動を論じたが、そうした出版についてローマ教皇庁から有罪判決を受けた。

% \item イギリスの\anaume{}は『プリンキピア・マセマティカ』(『自然哲学の数学的諸原理』)を著して、古典的な力学の体系を確立した。また微積分法という新しい数学的手法を開発した。

%  \item デカルトは絶対に確実な真理をもとめるために、\anaume{}を用い、絶対
%    確実な知識として「われ思う、ゆえにわれあり」という原理を発見した。こ
%    のような「明晰で判明」な公理から出発し推論によって複雑な真理を見いだす\anaume{}を
%    哲学の方法とした。

% \item 「知は力なり」と唱えた{\anaume{}}は、人間の思い込みや先入観を{\anaume{}}と呼んだ。自然の正しい解明のためには、観察や実験によって得られた知見を秩序づけ一般法則を導く帰納法を持ちいることが必要であると考えた。

\item ホッブズの著書\anaume{}では社会が成立する前の自然状態においては、各人は自分がしたいことをする{\anaume{}}をもっているとされる。しかし、相互の利害の対立によって「各人の各人に対する戦争」に陥ってしまう。これは各人の利益に反するため、相互の利益のために相互に契約を結んで、自分の権利を為政者に{\anaume{}}する。

  
 \item ロックの著書\anaume{}によれば、社会が成立する前の自然状態にいても、他人の生命・自由・財産を犯してはならないという{\anaume{}}が成立している。しかし多くの人は公正と正義を守ろうとしないために、生命・自由・財産などの維持は困難である。したがってその保全のために契約を結び、権利を一部を為政者にゆだねる。しかし為政者が国民からゆだねられた信託に反して行動した場合、国民はこれに{\anaume{}}する権利をもつ。

 \item フランスの\anaume{}は、『法の精神』において、ロックの政治論や当時のイギリスの政
   治状況からヒントを得て、立法・行政・司法の三権の分立を唱えた。

 \item フランスの\anaume{}とダランベールらは「「技術と学問のあらゆる領域にわたって参照されうるような、そしてただ自分自身のためにのみ自学する人々を啓蒙すると同時に他人の教育のために働く勇気を感じている人々を手引きするのにも役立つ」ような『百科全書』を刊行し、知識と合理的な考え方を一般市民に普及させようとした。

 \item フランスの{\anaume{}}は自然状態においてこそ人びとは自足し幸福だったと考えるが、{\anaume{}}の成立によって不平等が生じ、支配と隷属の関係が生じたと考えた。その社会契約説は、「一般意志」という概念を導入して社会秩序の正当性を論ずるところに最大の特徴をもっている。また著書『\anaume{}』では、子どもを思いやりがあり善意にみちた自然人に育てる教育法を提唱している。


 \item アメリカ独立宣言はジェファーソン、アダムズ、シャーマン、リビングストン、それに\anaume{}の独立宣言起草委員会によって書かれた。最後の人物は貧困から身を起こした人物で、彼の『自伝』は明治時代にも有名で人気があった。

 \item 『哲学書簡』で有名な\anaume{}は啓蒙主義の代表者であり、従来のキリスト教を迷信であるとして強く批判した。また言論の自由を強く擁護し、「君に言うことには反対だが、君がそれを言う権利は生命をかけて守る」と友人に語ったとされる。

 \item イギリスの思想家\anaume{}はアメリカ独立戦争は支持したが、『フランス革命の省察』ではフランス革命を知的で抽象的なドグマにもとづいたものにすぎず、過激で危険であると批判し、人類の叡智の積み重ねである伝統の重要性を問いた。

\end{enumerate}

\begin{flushleft}
\emph{選択肢}
\end{flushleft}
\setcounter{anumber}{1}
\sentakusi{『リヴァイアサン』}
\sentakusi{『市民政府論』}
% \sentakusi{イドラ}
\sentakusi{エミール}
\sentakusi{カント}
% \sentakusi{ガリレオ}
% \sentakusi{ケプラー}
% \sentakusi{コペルニクス}
\sentakusi{ディドロ}
% \sentakusi{ニュートン}
\sentakusi{バーク}
\sentakusi{パウロ}
\sentakusi{ピューリタン}
\sentakusi{フランクリン}
% \sentakusi{ベイコン(ベーコン)}
\sentakusi{マキアベリ}
\sentakusi{モンテスキュー}
\sentakusi{ルソー}
\sentakusi{ルネサンス}
\sentakusi{ヴォルテール}
\sentakusi{信託}
\sentakusi{哲人王}
\sentakusi{私有財産}
\sentakusi{自然権}
\sentakusi{自然法}
\sentakusi{譲渡}
\sentakusi{パクス・ロマーナ}
\sentakusi{ローマン・ホリデイ}
\sentakusi{トマス・モア}










\section*{問2}

ホッブズとロックの自然状態および社会契約説の違いを説明せよ。
(ヒント:問1のホッブズに関する記述を参考にせよ)


\end{document}



%%% Local Variables:
%%% mode: japanese-latex
%%% TeX-master: t
%%% coding: utf-8
%%% End:
