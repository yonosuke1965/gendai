\ifx\mybook\undefined
\documentclass[uplatex,dvipdfmx]{jsarticle}
\usepackage{okumacro,plext}
\usepackage{natbib}
\usepackage{url}
\usepackage{txfonts}
\usepackage[utf8]{inputenc}
\usepackage[T1]{fontenc}
\usepackage{otf}
%\usepackage{my_resume}
%\usepackage{graphicx,wrapfig}
%\usepackage[greek,english]{babel}
%\usepackage{teubner}
%\usepackage[dvipdfm,bookmarkstype=toc=true,pdfauthor={江口聡, EGUCHI Satoshi}, pdftitle={}, pdfsubject={},pdfkeywords={},bookmarks=false, bookmarksopen=false,colorlinks=true,urlcolor=blue,linkcolor=black,citecolor=black,linktocpage=true]{hyperref}
  \AtBeginDvi{\special{pdf:tounicode EUC-UCS2}}% platex-utf8 でも OK
\author{江口聡}
%\date{}
\title{19世紀の社会科学}
\if0 %----------------------------------------------------------------

\fi  %----------------------------------------------------------------
\begin{document}
\maketitle
\else\chapter{社会学の成立}\fi



\label{cha:19century}


\section{社会学の成立}

\section{コントの実証主義}

\begin{itemize}
\item コント (Auguste Comte, 1798--1857)。サン=シモンの弟子。
\item 人類の知的進歩。神学的段階(想像) → 形而上学的段階(理性・論理) → 実証的(科学的)段階 (観察、実証)。
\item 「社会学」。歴史学、心理学、経済学を統合。社会の秩序と人類の進歩に寄与する。
\end{itemize}



\section{社会学の成立}
\begin{itemize}
\item マルクス (Karl Marx, 1818-83)。『資本論』。宗教・思想・道徳などの上部構造は、経済という下部構造に依存する。
\item デュルケーム。(\'{E}mile Durkheim, 1858-1917)。『社会分業論』『宗教生活の原初形態』。『自殺論』。統計資料を駆使して、自殺を個人の心理ではなく社会的要因から説明。
\item ウェーバー(ヴェーバー)。(Max Weber, 1864-1920)。人間の内面から社会的行為を理解する理解社会学。『プロテスタンティズムの倫理と資本主義の精神』。資本主義の発展は、カルヴァン主義の禁欲と合理性。(⇔ マルクス)
\item ジンメル。(Georg Simmel, 1858-1918) 『貨幣の哲学』
\item 各種の統計の収集、分析 → 統計学の発展。エンゲルのエンゲル係数(家計における食費の割合)など。
\end{itemize}






\ifx\mybook\undefined
\bibliographystyle{eguchi}  
\bibliography{bib,library}


\end{document} %----------------------------------------------------------------
\fi






%%% Local Variables:
%%% mode: japanese-latex
%%% TeX-master: t
%%% coding: utf-8
%%% End:
