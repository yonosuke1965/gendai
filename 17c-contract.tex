\ifx\mybook\undefined
\documentclass[uplatex,dvipdfmx]{jsarticle} \usepackage{mystyle}%\author{} %date{}
\title{イギリス革命と社会契約説}

\if0 %----------------------------------------------------------------

\fi  %----------------------------------------------------------------
\begin{document}
\maketitle
\else\chapter{社会契約説}
\fi


\label{cha:contractarian}

\section{歴史的背景}
\begin{itemize}
\item 中世は封建制。数多くの領主がそれぞれ領土を支配。領主と農民、領主どうしが主従関係を結ぶ。地方分権。荘園での自給自足経済。 身分制度の固定。
\item ローマ教皇と皇帝の二重支配。
\item 宗教戦争。ユグノー戦争(1562--98)、オランダ独立戦争(1568--1609)、三十年戦争(1618--48)
 \item イギリス市民戦争(ピューリタン革命 1642--49)、 → 名誉革命(1688)
\item 軍事革命。小銃や大砲が利用されるようになり、騎士より歩兵が重要になる。
\item 封建的な家臣と傭兵中心の軍から、国民兵中心の軍を常設するようになる(常設軍)。封建制は次第に崩壊。
\item 軍事費の調達や徴兵のために、強大な権力と徴税を中心にした官僚組織が必要になる。
巨大な権力をもった国王(絶対君主)の出現。  近代的\emph{国民国家}(Nation State)。国家の政治的・経済的統一、官僚制の整備。
\item 境界のはっきりした国土、徴税と徴兵の対象となる国民、国を統治する主権の三要素をもつ近代的\emph{主権国家}の成立。

\end{itemize}


\section{絶対王政}

\begin{itemize}
\item イギリス(チューダー朝)、スペイン(ハプスブルク朝)、フランス(ヴァロワ朝→ブルボン朝)、オーストリア(ハプスブルク朝)。

\item 絶対王政。主権者は国王、支配者階級は貴族(領主)、大商人、金融業者。
 \item 絶対王政では王は法に拘束されない。ルイ14世(1634-1715)「朕は国家なり」
\item 経済は重商主義(貿易重視)。官僚組織と常備軍。
\end{itemize}



\if0
 \section{グロティウス}

 \begin{itemize}
 \item Hugo Grotius, 1583--1645。
 \item 宗教改革後。宗教的対立が、成立期の近代的国家の政治的対立と結びつき、ヨーロッパ全体で頻繁に戦争が行なわれる。
 \item 法律を共有しない集団間にも妥当する法はあるか?
 \item 自然法。「他人のものに対する節制」「もしなんらかの他人のものを有していたり、あるいはそこから利益を得たりした場合には返還すること」「約束を履行すべき義務」「過失によって加えられた損害の補償」「人々のあいだにおける罰という応報」等。
 \item 正戦論。自然法に反する人々を戦争を起こすことは不正ではない。「他人の権利が奪われないかぎり、みずからのために見通しをたて配慮することは、社会の自然本性に反しない。したがって、他人の権利を侵害しないような実力行使は不正ではない。」
 \end{itemize}
 \fi

\section{ホッブズ}

ホッブズ (Thomas Hobbes, 1588-1679)。ベーコンの弟子をしたり、チャールズ2世の家庭教師をしたことがある。『市民論 \emph{De Cive}』1642。『リヴァイアサン \emph{Leviathan}』1651。『人間論 \emph{De Hormine}』1658。

\subsection{歴史背景}




17世紀のヨーロッパでの市民戦争を目撃。1642-1648、クロムウェルの革命、チャールズ一世が処刑(1649)。

ジェームズ一世が王権神授説を唱えて専制政治を展開。議会を無視。国教会がカルヴァン派(ピューリタン)を抑圧。ピューリタンの一部は北アメリカに移住。

チャールズ一世が課税を強化。1628年議会が、権利の請願を提出。課税の際の議会の承認、法律によらない逮捕の禁止などを求める。

王党派と議会派に分かれ内戦となる。議会派を応援したピューリタンのクロムウェルが王党派軍を破り、チャールズ一世をとらえ、1749年処刑。王のいない共和制となる。1953年クロムウェルの独裁開始、1658年死去。1660年チャールズ二世が即位して王政復古。


\subsection{『リヴァイアサン』}



 \begin{enumerate}

 \item (1) 観察によって人間の本性を明らかにする。(2)人間は利己的で合理的。(3)利己的で合理的な人間が平和に暮すにはどういう社会が必要かを考える。

  \item \emph{自己保存}の欲求。

  \begin{quote}
    人間は生まれながらにして、しかも自然に、かれらが切望するものなら何でも奪い合い、できることなら世界を恐れさせ、かれらに従わせようとするであろう。
  \end{quote}

\item 法や道徳、国家などがまったく存在しない状態({\bf 自然状態})を想定してみる。

\item 人間の心身の諸能力の平等性。人間の間には全体としてみれば飛びぬけて他を圧倒してしまうほどの能力の差はない。

\item 各人の目標達成(自己保存と欲求充足)についての希望の平等性。→競
  争

\item 他人に出し抜かれるのではないかという不信。

\item 自己保存のための先制攻撃。
\item 安全の確保のため、他者を滅ぼし、支配しようとする。
\item 節度のない人の存在と度をすぎた征服行為→節度ある人々も同様の行為へ。
\item 「誇り」という情念のため、自分を軽視するものに対して力を誇示するための攻撃。

    \item ⇒ \emph{各人対各人の戦争}。


    \item 「各人対各人の戦争」においては、「継続的な恐怖と非業の死の危険が存在し、人間の生活は孤独で貧しく、陰険で残忍でしかも短い。」しかし「そこから脱却する可能性はあり、その可能性の一部は諸々の情念に、一部は理性にある。」「人々を平和へ向かわせる情念は、死への恐怖であり、快適な生活に必要なものを求める欲求であり、そして彼らの勤労によってそれらを獲得するという希望である。また、理性は、人々が同意へと導かれるような都合のよい平和の諸条項を示唆する。」これが\emph{自然法} (natural law)。

    \item \emph{自然権}=各人が自分自身の生命を維持するために、自分の欲するままに自分の力を用いるという各人の自由。自由は外的障害が存在しないということ。

    \item 自然法。理性によって見出された指令または一般的規則。


    \item 第1の自然法「各人は、平和を獲得する希望を彼が持つ限り、平和に向かって努力するべきである。そして彼が平和を獲得できないときは、戦争のあらゆる助けと利益を求め、用いてよい」

    \item 第2の自然法「人は、他の人々もまたそうである場合には、平和と自己防衛のためにそれが必要だと彼が考える限り、すべてのものに対する彼の権利を進んで捨てるべきである。また、他人が彼に対して許容するのと同じ程度の自由を自分も彼らに対して持つことで満足すべきである。」→つまり、「平和を求めるために必要な信約を結ぶべし」

    \item 第3の自然法「人々は自分の結んだ信約をはたすべきである。」→正義、所有の成立


    \item 「信約は、それをやぶることに対して悪い結果を保証する権利が存在するならば、それを履行するよう人々を拘束する。」信約を守らせるためには、人為的な強制力を設立する必要がある。→「主権」の設立の必要

    \item 主権者の権限。主権者には「平和と防衛に必要な事柄を判定し、そのために必要な行為をとるに十分な力」が必要。
    
    \end{enumerate}



\section{ロック}


John Locke, 1632--1704。『市民政府論(統治論)』(1689)。『寛容に関する書簡』(1689)など。



\subsection{『市民政府論』}



\begin{itemize}


\item イギリス名誉革命(1688)、「権利の章典」(1689)正当化。王の権利は神が与えるとする中世的な\emph{王権神授説}を否定。


  
\item 「政治権力とは、所有権の規制と維持のために、死刑、したがって当然それ以下のあらゆる刑罰を伴う法を作る権利であり、またこのような法の執行、および外敵からの国防のため共同体の力を用いる権利であり、しかもこれらすべてをただ公共善のためにのみ行使する権利である。」

\item 自然状態は平等で平和。

  \begin{quote} \small{}政治的権力を正しく理解し、それがよってきたところをたずねるためには、すべての人が自然の状態でどのような状態にあるかを考察しなければならない。すなわちそれは、人それぞれが他人の許可を求めたり、他人の意志に頼ったりすることなく、自然の法の範囲内で自分の行動を律し、自分が適当と思うままに自分の所有物と身体を処理するような完全に自由な状態である。

    それはまた平等な状態でもあり、そこでは権力と支配権はすべて互恵的であって、他人より多くもつ者は一人もいない。なぜなら、同じ種、同じ等級の被造物は、分けへだてなく生をうけ、自然の恵みをひとしく享受し、同じ能力を行使するのだから、すべての被造物の主である神がその意志を判然と表明して、だれかを他の者の上に置き、明快な命令によって疑いえない支配権と主権を与えるのでないかぎり、すべての者が相互に平等であって、従属や服従はありえないということは何よりも明白だからである。
  \end{quote}
  
\item 自然状態にも神の自然法がある。

\item ひとは自然本来、身体と財産に\emph{所有権}をもつ。(\emph{生命}、\emph{自由}、\emph{財産}は自然本来その人の所有物。)
\item 自分の体による労働で作りだしたものはそのひとの所有物。
\item 第二の自然法「誰も他人の生命、健康、自由、財産を傷つけてはならない」
\item しかしこのような自然法に違反する者がいる。「自然法を犯すことによって違反者は、神が人間の相互的安全のためにかれらの行為に加えた制限である理性と一般的公正以外の別の規則に従って生きることをみずから宣言」している。これらの者には各種の懲罰を加えてさしつかえない。
\item 第二「政治的権威への服従義務は各人の「同意」に由来する」
\item 第三「政治権力は「正義」の実現のために行使されなければならない。第四「政治権力は「公共善」のために行使されなければならない」
\item 第五。政府は公共善・公共の安全のために権力を委託されているだけなので、人民は\emph{抵抗権・革命権をもつ}。

\item → アメリカ独立革命に大きな影響。
\end{itemize}


\subsection{寛容}

\begin{itemize}
\item 
\end{itemize}

\subsection{教育論}







 \nocite{中公67:デカルト}
 \nocite{山岡龍一95:ロック}
 \nocite{田中浩98:ホッブズ}
 \nocite{浜林正夫96:ロック}
 \nocite{太田義器95:グロティウス}


\section{さらに学習するために}



\ifx\mybook\undefined
\bibliographystyle{jecon}
\bibliography{bib}
\end{document} %----------------------------------------------------------------

\fi





%%% Local Variables:
%%% mode: japanese-latex
%%% TeX-master: t
%%% coding: utf-8
%%% End:
