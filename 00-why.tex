\ifx\mybook\undefined
\documentclass[autodetect-engine,dvipdfmx-if-dvi,ja=standard]{bxjsarticle} \usepackage{mystyle}
\author{江口聡}
%\date{}
\title{なぜ社会思想を学ぶか}
\if0 %----------------------------------------------------------------

\fi  %----------------------------------------------------------------
\begin{document}
\maketitle

\else
\chapter*{はじめに:なぜ社会思想を学ぶか}
\fi
\addcontentsline{toc}{chapter}{はじめに:なぜ社会思想を学ぶか}

歴史や社会思想についての知識は、大学生・市民としての基礎教養である。大学4年間での学習・研究のため、このテキスト程度のことは理解しておいてほしい。

若い人々には気づきにくいことだが、人間の社会は大きく変化してきた。そしてこれからも社会は変わる!その変化に対応していくため、そして変化をよい方向にもっていくためにも、社会の変化の背景にある思想とその変化についてある程度の知識をもっておきたい。方針としては、断片でもいいのでなるべく思想家たちの生の言葉を味わって欲しいと考えている。




% 公務員試験、教職、入社試験 → このテキスト程度で「人文」教養の6、7割とれることを目指す。

% 授業ではだいたい古代ギリシアから20世紀初頭までの西洋思想を扱う。


\section{問題意識}


人類の歴史のある時点から、人間は、自分たち人間と社会はどんなものであるか、どうあるべきか、社会の成り立ちはどんなものであるか、どうあるべきか、といったことを考えはじめる。社会状況が思想を生み、誰かの思想が社会状況を変革する。





次のような問題意識をもって半期学んでほしい。

\begin{itemize}
\item 「自由」「平等」「正義」「人権」「人民主権」「民主主義」「生命尊重」「選挙権」「福祉国家」といったさまざまな概念とその由来。
\item 国家はなんのためにあるのか?
  
\item 法はなぜ必要か?どのような法がよい法か?

\item 社会のルールはどのようにして決まるのか? どのようにして決める\kenten{べき}か。

\item 社会的動物としての人間とはどのような存在だと考えらてきたか?

\item 現在は当然の権利と考えられているが以前はそうでなかったものにはどのような権利があるか?

\item 国際関係はどうあるべきだと理解されていたか?現在はどう考えられているか?

\item 新しい発想や考え方が出てきたときに、\kenten{それまでに存在しなかったもの}を考えてみる。

\end{itemize}






\section*{勉強ティプス}

\begin{itemize}

\item 年代はだいたいで覚える。細かいのは無理。〜世紀初頭、前半、なかば、後半、末、程度のおぼえかたでよい。

\item 高校世界史の教科書を手元に置いておく。年表を自分で作ってみるのもよい。
\item 高校で世界史を十分履修していない人は、まず中学校の「歴史」を確認しておこう。
\item Wikipediaもどんどん利用してよい。
\item 歴史の本はどんどん読むこと。高校で勉強した政治中心の歴史よりは、ものの歴史、文化の歴史などがおもし\item 人の名前は覚えにくい。肖像画や写真といっしょにすると覚えやすい。アルファベットでの綴り字も一応確認しておくと恥ずかしい読み間違いが減る(おぼえる必要はない)。人名、出来事等は、さっさとWikipediaなどで確認する癖をつけたい。
ろい。河出書房新社の「ふくろうの本」シリーズ、新潮社の「とんぼの本」シリーズのような図版を中心にしたヴィジュアル本が読みやすいし印象に残りやすい。
\item NHKの『映像の世紀』に代表される歴史と科学史のテレビ番組もどんどん見よう。
\end{itemize}




\section{おすすめ本}
\begin{itemize}
\item 水田洋、『社会思想小史』、ミネルヴァ書房、2006。
\item ミシェリン・イシェイ、『人権の歴史』、明石書店、2008。
\item リン・ハント、『人権を創造する』、岩波書店、2011。
\item スティーブン・ピンカー、『暴力の人類史』、2015。
  
\end{itemize}


\nocite{Pinker11:angel}
\nocite{hunt07:_inven_human_right}
\nocite{水田洋06:社会思想小史}
\nocite{ishay04:_histor_human_right}





\ifx\mybook\undefined
\addcontentsline{toc}{chapter}{\bibname}
\bibliographystyle{jecon}
\bibliography{bib}


\end{document} %----------------------------------------------------------------
\fi







%%% Local Variables:
%%% mode: japanese-latex
%%% TeX-master: t
%%% coding: utf-8
%%% End:
