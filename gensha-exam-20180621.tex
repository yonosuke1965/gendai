\documentclass[uplatex,dvipdfmx]{jsarticle}
\usepackage{okumacro,plext}
\usepackage{natbib}
\usepackage{url}
\usepackage{txfonts}
\usepackage[utf8]{inputenc}
\usepackage[T1]{fontenc}
\usepackage{otf}
%\usepackage{my_resume}
%\usepackage{graphicx,wrapfig}
%\usepackage[greek,english]{babel}
%\usepackage{teubner}
%\usepackage[dvipdfm,bookmarkstype=toc=true,pdfauthor={江口聡, EGUCHI Satoshi}, pdftitle={}, pdfsubject={},pdfkeywords={},bookmarks=false, bookmarksopen=false,colorlinks=true,urlcolor=blue,linkcolor=black,citecolor=black,linktocpage=true]{hyperref}
  \AtBeginDvi{\special{pdf:tounicode EUC-UCS2}}% platex-utf8 でも OK
\author{江口聡}
%\date{}
\title{現代社会入門I 小テスト}


% \def\anaumei#1{\xkskip%
% \framebox[1.5cm]{\bf (\rensuji\label{#1})}\xkskip}
% \def\anaakiref#1{{\bf (\ref{#1})}}

\usepackage{renban}



% \newcommand{\sentakusi}[4]{
% \hspace{.3zw}
% \emph{ア}\hspace{1zw} #1 \hspace{2zw} \emph{イ} \hspace{1zw}#2 \hspace{2zw}\emph{ウ}\hspace{1zw} #3 \hspace{2zw}\emph{エ}\hspace{1zw} #4

% }


\newcounter{qnumber}\setcounter{qnumber}{1}
\def\anaume{\hspace{.5zw}\framebox[1.5cm]{\bf \theqnumber}\hspace{.5zw}\stepcounter{qnumber}}

\newcounter{anumber}\setcounter{anumber}{1}
\newcommand\sentakusi[1]{{{{\bf \Alph{anumber}}~#1}\hspace{1zw}} \stepcounter{anumber}}

% \sentakusi{}{}{}{}



\if0 %----------------------------------------------------------------

\fi  %----------------------------------------------------------------
\begin{document}
\maketitle

\subsection*{問1 以下の文章を読み、空白にもっとも適切であるものの下の記号群から選べ }

\begin{enumerate}

% \setlength{\parskip}{.9zw}
% \setlength{\itemsep}{.9zw}


\item  前1000年ごろ、ヘブライ人は、\空欄数字{}王の時代に栄えたが、彼の死後、王国は分裂した。


\item 古代ギリシア人たちは、自分たちをヘレネス、異民族を\空欄数字{}と呼んだ。

  
% \item \空欄数字{}12神で、美の女神ヴィーナスの大理石の彫刻はもっとも有名である。


  % \item 紀元前8世紀ごろからギリシア人は地中海・黒海沿岸に都市国家(ポリス)を作りはじめる。城山(アクロポリス)のふもとにおかれた広場は\空欄数字{}と呼ばれた。

  
\item ペロポネソス戦争の時代、アテネではクレオンやアルキビアデスらの扇動政治家(\空欄数字{})が活動し、アテネはスパルタに敗北することになる。
  

\item 紀元前73年ごろ、共和政ローマでは\空欄数字{}にひきいられた剣闘士や奴隷による叛乱が大規模な戦争に発展した。

  
\item 共和政ローマでは、第1回三頭政治ののちに平民派の\空欄数字{}が台頭し終身独裁官となるが、暗殺される。

  
 \item アントニウスとエジプト女王\空欄数字{}の連合軍をやぶったオクタウィアヌスが、初代ローマ皇帝アウグストゥスとなる。
  
% \item のちに信者たちによって救世主とされるイエスが生まれたころのユダヤ王は\空欄数字{}である。

  
\item ソクラテスが民衆による投票によって刑死したのを見た\空欄数字{}は、著書『ソクラテスの弁明』『国家』などでソクラテスの姿を描くとともに、民主政を批判した。
  
\item 哲学者アリストテレスを教師にしていたマケドニアの\空欄数字{}は、ペルシアをほろぼしインダス川まで至る大帝国を築いた。

\item 紀元1世紀なかばのローマ皇帝\空欄数字{}は自分の母や妻、家庭教師をつとめた哲学者・政治家セネカらを殺害し、またローマ大火を理由にキリスト教徒を迫害し、初代教皇ペテロを処刑させたと言われる。

\item 紀元135年、ローマ帝国によってユダヤ人はイェルサレムへの立ち入りを禁止され、世界各地への離散することになる。これを\空欄数字{}という。
  
\begin{flushleft}
\emph{選択肢}
\end{flushleft}
\setcounter{anumber}{1}

\sentakusi{アウグストゥス}
\sentakusi{アレクサンドロス}
\sentakusi{カエサル}
\sentakusi{クレオパトラ}
\sentakusi{スパルタクス}
\sentakusi{ソロモン}
\sentakusi{ソロン}
\sentakusi{ダビデ}
\sentakusi{ディアスポラ}
\sentakusi{デマゴーグ}
\sentakusi{ネロ}
\sentakusi{ハドリアヌス}
\sentakusi{ハンニバル}
\sentakusi{バルバロイ}
\sentakusi{プラトン}
\sentakusi{ヘロデ}
\sentakusi{マルクス・アウレリウス}
\end{enumerate}


%% ----------------------------------------------------------------

\pagebreak{}

\subsection*{問2 以下の文章を読み、空白にもっとも適切であるものの下の記号群から選べ }

\begin{enumerate}


  
\item  1498年にポルトガルの\空欄数字{}がインドに到達した。

\item ポルトガルの\空欄数字{}は、1519年にスペインの艦隊を率いて香辛料の産地モルッカ諸島をめざして西まわりの大航海に出発し、1521年に現在のフィリピンで死亡した。しかし彼の艦船は1522年にスペインに帰港し、初めての世界一周をなしとげた。

  
\item  アメリカ大陸メキシコ高原にあったのアステカ帝国は1521年にスペインのコルテスによって、現在のペルー・ボリビア・エクアドル周辺にあったインカ帝国は1533年に同じくスペインの\空欄数字{}によって滅ぼされた。

\item 15〜6世紀フィレンツェの\空欄数字{}は、その著書『君主論』で、君主は人々の幸福ではなくスタト(国家)の保持と拡大のために自分の力量や手腕を使用しなければならなず、その際にいかなる権謀術策を持ちいてもかまわないと説いた。

\item カトリックの司祭でもあった\空欄数字{}によって、1511年に発表されて爆発的な売れ行きをみせた『痴愚神礼賛』は、モリアとよばれる痴愚の女神の自慢話というスタイルで当時の社会の病根をつぎつぎと暴いてゆくものだった。なかでも特に人々の戦争癖を痛烈に批判した。

\item \空欄数字{}がその著作のタイトルとした「ユートピア」とは、ギリシア語をもじった造語であり、「どこにもない所」を意味している。

% \item イギリスのカルヴァン主義者たちは特に\anaume{}と呼ばれ、質素で禁欲的な生活を重んじた。

  
\item  プロテスタント宗教改革に対応して、ローマ教会でも、\空欄数字{}が1534年に設立したイエズス会などを中心に宗教改革に対抗する改革がおこなわれた。

\item イギリスでは1642年に内乱が勃発し、\空欄数字{}が王党派を破り、チャールズ一世を処刑し独裁を行なった。
  
\item イギリスの\空欄数字{}は1687年に『自然哲学の数学的諸原理』(『プリンキピア』)を出版して、天体と地上の物体の運動をともに説明することができる古典力学の体系を確立し、また微積分学の基礎を築いた。
  
\end{enumerate}



\begin{flushleft}
\emph{選択肢}
\end{flushleft}
\setcounter{anumber}{1}

\sentakusi{イグナチオ・デ・ロヨラ}
\sentakusi{エラスムス}
\sentakusi{ガリレオ}
\sentakusi{クロムウェル}
\sentakusi{ケプラー}
\sentakusi{コペルニクス}
\sentakusi{コロンブス}
\sentakusi{チェーザレ・ボルジア}
\sentakusi{デスピサロ}
\sentakusi{トマス・モア}
\sentakusi{ニュートン}
\sentakusi{バスコ・ダ・ガマ}
\sentakusi{ピコ・デラ・ミランドラ}
\sentakusi{ピサロ}
\sentakusi{ピューリタン}
\sentakusi{フランシスコ・ザビエル}
\sentakusi{マキアベリ}
\sentakusi{マゼラン}
\sentakusi{ルイス・フロイス}
\sentakusi{ルター}
\pagebreak{}
%% ----------------------------------------------------------------

\subsection*{問3 以下の文章を読み、空白にもっとも適切であるものの下の記号群から選べ }

\begin{enumerate}

\item フランスの\空欄数字{}は、『法の精神』において、ロックの政治論や当時のイギリスの政治状況からヒントを得て、立法・行政・司法の三権の分立を唱えた。


\item 『哲学書簡』で有名な\空欄数字{}は啓蒙主義の代表者であり、従来のキリスト教を迷信であるとして強く批判した。また言論の自由を強く擁護し、「君に言うことには反対だが、君がそれを言う権利は生命をかけて守る」と友人に語ったとされる。  

\item \空欄数字{}の社会契約説は、「一般意志」という概念を導入して社会秩序の正当性を論ずるところに最大の特徴をもっている。一般意志とは、「つねに公的利益のみをめざす共同体の意志」をいう。

\item 18世紀後半、イギリスやフランスに遅れて、東ヨーロッパの君主たちは啓蒙思想を掲げて「上からの近代化」を目指した。プロイセン国王の\空欄数字{}、オーストリア大公のマリア・テレジア、ロシア帝国女帝のエカチェリーナ二世らがその代表であり、啓蒙専制君主と呼ばれる。

\item アメリカ独立宣言はジェファーソン、アダムズ、シャーマン、リビングストン、それに\空欄数字{}らの独立宣言起草委員会によって書かれた。最後の人物は貧困から身を起こした人物で、彼の『自伝』は明治時代にも有名で人気があり、また雷が電気であることを発見したことでも有名である。
  
\item 1989年のフランス革命後、ジャコバン派の\空欄数字{}が国民公会から独裁権を認められ、大規模な粛清を断行した。彼の「徳なき恐怖は忌まわしく、恐怖なき徳は無力である」言葉は有名で、こうした訴える政治手法は「テロリズム」と呼ばれる。
  
\end{enumerate}

\begin{flushleft}
\emph{選択肢}
\end{flushleft}
\setcounter{anumber}{1}


\sentakusi{カルロス三世}
\sentakusi{カント}
\sentakusi{グフタフ三世}
\sentakusi{ジェームズ二世}
\sentakusi{スミス}
\sentakusi{デカルト}
\sentakusi{フランクリン}
\sentakusi{フランソワ・ケネー}
\sentakusi{フリードリヒ二世}
\sentakusi{ベイコン}
\sentakusi{ホッブズ}
\sentakusi{マフムト二世}
\sentakusi{マルサス}
\sentakusi{マンデヴィル}
\sentakusi{モンテスキュー}
\sentakusi{リカード}
\sentakusi{ルソー}
\sentakusi{ロック}
\sentakusi{ロベスピエール}
\sentakusi{ヴォルテール}

\subsection*{問4 ホッブズは法や国家のない自然状態についてどのような状態であると考えたか、またなぜそのような状態になると考えたか、解答欄の許すかぎりで、できるだけわかりやすく詳細に説明せよ。}








\end{document} %----------------------------------------------------------------







%%% Local Variables:
%%% mode: japanese-latex
%%% TeX-master: t
%%% coding: utf-8
%%% End:
