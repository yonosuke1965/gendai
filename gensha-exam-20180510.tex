\documentclass[uplatex,dvipdfmx]{jsarticle}
\usepackage{okumacro,plext}
\usepackage{natbib}
\usepackage{url}
\usepackage{txfonts}
\usepackage[utf8]{inputenc}
\usepackage[T1]{fontenc}
\usepackage{otf}
%\usepackage{my_resume}
%\usepackage{graphicx,wrapfig}
%\usepackage[greek,english]{babel}
%\usepackage{teubner}
%\usepackage[dvipdfm,bookmarkstype=toc=true,pdfauthor={江口聡, EGUCHI Satoshi}, pdftitle={}, pdfsubject={},pdfkeywords={},bookmarks=false, bookmarksopen=false,colorlinks=true,urlcolor=blue,linkcolor=black,citecolor=black,linktocpage=true]{hyperref}
  \AtBeginDvi{\special{pdf:tounicode EUC-UCS2}}% platex-utf8 でも OK
\author{江口聡}
%\date{}
\title{現代社会入門I 小テスト}


% \def\anaumei#1{\xkskip%
% \framebox[1.5cm]{\bf (\rensuji\label{#1})}\xkskip}
% \def\anaakiref#1{{\bf (\ref{#1})}}

\usepackage{renban}



% \newcommand{\sentakusi}[4]{
% \hspace{.3zw}
% \emph{ア}\hspace{1zw} #1 \hspace{2zw} \emph{イ} \hspace{1zw}#2 \hspace{2zw}\emph{ウ}\hspace{1zw} #3 \hspace{2zw}\emph{エ}\hspace{1zw} #4

% }


\newcounter{qnumber}\setcounter{qnumber}{1}
\def\anaume{\hspace{.5zw}\framebox[1.5cm]{\bf \theqnumber}\hspace{.5zw}\stepcounter{qnumber}}

\newcounter{anumber}\setcounter{anumber}{1}
\newcommand\sentakusi[1]{{{{\bf \Alph{anumber}}~#1}\hspace{1zw}\\} \stepcounter{anumber}}

% \sentakusi{}{}{}{}



\if0 %----------------------------------------------------------------

\fi  %----------------------------------------------------------------
\begin{document}
\maketitle

\subsection*{問1 以下の文章を読み、空白にもっとも適切であるものの下の記号群から選べ }

\begin{enumerate}

\setlength{\parskip}{.9zw}
\setlength{\itemsep}{.9zw}


\item  前1000年ごろ、ヘブライ人は、\空欄数字{}王の時代に栄えたが、彼の死後、王国は分裂した。


\item 古代ギリシア人たちは、自分たちをヘレネス、異民族を\空欄数字{}と呼んだ。

  
\item \空欄数字{}12神で、美の女神ヴィーナスの大理石の彫刻はもっとも有名である。


  
\item 紀元前8世紀ごろからギリシア人は地中海・黒海沿岸に都市国家(ポリス)を作りはじめる。城山(アクロポリス)のふもとにおかれた広場は\空欄数字{}と呼ばれた。

  
\item ペロポネソス戦争の時代、アテネではクレオンやアルキビアデスらの扇動政治家(\空欄数字{})が活動し、アテネはスパルタに敗北することになる。
  

\item 紀元前73年ごろ、共和政ローマでは\空欄数字{}にひきいられた剣闘士や奴隷による叛乱が大規模な戦争に発展した。

  
\item 共和政ローマでは、第1回三頭政治ののちに平民派の\空欄数字{}が台頭し終身独裁官となるが、暗殺される。

  
\item アントニウスと\空欄数字{}の連合軍をやぶったオクタウィアヌスが、皇帝アウグストゥスとなる。
  
\item のちに信者たちによって救世主とされるイエスが生まれたころのユダヤ王は\空欄数字{}である。
  
\item 哲学者アリストテレスを教師にしていたマケドニアの\空欄数字{}は、ペルシアをほろぼしインダス川まで至る大帝国を築いた。
  

  

\pagebreak{}

\begin{flushleft}
\emph{選択肢}
\end{flushleft}
\setcounter{anumber}{1}

\sentakusi{アレクサンドロス}
\sentakusi{アゴラ}
\sentakusi{オリュンポス}
\sentakusi{ソロモン}
\sentakusi{バルバロイ}
\sentakusi{ヘロデ}
\sentakusi{サロメ}
\sentakusi{スパルタクス}
\sentakusi{カエサル}
\sentakusi{クレオパトラ}
\sentakusi{デマゴーグ}
\sentakusi{パクス・ロマーナ}
\sentakusi{ペイシストラトス}
\sentakusi{ドラコン}
\sentakusi{ソロン}
\sentakusi{ハンニバル}
\sentakusi{ネロ}


\end{enumerate}

% \subsection*{問2 (エキストラ問題、答えられれば最高+10点) }
% 「人権」という考え方は、世界史・思想史においてどのように成立したか、中学1年生に向けて2〜3分程度で説明するつもりで簡潔に述べよ。


\end{document} %----------------------------------------------------------------







%%% Local Variables:
%%% mode: japanese-latex
%%% TeX-master: t
%%% coding: utf-8
%%% End:
