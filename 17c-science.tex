\ifx\mybook\undefined
\documentclass[uplatex,dvipdfmx]{jsarticle}
\usepackage{okumacro,plext}
\usepackage{natbib}
\usepackage{url}
\usepackage{txfonts}
\usepackage[utf8]{inputenc}
\usepackage[T1]{fontenc}
\usepackage{otf}
%\usepackage{my_resume}
%\usepackage{graphicx,wrapfig}
%\usepackage[greek,english]{babel}
%\usepackage{teubner}
%\usepackage[dvipdfm,bookmarkstype=toc=true,pdfauthor={江口聡, EGUCHI Satoshi}, pdftitle={}, pdfsubject={},pdfkeywords={},bookmarks=false, bookmarksopen=false,colorlinks=true,urlcolor=blue,linkcolor=black,citecolor=black,linktocpage=true]{hyperref}
  \AtBeginDvi{\special{pdf:tounicode EUC-UCS2}}% platex-utf8 でも OK
\author{江口聡}
%\date{}
\title{宗教改革・科学革命}
\if0 %----------------------------------------------------------------

\fi  %----------------------------------------------------------------
\begin{document}
\maketitle
\else\chapter{科学革命}\fi


\section{宗教戦争}


\section{16世紀の科学革命}

\begin{itemize}
\item 宗教的権威の下落とともに、16世紀から自然科学が大きく発展する。
\item 古典中心の中世の学問からの脱脚。ex. 地動説→天動説。
\item 近代以前には古代の哲学者アリストテレスの学説が正しいと信じこまれていた。ex. 重いものは軽いものより速く落ちる。
\item コペルニクス(1473--1543)の地動説。
\item ガリレオ・ガリレイ(1564--1642)。望遠鏡の活用。振り子の等時性の発見。落体の法則。実験による検証。世界を数学的に理解する。「自然という書物は数学の言葉で書かれている」。
\item ケプラー(1571--1630)。惑星の楕円軌道を想定。引力の想定。
\item ニュートン(1643--1727)。力学の統合。万有引力。光学。微積分。
\item 学問の「方法」について反省。 → 近代的な学問の骨組ができる。
\item アリストテレスの目的論世界観と結びついたキリスト教神学の束縛からの解放。理性の重視。迷信の排除につながる。
\end{itemize}



\section{ベイコン}

\begin{itemize}
\item フランシス・ベーコン (1561--1626)。英国大法官などの重職につく。『新機関』『学問の進歩』『随想録』など。「\emph{知は力}」で有名。
 \item 「人類に授けうるすべての恩恵のなかで、人間の生活を改善するための新しい技術、能力、製品の発明ほどに偉大なものはない。」
 \end{itemize}
 
 \begin{quote}
   「とりわけ、なにか個別の発明をするのではなく、自然のなかに光を当てることに成功するならば、その人は宇宙に対する人間の帝国の布教者、自由の戦士、必然の支配者として人類の恩人となるでろう。というのはその光はまずはじめに、われわれの現在の知識の範囲を制限している境界領域を照らしだし、さらに広がりをましながら、この世界のなかで秘密に隠されていたものすべてを、やがて明るみにだし、視界に入るようにするからである。」
 \end{quote}
\begin{itemize}

\item それ以前の学問のように権威や原理からの演繹によるのではなく、\emph{実験・観察}の重視。方法的に収集された正確な情報にもとづく\emph{帰納}が重要。

\item 四つのイドラを避けることを提唱。種族のイドラ(人間の感覚の錯覚)、洞窟のイドラ(個人の習慣や性癖などによる誤り)、市場のイドラ(さまざまな言葉の使い方の混乱による誤り)、劇場のイドラ(思想や学説による誤り)。
\end{itemize}

\begin{quote}
  「いまや人間の知性をとらえてしまって、そこに深く根をおろしている\ruby{偶像}{イドラ}や偽りの概念は、すっかり人々の精神をとりかこみ、真理はその入口すら見つけることができなくなっている。そればかりでなく、たとえ真理への途が開かれたとしても、もしも人々が危険を前もって警告され、それらの攻撃からできるだけ自分を守るのでないかぎり、イドラはまたもや諸学の革新のさなかに現われ、われわれの邪魔をするであろう。」
\end{quote}
\begin{itemize}
\item 帰納。単に一、二の事例から結論を引き出すのではなく、用心深く

\end{itemize}



\section{デカルト}

\begin{itemize}
\item ルネ・デカルト(1596--1650)
\item デカルト。「学問」(科学)の作りなおし。
\item 良識・精神を正しく使えば誰もが確実な真理・正しい答を得ることができるはず。
\end{itemize}

\begin{quote}
    良識はこの世で最も公平に配分されているものである。というのは、だれもかれもそれを十分に与えられていると思っていて、他のすべてのことでは満足させることのはなはだむずかしい人々でさえも、良識については、自分のもっている以上を望まぬのがつねだからである。・・・したがって、われわれの意見がまちまちであるのは、われわれのうちのある者が他の者よりもより多く理性をもつから起こるのではなく、ただわれわれが自分の考えをいろいろちがった\ruby{途}{みち}によって導き、また考えているのが同一のことでない、ということから起こるのであると。というのは、よい精神をもつというだけでは十分ではないのであって、たいせつなことは精神をよく用いることだからである。
  \end{quote}

\begin{itemize}
\item \emph{方法的懐疑}。いろんなものが信じられないので、とにかく疑ってみる。
\item デカルトの「四つの規則」(明晰判明、分析、総合、枚挙)
\end{itemize}

\begin{quotation}
    第一は、私が明証的に真であると認めたうでなくてはいかなるものをも真理としては受け入れないこと。いいかえれば、注意深く即断と偏見とを避けること。そして、私がそれを疑ういかなる理由ももたないほど、明晰かつ判明に、私の精神に現われるもの以外の何ものをも、私の判断のうちにとり入れないこと。

    第二、私が吟味する問題のおのおのを、できる限り多くの、しかもその問題を最もよく解くために必要なだけの数の、小部分に分かつこと。

    第三、私の思想を順序に従って導くこと。最も単純で最も認識しやすいものからはじめて、少しずつ、いわば階段を踏んで、最も複雑なものの認識にまでのぼってゆき、かつ自然のままでは前後の順序をもたぬものの間にさえも順序を想定して進むこと。

    最後には、何ものをも見落とすことがなかったと確信するほどに、完全な枚挙と、全体にわたる通覧とを、あらゆる場合に行なうこと。(デカルト、『方法序説』(1637)、野田又夫訳、中公文庫、1974、p.27)
  \end{quotation}
  
\begin{itemize}
\item 「我思うゆえに我あり」
\end{itemize}



\nocite{中公67:デカルト}
\nocite{塚田富治96:ベイコン}






\section{モラリスト}

\begin{itemize}
\item moralist{\'e}。フランス文学の独特の系列。人間と道徳についての思索。モンテーニュ、パスカル、ラ・ロシュフコー(La Rochefoucauld) 、ラブリュイエール(La Bruy{\`e}re )

\item 「オネットム honne{\^e}te homme」。「簡単にいえば、オネットムとは良識を備え、思いやりがあり、人の気を傷つけず、人を楽しませる術を知り、何よりも専門家ぶらず、しかも何ごとにも関心を持ち、ひとかどの判断ができる教養人を指すのである。」\citep[p. 36]{前田陽一78:パスカルの人と思想}
\end{itemize}


\subsection{モンテーニュ}

\begin{itemize}
\item モンテーニュ (Michel Eyquem de Montaigne, 1533--1592)の『随想録』
\item フランス・ルネッサンス(16世紀初頭)。ギリシア・ローマの古典の復興。ユマニスム(人文学、古典研究)。

\item ボルドー市の裕福な一族。モンテーニュはラテン語を使って養育される。ボルドー最高法院の法官。40才前に引退。


\item エセー  Essais < essayer (v.) 試す。

  \begin{quote}\small{}
    判断力はあらゆる事柄に適用できる道具で、いたるところに関係する。この理由で、今ここでわたしが行なっている自分の判断力の試みにたいしても、わたしはあらゆる種類の機会を利用するのだ。\citep[p. 189]{モンテーニュ79:エセー}
  \end{quote}

\item あちこち旅行したのち、1581年よりボルドー市長、1685年退任。

\item \emph{懐疑主義}、判断停止(エポケー)、寛容。宗教戦争の悲惨を見て、教義の確信が災厄をもたらすと考え、常に懐疑をもちつづけることを決意。宗教的迫害を非難。言論の自由。


\end{itemize}



\subsection{パスカル}



\begin{itemize}
\item Blaise Pascal (1623--62)の『パンセ』 \emph{Pens{\'e}es}

\item 理系の天才。円錐曲線論、計算器の発明、真空の実験、パスカルの原理\footnote{密閉した流体の表面に加えられた圧力は、あらゆる方向にそのまま伝えられる。}、確率論と期待効用。


\item キリスト教弁神論(キリスト教教義の擁護)の執筆を企画。完成せず。
死後、近親や友人らが草稿をまとめて『パンセ』として出版。

\end{itemize}

\ifx\mybook\undefined
\bibliographystyle{eguchi}  
\bibliography{bib,library}
\end{document} %----------------------------------------------------------------
\fi





%%% Local Variables:
%%% mode: japanese-latex
%%% TeX-master: t
%%% coding: utf-8
%%% End:
