\ifx\mybook\undefined
\documentclass[dvipdfmx,uplatex]{jsarticle}
\usepackage{okumacro,plext}
\usepackage{natbib}
\usepackage{url}
\usepackage{txfonts}
\usepackage[utf8]{inputenc}
\usepackage[T1]{fontenc}
\usepackage{otf}
%\usepackage{my_resume}
%\usepackage{graphicx,wrapfig}
%\usepackage[greek,english]{babel}
%\usepackage{teubner}
%\usepackage[dvipdfm,bookmarkstype=toc=true,pdfauthor={江口聡, EGUCHI Satoshi}, pdftitle={}, pdfsubject={},pdfkeywords={},bookmarks=false, bookmarksopen=false,colorlinks=true,urlcolor=blue,linkcolor=black,citecolor=black,linktocpage=true]{hyperref}
  \AtBeginDvi{\special{pdf:tounicode EUC-UCS2}}% platex-utf8 でも OK
\author{江口聡}
%\date{}
\title{}
\if0 %----------------------------------------------------------------

\fi  %----------------------------------------------------------------
\begin{document}
\maketitle
\else\chapter{女性の解放}\fi

\section{第一波フェミニズム:女性参政権運動}


\subsection{欧米での女性運動}

欧米では、19世紀なかばに近代的な「人権」意識にもとづいて、女性の参政権や財産権、教育の機会などを求める女性運動が発生した。これらの運動は現在第一波フェミニズムと呼ばれる。

フランスでは、フランス革命勃発の後、フランス人権宣言(人および市民の権利宣言)(1789)(p. \pageref{french}参照)によって近代的「人権」が宣言される。宣言第一条では、すべての人が自由で権利において平等であること、第二条では政治的結合(国家や政府)の目的がひとの不可侵の権利の保全にあること、第三条ではすべての市民に法律の制定に参加する権利を認めた。しかしここで言われている政治形態に関する憲法審議のなかでは、選挙権をもつ者は能動市民(一定額の税金を納めている男子)に限られ、女性には選挙権も被選挙権も与えられていなかった。その他、ユダヤ人、有色の自由人、植民地の奴隷、奴婢等にも権利は認められなかった。1848年まで次第に権利を持つ者の範囲は拡大されたが、1848年にいたっても女性の権利は制限されていた。

米国では第一波の女性解放運動は奴隷制反対運動と強い結びつきがあった。1838年ニューヨーク州セネカフォールズでの女性権利大会大会の『女性の所信宣言』\footnote{p. \pageref{sentiment}参照}や、1856年オハイオ大会でのソジャーナ・トルースの演説(p.\pageref{truth}参)照が有名である。1865年、憲法修正第13条によって、奴隷制が廃止された。さらに憲法修正第14条で21歳以上の男性に投票権を認め、憲法修正第15条で人種、肌の色、以前の奴隷の身分等による差別を禁止した。しかし女性の投票権はいまだ認められなかった。1869年、全国婦人参政権協会(National Women's Suffrage Association, NWSA) や米国婦人参政権協会(American Women's Suffrage Association, AWSA)が結成された。男子普通選挙は1870年に施行されたが、男女ともに選挙権を持つ一般普通選挙は1920年になってから施行された。

一方英国では、女性運動は廃娼運動(売春禁止運動)と結びつきが強かった。1864年、性病の軍隊での性病の蔓延を防ぐための伝染病法(CD法)制定された。ジョセフィン・バトラーが売春婦のみを取り締り、買春者を取り締らないことは性道徳の\emph{二重基準(ダブルスタンダード)}であるとして伝染病法廃止を訴えた。同時に参政権運動も盛りあがりを見せ、1866年、J. S. ミルが婦人参政権を求める請願書を議会に提出したが、翌年の議会で否決された。英国で実際に男子普通選挙が行なわれるようになったのは1918年であり、一般普通選挙は1928年になってからであった。



% (マルクス主義。エンゲルス『家族・私有財産・国家の起源』。)

%  (フロイト)


\subsection{国内の女性運動}

それでは、日本国内では事情はどうだったろうか。

諸外国と同じように、国内でも女性運動は参政権運動の一部としてはじまった。明治民法のもとでは「家」制度が基本的だった。戸主(家長)が家族を統率し、戸主権と家産は長男子が継承する。また夫は夫婦の財産を管理し、子に対する親権をもつ。妻は無能力者とみなされ、妻には家督相続権がない。さらに妻には貞操義務があるが夫にはない、と男女は大きく不平等に扱われていた。

明治憲法(1889)で選挙権が与えられるのは、一定額を納税した男子だった。女性の政治活動は治安警察法第5条によって禁止された。1886年、キリスト教系の矯風会が社会風俗の改善を求め、未成年の禁煙、禁酒法の制定、売春の禁止(廃娼)などを求めた。

1911年、女性運動団体の青鞜社が結成される。1914年、機関誌『青鞜』で、貞操論争\footnote{生田花世が貧困から貞操を破ったことを告白。原田皐月は食べることより貞操を重んじるべきだと反論。平塚らいてうは男女の性道徳の歳を批判し、女だけに貞操を要求する道徳を否定。伊藤野枝は貞操観からの解放を提唱。} 堕胎論争\footnote{原田皐月が、「受胎したと云ふだけでは生命も人格も感じ得ません」として堕胎を肯定。伊藤野枝が母親の都合のために「いのち」を殺すことは自然を侮辱し「生命」を軽視した行為だと批判。平塚らいてうがさまざまな理由から堕胎もやむをえない場合があることを肯定。} 、廃娼論争\footnote{伊藤野枝が中上流の婦人団体の事業や活動の欺瞞を指摘し、特に矯風会の公娼廃止運動も批判。青山菊栄が野枝を批判し、公娼の悲惨な生活を指摘して公娼廃止が必要であることを主張。}、母性保護論争\footnote{与謝野晶子が、妊娠出産期の女性が国家から経済的な保護を受けることを要求することは、国家への寄食であり依頼主義であるとして批判。それに対して、平塚らいてうが母性保護を国家に求めるのは当然であると反論。山川菊栄が、この両者の立場は相反するものではなく、女性に二者択一を強いる経済関係そのものを改変しなければならないと主張。} など諸々の活発な論争が行なわれる\footnote{\citet{新フェミニズム批評の会98:青鞜を読む}を参照}。

1919年、新婦人協会が結成される。治安警察法の改正運動。1922年改正。妻の財産権、刑法の姦通罪削除、男女の機会均等などの要求。1924年婦人参政権獲得既成同盟会結成され、1925年、男子普通選挙が施行される(実施は1928年)。


戦後の新憲法と民法改正によって家制度は廃止され、公民権、参政権、教育の機会、法制上の妻の地位の平等が実現した。


\subsection{女性の政治参加への反対論と性差}

このように各国において、女性の政治参加には強い抵抗があった。ここで
その反対論の主なものを列挙してみよう。

\begin{itemize}

\item 男性の方がより有能である。「夫と妻とは、ただひとつの共通の関心をもっているとはいえ、その理解力も違っているので、時にはまた違った意志をもつのもやむをえまい。ところで最後の決定権すなわち支配権というものはどこかに置かれていなければならないので、自然それは、より有能で、より強い、男の方の手に置かれるのである。」(ジョン・ロック(1632-1704)『市民政府論』、岩波文庫, pp.84-85)

\item 女性は男性を助ける役目を果すことが自然である。「女性の教育はすべて男性に関連させて考えられなければならない。男性の気に入り、役に立ち、男性から愛され、尊敬され、男性が幼いときは育て、大きくなれば世話をやき、助言を与え、なぐさめ、生活を楽しく快いものにしてる、こういうことがあらゆる時代における女性の義務であり、女性の子どものときから教えられなければならないことだ」「男性は外、女性は内、これこそ自然の法則である」(ルソー(1712-1778)『エミール』下巻、岩波文庫p. 21)

\item 女性の本質は子どもを養育することにある。「女性の固有の特性は、子どもを教育することである。このようなことが家事に続く女性の任務であり、女性は生来的に徳を愛するように運命づけられているのである。(フランス保安委員のアマールの演説) (\citet{奥田暁子03:フェミニズム思想史}p. 73)

\item 女性は家事に向いている。「公務に従事できる市民は非常にわずかな数でしかないということは確かである・・・女性から家事を取りあげるとは、農民から鍬を、職人から仕事場を取りあげられないことと同様に、できないことである」(コンドルセ、\citet{奥田暁子03:フェミニズム思想史}, p.81)


\item 「女性には男性ほどの正義の感覚が見られない。」「好意や敵意の感情によって判断が左右されやすい。」(精神医学者 シーグムント・フロイト)

\item 女性には政治に必要とされる精神的・肉体的能力がない。
\item 女性の貞淑さや羞恥心が政治参加に向かない。
\item 女性は興奮しやすく錯乱・無秩序になりやすい。
\item 女性は家族の世話という仕事を捨てて政治に口を出すために家庭を出るべ
  きではない。

\end{itemize}
 
このように男女の性差が社会的な通念とされ、女性の政治参加の障害となると考えられたのである。


\subsection{J. S. ミルの『女性の解放』}


上のような女性の社会参加・政治参加に否定的な雰囲気のなかで、男女の平等を擁護する議論も多く提出された。フランス革命期に活躍したオランプ・ドゥ・グージュ(Olympe de Gouges, 1748-93)の『女性の権利宣言』(1791)やメアリ・ウルストンクラフト(Mary Wollstonecraft, 1759-97)の『女性の権利の擁護』(1791)などが有名であるが、最も洗練されており重要なものがイギリスの哲学者であり『自由論』の著者である J.  S. ミルの『女性の解放(\emph{Subjection of Women})』\citep{mill1869:_subjec_of_women}である。この著作は国内でも早くから翻訳され、女性運動に大きな影響を与えた。その議論は現在では当然視されることが多いものだが、基本的な知識としておさえておきたい。

\begin{itemize}
\item 女性の男性に対する法的隷属は誤りである。女性に平等な地位を認めることに対する反対は、議論の結果ではなく、実は\emph{感情}に根ざしている。女性の法的な従属は、それが人類にとって最良であるという理由によるものではなく、男性が強さにおいてまさっていうという肉体的事実が法律上の権利に変えられ、社会によって認められているにすぎない。女性がみずから進んで自分たちの無能力な状態を受けいれているという議論がある。しかし女らしくない望みは押えるようにという教育がなかったら、もっと多くの女性が抗議をするだろう。集団的に反抗することを女性が好まない理由のもう一つは、男性をひきつけるような人となることが女性の教育の最終目標になっているからである。

\item ひとは各々自分がもっとも好ましいと思う運命を試す自由を持つべきである。自由競争が行なわれれば、不適当なものは自然に排除される。法律上男女が平等であることは、家庭において道徳的情操を訓練するための方策としても有用である。政治生活や高報酬の仕事から女性をしめだすのは、家庭生活における女性の隷属を永久化するためである。ひとが各々そのもっとも好ましいと判断する運命を試す自由を持つことが近代社会の特色である。誰もたまたま自分の生まれついた運命に鎖で縛られている必要はない。したがって、女に生まれたからといって、その人が社会的な地位や職業につくことを禁じてはならない。もしその地位につくのに適切な女性を社会から排除すれば、それは実質的な損失となる。


\item 男性は支配し、女性は服従するのが天性であるという議論は無益である。両性の性質は現在の不自然な関係のなかで観察することしかできないからである。そのうえわれわれはどのような(社会や環境の)影響が人間の性格を形成するかについて無知である。

\item 男性は女性がどのような意見を持っているかを十分に知らないのだから、女性自身が語るべきことを語る自由を持たなければならない。
   

\item 女性の本性に自由な活動を許しても、女性がその本性に反して行動するということにはならない。自由競争の結果、女性は自分がもっとも必要とされ、最も適当である職務に従うようになるだろう。

 \item 女性を解放することにはさらに多くの社会的および個人的な利益がある。

   \begin{enumerate}
   \item 男性が根拠のない優越感を持ち尊大になることを防ぎ、
   \item 社会への奉仕に用いられる精神的活動を倍増させ、
   \item 男性とは違った視点からの意見の影響を増加させ、
   \item 各人の幸福の最も重要な要素としての自由を実現する、などである。
   \end{enumerate}

 \end{itemize}

 このようなミルの立場は、男女の性役割を認めたとしても、社会的効用および個人の幸福の観点からして女性に社会参加の自由を認める方がより好ましい結果をもたらす、という徹底して功利主義的でリベラルな立場である。そしてこの立場が、以降の女性解放の基本的な理論的支柱となった。


\section{第二波フェミニズム}



\subsection{大戦後}

第二次大戦後、日本を含め先進国では女性の参政権や法の前での男女平等は原則として実現された。しかしそれでも女性に対する実質的な差別の問題はなくならなかったと考える人びとがいた。


よく読まれたのが、\emph{シモーヌ・ド・ボーボワール Simone de Beauvoir}(1908-1986)の\emph{『第二の性』}(1949)である。「ひとは女に生まれない。女になるのだ。」「社会において人間の雌がとっている形態を定めているのは、生理的宿命、心理的宿命、経済的宿命のどれでもない、そうではなく、「文明全体」が「女」をつくりあげるのだ」と宣言し、自分自身や他の多くの女性の生の経験を詳細に検討した。女としての子ども時代からの成長、恋愛、性交渉、避妊、性病、妊娠、中絶、結婚、婚外交渉、売春など、それまで十分には議論されていなかった女性としてかかえる問題が赤裸裸に議論された。そして、男性中心的な思想や理論では理解することも解決することもできない「女の問題」が多数存在していることを指摘した。

%% 引用チェック


\emph{ベティー・フリーダン Betty Friedan}の(1921-)\emph{『女らしさの神話 Feminine Mystique』} (1963) (邦訳『新しい女性の創造』)は、戦後の豊かな米国中流階級の既婚女性を襲った満たされない心理状態を「名前のない問題」と名づけ分析した。女性が経済的に豊かになっても幸福を感じることができず、抑鬱状態にあるのはなぜだろうか、とフリーダンは問う。

\begin{quote}
  私は何がこんなに不満なんだろうと自分に問いかけます。私は健康だし、素晴しい子どもがいて、きれいな新築の家もあり、お金も十分持っている・・・とにかく、小さいころから、ずっと、誰か、それとも何かが、私の人生の面倒を見てくれていたような感じがします。——まず両親でしょう、それから大学に行って、恋愛をして、子どもができて、新しい家に引っ越すというふうに。そんな具合にいつも決まった目標があった。それである朝目が醒めてみると、これから先何を楽しみにすればいいのかわからなくなっていたのです。
\end{quote}

「名前のない問題」の原因は、女性の社会的自己実現欲求が阻害されていることにある。フリーダンは1966年全米女性機構(National Organization of Women, NOW)を結成し、リベラルな立場から女性差別の撤廃と女性の社会参画を目指した。彼女たちは、ミルに代表されるようなリベラリズムの立場をつらぬき、公的な領域への女性の社会進出をめざした。一方で家事や育児、性愛などの私的領域は基本的には個人の責任で解決すべき問題であるとした。

一方、19世紀後半からの自然科学の発展にともない、知性やその他の心理的特徴の生物学的・統計的分析が行なわれるようになった。男女の性差の科学的な研究がはじめられた。女性の知性の研究。骨相学や神経解剖学。女性と男性の脳には違いがある。女性は男性に比べて小さな脳を持ち、脳の大きさは知性をある程度反映していると考えられた。

% 1910年にはHelen Thompson Wooleyがそれまでの性差研究を批判。
%  1940年代、ビネーのIQ尺度調査での男女の性差研究。

% 「生まれか育ちか(nature vs nurture)?」

文化人類学の分野では、\emph{マーガレット・ミード}(Margaret Mead) (1901-1978)がが強い影響力を持っていた。ミードは南太平洋サモアのフィールドワークから、男性性や女性性が文化は社会によって多様であり、それぞれの社会はその社会特有のジェンダー秩序によって構造化されていると主張した\citep{ミード61:男性と女性}。たとえばサモアの人びとは西欧社会とは比べものにならないほど性的に自由であり、おだやかで対人関係での衝突が少なく、また西欧社会で思春期の少年少女たちが経験するような心理的葛藤を持たない。西欧社会の強い性規範は普遍的なものではなく、文化によって形づくられているにすぎない。ここから、文化決定論、文化相対主義と呼ばれる立場が有力になる。


性医学者\emph{ジョン・マネー}や精神科医\emph{ロバート・ストーラー} \citep{ストーラー73:性と性別}は、性転換者、衣裳倒錯者、生物学的に性に異常のある患者などを研究し、生物学的性別と社会的・文化的な性別は必ずしも一致しないと主張し、人間のジェンダー自認(gender identity)は生物学的要因ではなく、親の養育態度など心理学的要因によるものであるとした。ジェンダーはもともと西洋語の文法上の性の分類を示す用語である。たとえばドイツ語では太陽`die Sonne'は女性名詞であり、月`der Mond'は男性名詞である。いっぽう、フランス語で `le soleil' は男性名詞であり、月`la lune' は女性名詞である。このような名詞のジェンダー区分はしばしば言語的に恣意的・偶然的であり、それ以上の根拠を持たないことから、男女の性差に関しても生物学的な基盤を持たないものを「ジェンダー」と呼ぶようになったのである。そしてこのような「性差は養育の結果である」という観点は、女性解放運動に大きな影響を与えることになる。


\subsection{意識向上運動}

アメリカ公民権運動、あるいはフリーダンの著書や活動をきっかけにして、60年代末から、法や制度における男女平等だけでは女性の解放には十分ではなく、女性自身の意識改革が必要であるという自覚が生まれた。女性は、自分たちが内面化してしまっている「女らしさ」「男らしさ」などの規範意識や行動から逃れる必要があると考えられた。女性だけのグループを形成し、問題と直面した。このような運動は\emph{意識向上運動} (Consciousness Raising, CR)と呼ばれた。法や制度など公的な場での男女の不平等だけでなく、それまでは個人的 private な領域とされてきた家族や恋愛関係、性欲や性行動などが議論の題材として問題として扱われ、それまで個人的なものであるとされていた女性の種々の抑圧や差別の体験は、実は女性が共通してもつ社会的体験であることが「発見」された。「\emph{個人的なことは政治的である The personal is the political.}」という有名な宣言が第二波フェミニズムのモットーとなった。

これらのグループは、それまでのリベラルな女性平等運動に比して、女性差別を\emph{根本から}認識しつくがえそうとしてみずから「\emph{ラディカル}フェミニズム」を名乗った。


\subsection{男女関係の政治学}

ラディカルフェミニズムの中心人物の一人が、『性の政治学 (Sexual Polisics)』(1971) \citep{millett70:_sexual_politics}を著したケイト・ミレットである。ミレットによれば、ある人間のグループが他の人間のグループに支配される仕組が政治である。そして、\emph{家父長制(父権制、patriarchy)}とは、年長の男性が年少者を支配するに加えて、男性が女性を支配しているという年齢と性別による二重の制度である。家父長制はあらゆる政治的、社会的経済的形態にいきわたっており、恋愛関係や性関係なども例外ではない。

\begin{quotation}
  生得の権利によって統治する集団は、急速に消滅しつつあるが、にもかかわらず生まれによる一つの集団を、別の生まれによる集団が支配するという、古くからある普遍的図式が一つ残っている——性の分野にはびこっている図式がそれである。

  ・・・われわれの社会秩序の中で、ほとんど検討されることもなく、いや気づかれることさえなく(にもかかわらず制度化されて)まかりとおっているのが、生得権による優位であり、これによって男が女を支配しているのだ。この体制を通じて、「内部植民地化」がひどく巧妙なかたちで成しとげられてきた。・・・性による支配はわれわれの文化のおそらくもっともいきわたったイデオロギーとして通用し、またわれわれの文化のもっとも基本的な権力概念を与えている。

  これというのも、われわれの社会が、他のあらゆる歴史上の文明と同じく、父権制だからである。軍隊、産業、テクノロジー、大学、科学、行政官庁、経済——要するに、社会の中のあらゆる権力の通路は・・・すべて男性の手中にあることを思い起こせば、この事実はただちに明らかになる。(pp. 71-2)

\end{quotation}



このような「性の政治」は文化のあらゆる側面に染みわたっている。

男は攻撃的で知的で強く、女は受動的で感情的で無能で従順であるというステレオタイプ的視点が存在する。自然科学はそのようなステレオタイプ的視点を生得的なものとして正当化しようとする。しかし実際には「男らしさ」「女らしさ」は\emph{社会化の結果}でしなかない。

家父長制は女性を「妻」と「売春婦」、美醜、年齢といったさまざまな階級に分け、互いに敵対させる。また、男性は女性を経済的に支配している。

さらに家父長社会では、性と暴力が密接に結びつけられている。

\begin{quote}
  われわれは父権制と暴力とを結びつけて考えることには慣れていない。父権制は、暴力を使う必要はほとんどないほどに、その社会化の体制は完璧であり、その価値観にたいする一般の同意ぶりは徹底しており、人間社会に永年にわたって、あまねくいきわたってきたのだ。ふつうわれわれは過去の父権制の蛮行を、異国のあるいは「原始的な」習慣と考える。現在の蛮行は病理学的、ないしは例外的な行動に限られた、一般的重要性をもたない個人的逸脱的行為とみなされる。しかし、ほかの全体主義的イデオロギーと同様に(人種差別主義や植民地主義はこの点でいくらか父権制と類似している)、父権制社会における統制は、緊急事態にも、恒常的な威嚇の道具としても、暴力支配によらずしては完徹しないだろうし、操作不能にすら陥るだろう。
\end{quote}

このような暴力には、強姦、ポルノグラフィー、インドでの妻の殉死、旧中国の纏足、中絶の禁止がもたらす非合法の中絶、イスラムでの女性のベール、アフリカ諸国での性器切除(FGM)、人身売買、本人の意志によらない結婚、畜妾、売春などが含まれる。


宗教においても女性は常に劣位に置かれる。また処女性がさまざまな儀礼と禁令によって重んじられている。


\vspace{1zw}もう一人大きな影響力を持ったのが『性の弁証法』(1971) \citep{firestone70:_dialec_of_sex}を著したシュラミス・ファイアストーンである。彼女は一見望ましいとされるロマンチックな恋愛関係のなかにも男女の政治があることを指摘した。女性抑圧の根源は性の階級制度にある。男性は女性を恋愛と生殖を基盤として支配している。ファイアストーンは、「女と愛は、社会を支えている土台であり、このふたつを調べてみれば、文化の構造そのものを脅かすことになる」男性文化とは、見返りを求めない女の愛情の力に寄生したものであったし、いまもそうである。」という。

彼女によれば、男性優位の起源は、生物学的家族という生殖の基本単位にある。妊娠出産能力を持つ女性は生物学的家族に拘束され、社会は男女という不平等な生物学的階級に分断されている。一方、女性はセックスを武器に男性から感情的な安定や経済的な階級保障を引きだしている。

このような場で、「ロマンス」は「性による女性差別を強化する文化装置」として働いている。現在の社会では、女性は男性にとってただセックスの対象であり、女性は自分のことをエロティックな存在であると自認するほどになっている。その際、女性のセックスは規格品化されてしまっている。つまり女性は常に性的なものと考えられており、肉体的な属性によって自分の価値をはかるように強制されている。「女性の性の規格品化は、女性が、全体としての階級制に目隠しをされる過程であり、そのことは、男性の目には、女性は個性をもった個人としては見えないようになることを意味している」と言う。たとえば男性の「俺はブロンドが好きだ」という発言によってブロンドの女性は侮辱されたと感じる。それは自分を「他の女とは違ったものにしている肉体的な属性によって自分の価値をはかるようになっているからである。」

そしてファイアストーンは、このような性による階級支配を解体するには、(1)妊娠・出産・育児といった生物学的な専制からの女性の解放、(2)すべての人の経済的独立、(3)女性と子どものより広い社会との結合、(4)性の自由が必要であるとした。具体的には現在の閉鎖的な結婚家族制度を解体し、生活形態、性的志向などを自由に組み合わせたライフスタイルを選べるような社会が望ましいとした。



\subsection{性暴力}

第二波のフェミニズムは、理論的な側面だけでなく、
社会に浸透しているが隠されてしまっている性暴力に注目した。



\emph{スーザン・ブラウンミラー}の\emph{『われわれの意思に反して』} (邦訳『レイプ・踏みにじられた意思』) \citep{brownmiller75:_again_our_will} やスーザン・エストリッチの『リアル・レイプ』\citep{estrich87:_real_rape}は、「強姦とは無分別で抑えがたい欲望から生じる犯罪ではなく、征服欲にかられた者が相手を貶め、自分の物にするために行う敵対的で暴力的な行為である」として、性暴力が行なわれる現実の場を暴露した。ブラウンミラーは、強姦とは単に性欲に駆りたてられた男性が犯してしまう犯罪というだけでなく、「強姦とは、すべての男がすべての女を恐怖状態にとどめておくことによって成立する、意識的な威嚇のプロセスに他なららない。\citep[邦訳p.6]{brownmiller75:_again_our_will}」という。デートレイプ、ドメスティックバイオレンスなどそれまでは強姦と認められていなかった性暴力が暴露されることになった。



% 強姦は暴力かセックスか、男性は生物学的に強姦への傾向を持つか。

\emph{キャサリン・マッキノン}\citep{mackinnon79:_sexual_haras_of_workin_women} )は、セクシャルハラスメントを女性に対する差別であるとして批判し、法的な運動を起こした。マッキノンはまた、ポルノグラフィーをも批判する\citep{mackinnon93:_only_words}。「ポルノは理論、強姦は実践」であって、ポルノグラフィーは女性の従属と搾取を構造的に制度化し、強姦その他性暴力を促す。
% \emph{家父長制}(父権制、 partiarchy)。
% 暴力、男らしさの誇示、女性嫌悪、サディズム、同性愛恐怖、強姦、
% ポルノグラフィー。

買春に代表される\emph{性の商品化}は一夫一婦制と裏表の関係にあることが確認され、(1)女性の性を断片化し、「モノ」化する、(2)男性と女性の非対称性によって権力関係を強化し、(3)商品化に値する性とそうでない性とに女性の性を振りわけるとして強く非難された\footnote{ただし一方で、売春をセックスワークとして合法化するべきであるとするフェミニストも現われた。}。

このようにして、社会におけるさまざまな隠された性差別や性暴力の存在が、フェミニズムの批判的な眼差しのもとで次々と暴露され、問題化されていった。この分析の上で役に立ったのが、先に挙げた私的な関係での政治力学であり、また単なる生物学的な性差としてのセックスと、社会的に押しつけられたものとしてのジェンダーの区別だったのである。


\section{分散と反発}

\subsection{フェミニズムの分散}

しかし70年代からフェミニズムは急速に分散し多様化していった。ここでは詳しく見ることができないが、リベラル対ラディカルというフェミニズム内部での対立に加え、マルクス主義フェミニズム、精神分析派フェミニズム、ポストモダンフェミニズム、エコロジカルフェミニズム、ブラックフェミニズム、母性主義フェミニズム、カルチュラルフェミニズムとさまざまな「冠つき」フェミニズムが主張されるようになった。

% 現状では男女の真の平等は不可能であり、女性が真に解放されるためには\emph{強制的異性愛}に対抗し、女性同士での性愛による連帯が必要であると主張する人びとがいた。女性抑圧の根源は異性愛の強制であり、この異性愛の制度を崩壊させる必要がある。

% \begin{quote}
%   レズビアンとは何であろう。レズビアンとは、爆発点にまで達したすべての女性の怒りである。レズビアンとは、社会が女性に認める以上に、もっと全人的で自由な人間でありたいと願う、内なる衝動のままに行動する女性である。レズビアンとは、社会の最も基本的な役割である女性の役割に課せられた制限や抑圧を断固拒否した女性である。
% \end{quote}

なかでも、1984年に出版されたキャロル・ギリガンの『もうひとつの声\emph{ In a Different Voice}』\citep{gilligan82:_in_differ_voice}は大きな話題を読んだ。これは、道徳的思考の発達において女性と男性の間には大きな違いがあり、女性は、具体的な人間関係やケア(配慮)を重視して道徳的思考を行ない、男性の抽象的で、正義や権利を中心にした思考とは対照的であるとするものだった。



\subsection{進化論、社会生物学、遺伝学}


80年代後半から、生物学者たちが進化論や遺伝学、動物行動学の成果を踏まえて人間の「本性」について新しい見解を打ち出すようになった。われわれの性差はどの程度生物学・遺伝的的要因に由来するのか、どの程度が文化に左右されているのか?

先のキャロル・ギリガンに触発された心理学的・社会学的研究、\emph{エドワード・O・ウィルソン}らの1980年代から進化論を背景とした社会生物学が注目を浴びている。

また、先に挙げたマーガレット・ミードの研究は問題が多いことが指摘された。ミードはサモア島の恋愛と性愛はヨーロッパの基準に比して自由であり、また少年少女は思春期にヨーロッパの少年たちが体験する葛藤を経験しないと報告したが、これらは意図的に誤った情報を伝えられた結果であって、実際にはサモアでも強い性規範と思春期の葛藤が存在することが、デレク・フリーマン\citep{freeman83:_margar_mead_and_samoa}らによって確認された。

またストーラーの研究に反して、「文化的」であると思われていた性差はやはり生物学的な基盤を持つと主張されるようになっている。たとえば哲学者スティーブン・ピンカーはすでに確証された事実として以下のようなものを挙げる\citep{pinker02:_blank_slate}。

\begin{itemize}

\item すべての文化で、男性は女性より攻撃的で犯罪傾向がある。

\item 多くの性差は類人猿共通である。

\item 社会的条件づけが認められないような年齢の小児にも性差が認められる。また、女性として育てられた性器異常の男性の多くは、自分を男性と考える。したがって性自認は社会的条件づけの結果であるという説は疑わしい。

\item ミトコンドリアと細胞核のDNA比較調査から、ミトコンドリアの遺伝子の方が核より多様であることが発見された。男性の方が女性より繁殖成功率の幅が広い、つまり、残した子どもの数に個体間で大きな差があったことがうかがわれる。これは、男性の方がより乱交的行動傾向を持つことを意味するかもしれない。

\item 性ホルモンのレベルの違いに応じて攻撃性に変化があることが示されている。また、男女の脳には構造的な違いが存在する。女性は生理周期によって心的活動に影響を受ける。女性ホルモンが低濃度の時期は一般に男性の方が得意であるとされる分野(空間図形の回転など)のテストでで好成績をおさめる。

\end{itemize}

このような最近の研究は、先天的・生物学的な性差と社会的構築物であるジェンダーという単純な区分に対する疑いをひきおこしている。もっとも極端な社会生物学者のなかには、女性の社会進出を阻んでいるのは社会制度ではなく、むしろ女性自身が社会進出や昇進を望んでいないからであり問題はないとさえ主張する人びともいる。\citep{browne98:_david_labour}

\subsection{差異と平等、事実と価値、個人と集団}


ここまで、19世紀から20世紀にかけての女性解放運動を駆足で見てきた。

さて、最初の問いに戻ろう。「セックス」に加え、「ジェンダー」概念が必要とされたのはどうしてだろうか。上で見たような第二波フェミニズムが家父長制{\――}女性を抑圧する男性支配の権力制度{\――}を批判する動きのなかで、女性に対する抑圧を分析する手段として自然的・生物学的な性差であるセックスに加えて、たんに歴史的・偶然的に女性たちに押しつけられた性差としてのジェンダーが概念が必要とされたわけである。「ジェンダー」は女性に対する抑圧を分析する道具であった。

公的な場への女性の進出を求める古典的なリベラルな立場と、私的な局面こそ女性抑圧の場であるとするラディカルな立場、そしてその後のフェミニズムの分散とフェミニズムへの反発は、男女の差と男女の平等をめぐる争いであると言うことができる。自由か平等か、平等か差異か、性差はどの程度文化の影響を受けているのか、ということについては現在も議論が進行中である。

最後に二つの点について注意をうながしておかねばならない。

第一に、性差は集団としての平均的な差であり、個人の差ではないことである。生物学的な男女を比較すれば、平均した場合男性の方が身長が高いが、だからといってすべての男性が女性より背が高いわけではない。

第二に、より重要な点として、単なる事実に関する知識だけから、価値や規範を主張することはできない。「人間は病気になる」という事実から、「病気になってもよい」という価値に関する主張を導くことはできない。「セックス」の意味でも「ジェンダー」の意味でも、「性差」は事実に関する事柄であり、「平等」は実は「人々はその人に応じて平等に扱われるべきだ」という価値や規範にかかわる事柄である。

もちろん、集団としての男女の間にどのような差があるかという事実に関する問題は、よりよい社会を構想する上で非常に重要だが、それは性差の事実に関する知識から直接に導かれるわけではない。よりよい社会とはどのような社会であるかという価値にかかわる判断が必要なのである。



% 統計的に男女に性差があることはどういう意味を含むだろうか。
% 性差があることは女性差別とどうかかわるだろうか。最後に
% J. S. ミルの流れをひく現代の倫理学者ピーター・シンガーの
% 男女の平等に関する議論を簡単に見てみよう。

% シンガーの立場は伝統的なリベラルの立場であり、
% ラジカルフェミニストたちはおそらくこのような
% 形式的でリベラルな平等の原則によっては、女性に対する
% 差別は廃絶できないと主張するであろう。


\if0
\twocolumn{}
\section{資料}


\subsection{グージュ『女性の権利宣言』}


\subsection{ウルストンクラフト}



\subsection{女性の所信宣言 セネカ・フォールズ、1848年}
\label{sentiment}

  

人類の歴史の過程におき、それらのうちのある一部の人々が、従来占めていた地位と異なりはするが自然法と自然の神の法が彼らに授けた地位を、世界の人々のなかで占めなければならなくなった時に、人類が行ってきた意見表明に対して当然の敬意を払おうとするならば、なぜ自分たちがそのような道程を取らざるを得なくなったかという理由をその人々は宣言すべきである。

よって、わたしたちは、以下の真実を自明の理として宣言する。すべての男性とすべての女性は平等に造られていることを、彼らはともに造物主によって一定の不可譲の諸権利を賦与されていることを、これらの諸権利のなかには、生命、自由、幸福の追求の権利が含まれていることを、これらの諸権利を保障するために政府は創設され、その正当な権力は被治者の同意に由来することを。これらの目的に害をなす政府があるならば、その政府に対する忠誠を拒否し新しい政府の創設を主張することは、その政府に忍従を余儀なくされている人々の権利である。新しい政府は、彼らの安全と幸福にもっとも効果があると思われる諸原則に基礎を置き、そのような形体へと政府の権限を組織するものでなければならない。たしかに、長く存続した政府が軽微で一時的な事由で変更を受けるべきではないことは、分別の命じるところではあるのだが、その結果のすべての経験が示すところは、人間には、自分たちが慣れ親しんできた形体を廃止してあるべき状態に戻るよりは、害悪に耐えうる間はこれに耐える傾向がある、ということである。しかしながら、常に同じ目的を果たすための長期にわたる一連の権力の濫用と簒奪が、人々に絶対的な専制を強いようとする意図を明らかにする場合には、その政府を捨て去り自らの将来の安全のために新しい防護の組織を備えることこそが、人間の義務である。そして、現在の政府のもとで女性たちが耐えてきた忍従こそ、まさにそのような忍従だったのであり、また、本来その資格がある平等な地位を女性たちが今この時要求せざるを得ないのは、まさにそのようなやむにやまれぬ必要性からなのである。

人類の歴史は、女性に対する男性の絶対的な専制の確立を直接の目的とした、男性による女性に対しての権利侵害と剥奪の繰り返しの歴史である。これを証明するために、公平なる世界に向けて、以下の事実を明らかにしよう。

男性は、不可譲の権利である選挙権の行使を、決して女性に認めてこなかった。

男性は、女性がその制定過程に何の発言権も持っていない法に、女性を従わしめようとしてきた。

男性は、もっとも無知で身分が低い男性であろうとも、自国民であるか外国人であるかを問わず与えられている諸権利を、女性には制限してきた。

男性は、選挙権という市民として第一の権利を女性から奪い、それによって立法府で女性が代表されないままにしておくことによって、女性をあらゆる面から抑圧してきた。

男性は、女性が結婚した場合には、法律的に民事上の無能力者としてきた。

男性は、女性が自ら働いて得た賃金をも含むあらゆる財産上の権利を、女性から奪ってきた。

男性は、多くの犯罪を、夫の面前で行われたならば妻がそれらの罪を犯しても不可罰にすることで、女性を道徳的に責任を取れない存在にしてきた。法が妻の自由を奪い貞操を管理する権限を夫に与えたせいで、婚姻契約において妻は夫への恭順を約束することを余儀なくされ、夫は事実上、妻の主人となっているのである。

男性は、離婚法を制定するに際し、何が離婚の正当な事由となるかについて、また別離の場合に誰に子供の監護権が与えるかについて、女性の幸福には一顧だにしなかった。離婚法は、すべての場合において、男性の優位という誤った前提に立っており、男性の手にすべての権力を委ねているのである。

男性は、既婚女性としてのすべての権利を女性から奪っただけでなく、女性が独身で不動産を所有する場合には、税を課して政府を支えさせるが、政府がこの女性の存在を認めるのは、彼女の財産が政府にとって利益を生じるときのみである。

男性は、利益のある仕事のほとんどすべてを独占してきたのであり、女性が就くのを許された仕事からは、女性はほんのわずかの報酬しか得られない。男性は、女性に対し、男性によって最も名誉あると思われる冨と特典へのあらゆる道を閉ざしている。名を知られた神学、医学、法学の教師は、女性にはひとりもいない。

男性は、女性が十分な教育を受ける機会をなくしてきた。あらゆる大学教育は、女性に門を閉ざしている。

男性は、使徒伝の権威を理由に、いくつかの例外はあるが教会の職務への公的な参加から女性を排除したり聖職から女性を排除することによって、国家において従属的地位しか与えなかったのと同じように教会においても従属的な地位しか、女性に与えてこなかった。

男性は、男女で異なる道徳律を世の中に示すことにより、人々に誤った道徳観を創り出した。この道徳律へに違反すれば、女性は社会から締め出されるが、男性の場合は、許容されるだけでなくほとんど問題にされない。

男性は、女性の行動範囲を定め得るのはその女性の良心と彼女の信じる神のみであるにもかかわらず、これを自分たち男性の権利であると主張して、エホバその人の特権を奪い取ってきた。

男性は、為しうるかぎりのあらゆる方法でもって、自分たちの力についての女性の自信を砕き、女性の自尊心を弱め、女性が従属的で屈辱的な人生に甘んじるように努めてきた。
  

さて、この国の半分の人々が選挙権を完全に奪われ、社会的にも宗教的にも劣位におとしめられていることを考慮し、また上記のような不正な諸法を考慮し、さらに、女性たちが、虐げられ抑圧され自分たちの最も神聖な諸権利を不正に奪われていると感じているがゆえに、わたしたちは、合衆国市民として女性に属するすべての権利と特権が、直ちに女性たちに認められるべきであると主張する。
  
わたしたちの目の前にある偉大な事業を始めるにあたり、かなりの誤解や間違った理解、嘲笑を受けるだろうことを予想している。しかし、わたしたちは、自分たちの目的を達成するために、できる範囲のすべての手段を利用するつもりである。わたしたちは、職員を雇い、パンフレットを配布し、州と連邦の立法府に請願し、聖職者と出版界に私たちへの協力を要請するよう努める。わたしたちは、この会議に続き、この国の各地で次々と同じような会議が続かんことを希求する。


\subsubsection*{決議}





(以上に鑑み、)自然の偉大なる教えが、「人間はその人自身の真の実質的(substantial)幸福を求める」ということであることは、広く知られている。ブラックストーンは著書の「法律釈義」のなかで、この自然法は、人類と同じくらい古く、神ご自身が命じたものであって、当然あらゆる義務に優越する、と述べている。自然法は、世界中で、あらゆる国々で、あらゆる時代において、人々を拘束する。人間の法の中でこの法に背く法は、何の効力も持たず、一方、効力のある法は、その力のすべて、その効力のすべて、その権威のすべてが、直接的にも間接的にも、本来の法であるこの自然法に由来するのである。ゆえに、わたしたちは、以下のことがらを決議する。
  
自然の教えこそが「あらゆる義務に優越する」がゆえに、いかなる形であれ女性の真なる実質的幸福に相反するような法は、自然の偉大なる教えと矛盾するものであることを、決議する。

自分たちの良心にかなった社会的地位を女性が占めるのを阻んだり、男性の地位よりも劣った地位に女性を位置づけたりするあらゆる法は、偉大なる自然の教えに反しており、ゆえに何の力も権威も持たないことを、決議する。

女性は男性と対等であり、造物主によってそのように作られており、人類の至高の徳に従うならば女性はそのようなものとして認められるべきことを、決議する。

この国の女性たちが、彼女らがそのもとに生きる法について知識を得るべきこと、また彼女らが、現在の自分の地位で満足だと宣言することで自分たちの卑しい地位を明らかにしてしまったり、自分が望んでいる権利はすべて手にしていると主張して無知をさらけだしたりすることがないようにと、決議する。

男性が自分たちの知的優越性を主張する一方で女性に道徳的優越性を与えていることに鑑みれば、あらゆる宗教的な会合において機会がある場合には、女性が演説し説教するように推奨することは、まったくもって男性の義務であることを、決議する。

社会において女性に求められるのと同じだけの貞淑さ、繊細さ、洗練が、男性にも求められなければならず、これを破った場合には、男性にも女性にも同じだけの厳しさで対応がなされねばならないことを、決議する。

女性が聴衆に向けて話しかけたときにしばしば投げかけられる不躾で不作法な非難の声は、女性を力づけようという気持ちでステージ、コンサート、サーカスでの女性の演技や発表を見に来ている人たちからは、生じるべくもないものであることを、決議する。

聖書の歪曲した適用によって生じたものであり、市民の心を腐敗させてしまう厳しい制約に対して、女性はあまりにも長い間不満を覚えずにいたのであるが、今や、偉大な造物主が彼女に課したより大きな領域へと動かねばならないときが来たことを、決議する。

選挙権という女性の神聖な権利を確実に手にすることが、この国の女性たちの義務であることを、決議する。

  人権の平等は、人間が持つ能力と責任が[性にかかわらず]同じものであるという事実から、必然的に帰結するものであることを、決議する。

  ゆえに、以下を決議する。造物主によって自分たちの行為に関して[男性と]同一の能力と同一の責任を与えられているのであるから、あらゆる正当な手段によってあらゆる正当な主義主張を訴えていくことが、男性同様女性の権利であり義務であることは明らかである。とりわけ、道徳と宗教という重要な課題に関しては、私的生活においても公の場においても、利用に問題のないあらゆる手段を利用し、開催に問題のないあらゆる集会で、文章を書いたり公演したりすることで、男性の同胞とともにこれらの課題にかんして啓蒙を行うことに参加するのが、女性の権利であることは自明である。そして、これらのことは、神に教えられた人間の本性の諸原則から生じた自明の真理であり、これらに反するどのような習慣や権威も、近代のものであれ恐ろしい制裁を伴う古代のものであれ、自明の誤りとみなされ、人類の存在とあい争うものなのである。

  [最後の会合で、ルクレシア・モットは、次の決議を提案し取り上げた。]

われわれの主張が迅速に達成されるためには、聖職の独占の打倒を求めての、また、さまざまな交易、専門職、商業に対し、男性と平等な参加を女性に保障することを求めての、両性によるところの熱心でたゆまぬ努力が必要であることを、決議する。
\begin{flushright}
  (澤敬子訳)
\end{flushright}





\subsection{ソジャーナ・トルースの演説}
\label{truth}
「白人男性はやがて権利について話し合っている南部の黒人と北部の女性の間で板挟みになるだろう。しかしここで一体何について話しあっているのかい。あの向うにいる男性は、女が馬車に乗るのに手を貸したり、抱き上げて溝を渡したり、いたるところで一番いい場所を譲る必要があると言っている。今までだれも私を馬車に助け上げてくれたり、田圃から引っ張り上げてくれたり、最上の場所をくれたりはしなかった。私は女じゃないのかい。見てご覧、私の腕を・・・土を耕し、苗を植え、収穫を納屋へ運んで働いてきた。男だって私には勝てない。それでも私は女ではないのかい。私は男と同じくらい働き、男と同じくらい食べることができる。そして同様に苦痛に耐えることができる。だからって、私は女ではないのかい。私は13人の子どもを生み、みんな奴隷に売られていった。泣き叫んでもイエス様以外だれも私の叫びを聞いてくれなかった。それでも私は女ではないのかい。

そこの黒い服を来た小さな男の牧師は男と同じ権利を持つことはできない、なぜならキリストは女ではなかったから、と云う。キリストはどこからきたんだい。・・・神様から、そして女から生まれたのだ。男は神とはなんの関係もなかった。(後略)」

\begin{flushright}
  (奥田暁子訳)
\end{flushright}


\section{さらに学習するために}

\begin{itemize}
\item 辻村みよ子、『女性と人権』、日本評論社、1997。
\end{itemize}


\fi
\ifx\mybook\undefined
\bibliographystyle{eguchi}  
\bibliography{bib,library}
\end{document} %----------------------------------------------------------------
\fi






%%% Local Variables:
%%% mode: japanese-latex
%%% TeX-master: t
%%% coding: utf-8
%%% End:
