
\chapter{マキアベリ}


わたくしのねらいは、読むひとに直接役に立つものを書くことであって、物事についての想像の世界のことより、生々しい真実を追うほうがふさわしいと、わたしは思う。これまで多くの人は、現実のさまを見もせず知りもせずに、共和国や君主国を想像で論じてきた。しかし、いかに生きているかということと、いかに生きるべきかということは、はなはだかけ離れている。だから、人間いかに生きるべきかを見て、現に生きている現実の姿を見逃す人間は、自立するどころか、破滅を思い知らされるのが落ちである。なぜなら、何ごとについても善い行いをすると広言する人間は、よからぬ多数の人々の中にあって、破滅せざるをえない。したがって、自分の身を守ろうとする君主は、よくない人間にもなることを習い覚える必要がある。そして、この態度を必要に応じて使ったり使わなかったりしなければならない。(『君主論』)
