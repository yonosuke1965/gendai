\ifx\mybook\undefined
\documentclass[autodetect-engine,dvipdfmx-if-dvi,ja=standard]{bxjsarticle} \usepackage{mystyle}
\author{江口聡}
%\date{}
\title{マキアヴェッリの『君主論』}
\if0 %----------------------------------------------------------------

\fi  %----------------------------------------------------------------
\begin{document}
\maketitle
\else\chapter{}
\fi


\href{http://bit.ly/2W4939e}{ニコロ・マキャベッリ}(Nicol{\`o} Machiavelli, 1469-1527)。イタリア人、フィレンツェ共和国の外交官。「マキャヴェリ」「マキャベリ」「マキァヴェリ」「マキァヴェッリ」などいろいろ表記される。

\begin{itemize}
\item \href{http://bit.ly/2VWy1Y3}{チェーザレ・ボルジア}を尊敬。
\item メディチ家最盛期の当主(\href{http://bit.ly/2HKxOgu}{ロレンツォ・デ・メディチ})に『君主論』を献呈。
\end{itemize}




% \chapter{マキアベリ}


\section{現実を見るべし}



\begin{oframed}
わたくしのねらいは、読むひとに直接役に立つものを書くことであって、物事についての想像の世界のことより、生々しい真実を追うほうがふさわしいと、わたしは思う。これまで多くの人は、現実のさまを見もせず知りもせずに、共和国や君主国を想像で論じてきた。しかし、いかに生きているかということと、いかに生きるべきかということは、はなはだかけ離れている。だから、人間いかに生きるべきかを見て、現に生きている現実の姿を見逃す人間は、自立するどころか、破滅を思い知らされるのが落ちである。なぜなら、何ごとについても善い行いをすると広言する人間は、よからぬ多数の人々の中にあって、破滅せざるをえない。したがって、自分の身を守ろうとする君主は、よくない人間にもなることを習い覚える必要がある。そして、この態度を必要に応じて使ったり使わなかったりしなければならない。(『君主論』)
\end{oframed}

\begin{itemize}
\item はっきりした政治的現実主義(リアリズム)。
\end{itemize}


\section{愛されるより恐れられよう}



\begin{oframed}
  〔君主は国民に〕恐れられるよりも愛される方がいいか、あるいは反対であるか。
  これに対して双方であることが望ましいと人は答えるであろうが、しかし両者を得ることは難しく、したがって両者のうちどちらかが欠けざるをえない場合には、愛されるよりも恐れられる方が安全である。それというのも人間に関しては一般的に次のように言いうるからである。人間は恩知らずで気が変わり易く、偽善的で自らを偽り、臆病で貪欲である。君主が彼らに対して恩恵を施している限り彼らは君主のものであり、生命、財産、血、子供を君主に対して提供する。しかしこれはすでに述べたようにその必要が差し迫っていない場合のことであり、その必要が切迫すると彼らは裏切る。したがって彼らの言葉に全幅の信頼をおいている君主は他の準備を整えていないために滅亡する。

  ……人間は恐れている者よりも愛している者を害するのに躊躇しない。なぜならは好意は義務の鎖でつながれているが、人間は生来邪悪であるからいつでも自己の利益に従ってこの鎖を破壊するのに対して、恐怖は君主は常に一体不可分である処罰に対する恐怖によって維持されているからである。

  それにしても、君主は仮に好意を得ることがないとしても、憎悪を避けるような形で恐れられなければならない。恐れられることと憎まれないことは、恐れられることと愛されることよりもより容易に両立しうる。このことは君主が市民や臣民の財産と彼らの婦女子に手を出さないならば、必ずや実現されると思われる。かりに誰かの血を流すことが必要な場合には、適切な正当化と明確な理由の下に行なわれなければならない。しかしなによりも他人の財産に手を出さないようにするべきである。それというのは人間は財産の喪失よりも、父親の死の方をより速やかに忘れるものだからである。……

  君主が軍隊とともにあって多くの兵士を統率する場合には、残酷であるという評判をまったく気にする必要はない。なぜならばかかる評判なしに、なんらかの軍事行動の準備をすることは不可能であるからである。{ハンニバル}の驚嘆すべき行為の一つとして、この評判を気しなかったことが挙げられる……(第17章)
  
\end{oframed}

\begin{itemize}
\item 人間に対する冷い視線。

\item この断片の最後で古代カルタゴの名将\href{http://bit.ly/2W0IyBh}{ハンニバル}の名前が出てくるが、『君主論』を通じて歴史的有名人・事件からの教訓が数多く出てくる。多くはローマ史から。
\end{itemize}

\section{信義なんか守る必要がない}



\begin{oframed}
  ……賢明な君主は信義を守るのが自らにとって不都合で、約束した際の根拠が失われたような場合、信義を守ることができないし、守るべきでない。

  もし人間がすべて善人であるならば、このような勧告は好ましくないであろうが、人間は邪悪で君主に対する信義を守らないのであるから君主もまたそれを守る必要はない。そして君主は信義を守らないことを潤色する正当な口実を必ずや見いだすものである。……〔君主は〕優れた偽善者、偽装者たることが必要である。人間というものは非常に単純で目先の必要によってはなはだ左右されるので、人間を欺こうとする人は欺かれる人間を常に見いだすものである。(第18章)

\end{oframed}



\begin{itemize}
\item もっとも有名な個所。
\end{itemize}

\begin{oframed}

……君主にとて必要なのは上に述べたような資質(慈悲や信義など)を有することではなく、それらを持っているように見えることである。さらにあえて述べるならば、君主がこれらの資質を備え、それに従って行動するのは有害であるが、それを備えているように見えるのは有益である。すなわち、慈悲深く、信義に厚く、人間性に富み、正直で信心深く見え、そうあるのは有益である。しかしそうでない必要が生じた時にはその正反対の態度をとることがけい、そうする術を知るように、自らの気質をあらかじめ作り上げておくことが必要である。君主、特に新しい君主は、人間が良いと考える事柄に従ってすべて行動できるものではなく、権力を維持するためには信義にそむき、慈悲心に反し、人間性に逆らい、宗教に違反した行為をしばしばせざるをえない、ということを知っておかなければならない。(第18章)
  
\end{oframed}



\section{読書案内}

『君主論』はビジネスマンに非常に好まれる。ただし「超訳」ものはやめておこう。


\begin{itemize}
\item \href{https://amzn.to/2JWukdy}{架神恭介の『よいこの君主論』}(ちくま文庫)はおもしろいのでぜひ読んで、サークル・バイト先などを制圧し平定しよう。(amazonの「なか見!検索」で目次と最初の章が読める)
\item 塩野七生『マギアヴェッリ語録』(新潮文庫)
\item 鹿島茂『社長のためのマキアヴェリ入門』(中公文庫)なども。
\end{itemize}


\nocite{マキアヴェリ01:君主論,マキアヴェッリ98:君主論,佐々木毅94マキアヴェッリ,machiavelli1513:_il_princ}

\nocite{架神09よいこの君主論}

\ifx\mybook\undefined
\addcontentsline{toc}{chapter}{\bibname}
\bibliographystyle{jecon}
\bibliography{bib}


\end{document} %----------------------------------------------------------------
\fi






%%% Local Variables:
%%% mode: japanese-latex
%%% TeX-master: t
%%% coding: utf-8
%%% End:
