\documentclass[uplatex,dvipdfmx]{jsarticle}
\usepackage{okumacro,plext}
\usepackage{natbib}
\usepackage{url}
\usepackage{txfonts}
\usepackage[utf8]{inputenc}
\usepackage[T1]{fontenc}
\usepackage{otf}
%\usepackage{my_resume}
%\usepackage{graphicx,wrapfig}
%\usepackage[greek,english]{babel}
%\usepackage{teubner}
%\usepackage[dvipdfm,bookmarkstype=toc=true,pdfauthor={江口聡, EGUCHI Satoshi}, pdftitle={}, pdfsubject={},pdfkeywords={},bookmarks=false, bookmarksopen=false,colorlinks=true,urlcolor=blue,linkcolor=black,citecolor=black,linktocpage=true]{hyperref}
  \AtBeginDvi{\special{pdf:tounicode EUC-UCS2}}% platex-utf8 でも OK
\author{江口聡}
%\date{}
\title{現代社会入門I 期末テスト}


% \def\anaumei#1{\xkskip%
% \framebox[1.5cm]{\bf (\rensuji\label{#1})}\xkskip}
% \def\anaakiref#1{{\bf (\ref{#1})}}

\usepackage{renban}



\newcommand{\sentakusi}[4]{
\hspace{.3zw}
\emph{ア}\hspace{1zw} #1 \hspace{2zw} \emph{イ} \hspace{1zw}#2 \hspace{2zw}\emph{ウ}\hspace{1zw} #3 \hspace{2zw}\emph{エ}\hspace{1zw} #4

}



% \sentakusi{}{}{}{}



\if0 %----------------------------------------------------------------

\fi  %----------------------------------------------------------------
\begin{document}
\maketitle

\subsection*{問1 以下の文章を読み、空白にもっとも適切であるものの下のア〜エから選べ (1問2点、60点満点)}

\begin{enumerate}

\setlength{\parskip}{.9zw}
\setlength{\itemsep}{.9zw}


\item   古代ギリシアの哲学者プラトンは理想的な国家は\空欄数字{}であると考えた。


  \sentakusi
{民主制}
{君主制}
{貴族制}
{僭主制}


\item   15〜16世紀フィレンツェの\空欄数字{}は『君主論』で、君主は国家の保持と拡大のためにはあらゆる権謀術策を使用するべきであり、嘘や約束やぶりも辞すべきでないと主張した。

\sentakusi
{エラスムス}
{チェーザレ・ボルジア}
{ピコ・デラ・ミランドラ}
{マキアベリ}

\item 「知は力なり」と唱えたベイコンは、人間の思い込みや先入観を\空欄数字{}と呼び、自然の正しい解明のためには、観察や実験にもとづいて得られた知見を秩序づけ一般的な法則を導く帰納法を用いることが必要であると考えた。

\sentakusi
{イドラ}
{イマーゴ}
{アタラクシア}
{ノモス}


\item \空欄数字{}は名誉革命を正当化する『市民政府論』を書き、王権神授説を反駁し、政治権力の期限を社会契約にあるとした。

\sentakusi
{ピープス}
{フィルマー}
{ホッブズ}
{ロック}


\item   18世紀フランスの哲学者\空欄数字{}の社会契約説は、「一般意志」という概念を導入して社会秩序の正当性を論ずるところに最大の特徴をもっている。一般意志とは、「つねに公的利益のみをめざす共同体の意志」をいう。

\sentakusi
{ホッブズ}
{ルソー}
{ロック}
{ヴォルテール}


\item   イギリスのカルヴァン主義者たちは特に\空欄数字{}と呼ばれ、質素で禁欲的な生活を重んじた。

\sentakusi
{クェーカー}
{バプティスト}
{ピューリタン}
{メソジスト}


\item   18世紀イタリアの法学者\空欄数字{}は、『犯罪と刑罰』で拷問と死刑に対する反対論を展開し、教育による犯罪の防止を提唱した。

\sentakusi
{フィチーノ}
{ブルーノ}
{ベッカリーア}
{ラ・メトリ}

\item   『哲学書簡』『カンディード』などの著作や、「君の意見には反対だが君がそれを発言する権利は死んでも守る」という発言で有名な\空欄数字{}はフランス啓蒙主義の代表者であり、従来のキリスト教を迷信であるとして強く批判した。

\sentakusi
{ダランベール}
{ディドロ}
{ラグランジュ}
{ヴォルテール}


\item   \空欄数字{}は経済活動は、人々の利己的な自発性と、それを規制するルールにもとづく自由競争によって活性化すると考えた。各人の私益の追求は「見えざる手」に導かれ自然な秩序形成がおこなわれ、公益を生み出す。

\sentakusi
{ケインズ}
{スミス}
{マンデヴィル}
{リカード}

\item   18世紀末、イギリスの経済学者\空欄数字{}は『人口論』で人口の増加と生産力の増大の不均衡から、人口抑制の手段なくしては貧困は必然であると論じた。

\sentakusi
{ゴドウィン}
{ジェヴォンズ}
{マルサス}
{ワルラス}

\item 1989年のフランス革命後、ジャコバン派の\空欄数字{}が国民公会から独裁権を認められ、大規模な粛清を断行した。彼の「徳なき恐怖は忌まわしく、恐怖なき徳は無力である」言葉は有名で、こうした訴える政治手法は「テロリズム」と呼ばれる。

\sentakusi
{エベール}
{ダントン}
{ナポレオン}
{ロヴェスピエール}


\item   \空欄数字{}は『女性の権利の擁護』を著してルソーを批判し、女性が男子と同様の教育を受ければ同等の能力を身につけるであろうと主張し、平等な教育を要求した。

\sentakusi
{メアリ・ウルストンクラフト}
{オランプ・ド・グージュ}
{シュラミス・フィアストーン}
{ケイト・ミレット}

\item   ドイツの哲学者\空欄数字{}は、「啓蒙とは人間がみずから招いた未成年状態から抜けでることである」という有名な言葉を残した。また、著書『永遠平和のために』のなかで、国家間の紛争の解決のためには、民主的な国家間の協定による国際機関がつくられねばならないと主張した。

\sentakusi
{カント}
{シェリング}
{ヘーゲル}
{フィヒテ}


\item   18世紀イギリスの保守思想家\空欄数字{}は『フランス革命についての省察』で、伝統と漸進的な改革の重要性を説き、フランス革命は悪しき人為社会を設立する試みであり、なんら積極的な成果を産み出さないだろうと予言した。

\sentakusi
{スミス}
{ハミルトン}
{バーク}
{プライス}

\item   フランスの\空欄数字{}は、『法の精神』でイギリスの政治状況からヒントを得て、立法・行政・司法の三権の分立の必要を唱えた。

\sentakusi
{マルブランシュ}
{モンテスキュー}
{モンテーニュ}
{ルソー}


\item   18世紀の理性重視の啓蒙主義に対して、19世紀には個性や直接的な感情を重視する文学運動である\空欄数字{}が盛んになった。ドイツではシラーやグリム兄弟、イギリスではワーズワースやコールリッジなどが有名である。

\sentakusi
{ロマン主義}
{合理主義}
{実存主義}
{民族主義}

\item   ドイツの哲学者\空欄数字{}は、親密な愛によって結ばれている家族と、私的利益の追求がおこなわれる市民社会の矛盾を止揚した発展段階の最高のものとしての国家において、個人の利益と全体の利益が一致調和すると考えた。

\sentakusi
{ゲーデル}
{シュライエルマッハー}
{ショーペンハウアー}
{ヘーゲル}

\item   19世紀イギリスの工場主\空欄数字{}は労働者の労働条件改善のために尽力し、また私有財産のない共同社会を構想しアメリカでニューハモニー村を経営する実験を行なった。

\sentakusi
{オーウェル}
{オーウエン}
{フーリエ}
{ベンサム}


\item 『自由論』の著者である\空欄数字{}は、『女性の隷属』(『女性の解放』)を発表して女性の解放と生活・職業の自由を訴え、また国会議員として女性の参政権を擁護した。

\sentakusi
{ミル}
{バーリン}
{ハクスリー}
{トゥルース}


\item   マルクスは、\空欄数字{}とともに『共産党宣言』を発表し、労働者に団結を呼びかけた。

\sentakusi
{エンゲルス}
{クロトポキン}
{プルードン}
{レーニン}


\item   \空欄数字{}は、遺伝、変異、自然淘汰などによって生物は環境に適応し進化・多様化すると考え、19 世紀なかばに 『種の起源』などを著し、生物学と思想界の大きな影響を与えた。

\sentakusi
{スティーブンソン}
{ダーウィン}
{ファラデー}
{プリーストリー}


\item   19世紀アメリカの思想家\空欄数字{}は『ウォールデン:森の生活』などで自然と一体となった生活の価値を称揚し、また奴隷制や圧政に反対するため、市民的不服従の手段を用いるべきだと主張した。

\sentakusi
{ジェファーソン}
{ソロー}
{フランクリン}
{マディソン}



\item   フランスの\空欄数字{}は各種の統計を用いて自殺は社会的な原因にもとづくとする『自殺論』などを著わし、実証的な社会学のさきがけとなった。

\sentakusi
{ウェーバー}
{コント}
{デュルケーム}
{ラッサール}


\item   19世紀後半、イギリスの\空欄数字{}は社会を生物と類似的にとらえる社会有機体説をとなえ、自由競争のもとでの適者生存の法則が社会を発展させ進化させると考えた。


\sentakusi
{シジウィック}
{スペンサー}
{ニーチェ}
{ヘッケル}

\item   イタリアの医学者\空欄数字{}は、犯罪者の体格や容貌などを計測し分類する犯罪人類学を提唱した。

\sentakusi
{ガル}
{スエーデンボルグ}
{ピネル}
{ロンブローゾ}


\item ベルンシュタインらは、マルクスらの暴力革命・プロレタリア独裁という思想を否定し、議会民主政治のもとでの社会の漸進的改良をもたらす\空欄数字{}を唱えた。

\sentakusi
{修正社会主義}
{功利主義}
{実用主義(プラグマティズム)}
{国家社会主義}


\item   19世紀末イギリスでウェッブ夫妻や劇作家バーナード・ショーを中心に形成された\空欄数字{}は、議会を通じて資本主義社会の弊害を修正しながら漸進的に社会主義を実現しようとした。


\sentakusi
{キェルケゴール協会}
{ケンブリッジ協会}
{フェビアン協会}
{ブルームズベリー・グループ}


\item   \空欄数字{}は遺伝の研究から、家畜の品種改良と同じように、人間にも人為選択を適用すればより良い社会ができるとする「優生学」を発想した。

\sentakusi
{ピアソン}
{メンデル}
{ゴルトン}
{フィッシャー}



\item \空欄数字{}はレーニンの死後、世界革命論に対抗して一国社会主義を主張し、国内体制の維持を優先するべきだとして成功を収め、1953年に没するまでソ連の政治権力を手中におさめた。

\sentakusi
{ゴルバチョフ}
{スターリン}
{トロツキー}
{フルシチョフ}


\item   \空欄数字{}・ルーズベルトは第二次世界大戦後、国連に働きかけ、世界人権宣言の起草に貢献した。

\sentakusi
{エレノア}
{セオドア}
{デラノ}
{フランクリン}

% \item 社会主義運動家であり「非戦論」を唱えた\空欄数字{}は明治44年、大逆事件で刑死した。

% \sentakusi
% {中江兆民}
% {吉野作造}
% {川上肇}
% {幸徳秋水}

% \item 明治44年、平塚らいてふらは婦人問題に関する文芸雑誌『青鞜』を発行した。中心的なメンバーには野上八重子、与謝野晶子、\空欄数字{}らがいた。

% \sentakusi
% {伊藤野枝}
% {加藤シヅエ}
% {榎美沙子}
% {田中美津}



\end{enumerate}



% \subsection*{2. ホッブズの自然状態についての考え方を可能なかぎり詳細に説明せよ}


% \subsection*{2. ホッブズとロックの自然状態についての考え方の違いを可能なかぎり詳細に説明せよ}


\subsection*{問2 (エキストラ問題、答えられれば最高+10点) }
「人権」という考え方は、世界史・思想史においてどのように成立したか、中学1年生に向けて2〜3分程度で説明するつもりで簡潔に述べよ。


\end{document} %----------------------------------------------------------------







%%% Local Variables:
%%% mode: japanese-latex
%%% TeX-master: t
%%% coding: utf-8
%%% End:
