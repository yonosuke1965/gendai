\documentclass[dvipdfmx]{jsarticle}
\usepackage{okumacro,plext}
\usepackage{my_resume}
\usepackage{natbib}
\usepackage{url}
\usepackage{txfonts}
\usepackage[utf8]{inputenc}
\usepackage[T1]{fontenc}
\usepackage{lineno}
\usepackage{otf}
%\usepackage{my_resume}
%\usepackage{graphicx,wrapfig}
%\usepackage[greek,english]{babel}
%\usepackage{teubner}
%\usepackage[dvipdfm,bookmarkstype=toc=true,pdfauthor={江口聡, EGUCHI Satoshi}, pdftitle={}, pdfsubject={},pdfkeywords={},bookmarks=false, bookmarksopen=false,colorlinks=true,urlcolor=blue,linkcolor=black,citecolor=black,linktocpage=true]{hyperref}
  \AtBeginDvi{\special{pdf:tounicode EUC-UCS2}}% platex-utf8 でも OK
%\author{江口聡}
%\date{}
\title{現代社会入門I:世界史知識チェック(中学『歴史』内容)}


\usepackage{renban}



\newcommand{\sentakusi}[4]{
\hspace{.3zw}
\emph{ア}\hspace{1zw} #1 \hspace{2zw} \emph{イ} \hspace{1zw}#2 \hspace{2zw}\emph{ウ}\hspace{1zw} #3 \hspace{2zw}\emph{エ}\hspace{1zw} #4

}



\if0 %----------------------------------------------------------------

\fi  %----------------------------------------------------------------
\begin{document}
\maketitle

\subsection*{問1 以下の文章を読み、空白にもっとも適切であるものの下のア〜エから選べ}


\begin{enumerate}

\setlength{\parskip}{.8zw}
\setlength{\itemsep}{.8zw}


% \item 紀元前1500年頃、黄河流域で起こった国(王朝)はなんですか。

\item 紀元前3世紀、中国を統一した国はなんといいますか。

\sentakusi{周}{魏}{秦}{漢}

\item 1642年頃イギリスで起こった、議会派と国王軍との間の戦いをなんといいますか。

\sentakusi{清教徒革命}{ばら戦争}{三十年戦争}{七年戦争}


\item イギリス議会が国王を追放した1688年のできごとを何といいますか。

\sentakusi{名誉革命}{清教徒革命}{七月革命}{11月革命}


\item アメリカ合衆国の初代大統領となったのは誰ですか。

\sentakusi{フランクリン}{マディソン}{アダムズ}{ワシントン}



\item フランス革命が起こったときの国王は誰ですか。

\sentakusi{ルイ14世}{ルイ16世}{ルイ18世}{ナポレオン3世}


\item 19世紀前半にイギリスで起こった、選挙権の拡大を求める運動を何といいますか。

\sentakusi{ブルーストッキング}{サフラジェット}{チャーチスト運動}{普選運動}



\item アメリカ北部出身で、1862年奴隷解放を宣言したアメリカ大統領は誰ですか。

\sentakusi{ジェファーソン}{フランクリン}{モンロー}{リンカーン}





\item イギリスと清との間で1840年に始まった戦争を何といいますか。

\sentakusi{アヘン戦争}{アロー戦争}{薩英戦争}{辛未洋擾}



\item インドのムガル帝国を滅ぼし直接統治をした国はどこですか。

\sentakusi{イギリス}{アメリカ}{オランダ}{フランス}


\item 1906年に中国で三民主義を唱えたのは誰ですか。

\sentakusi{毛沢東}{孫文}{張作霖}{蒋介石}

\item ロシア革命が1917年に起こったときの指導者は誰ですか。

  \sentakusi{ブレジネフ}{フルシチョフ}{レーニン}{スターリン}


\item 第一次世界大戦の講和会議が1919年に開かれた都市はどこですか。

\sentakusi{パリ}{ウィーン}{ポーツマス}{ジュネーブ}


\item 国際連盟の設立を提唱したアメリカの大統領は誰ですか。

\sentakusi{ウィルソン}{ルーズベルト}{トルーマン}{ケネディ}


\item イタリアで1922年にイタリア国王から首相に任ぜられ独裁政治を始めたのはだれですか。


\sentakusi{オクタヴィアヌス}{ガリヴァルディ}{コルレオーネ}{ムッソリーニ}


%\item 日本海軍が大敗した1942年6月の戦いを何といいますか。
%#\sentakusi{マレー沖海戦}{ミッドウェー海戦}{マリアナ沖海戦}{レイテ沖海戦}

\item 第二次世界大戦後東西に分裂し、1990年に統一された国はどこですか。

  \sentakusi{ドイツ}{ベトナム}{オーストリア}{ハンガリー}


\item 下のうち、1960年代に起きた戦争・紛争はどれですか。

\sentakusi{ユーゴスラビア紛争}{ベトナム戦争}{朝鮮戦争}{ルワンダ内戦}

\item ニュートンが万有引力の法則を発表した。

\sentakusi{16世紀}{17世紀}{18世紀}{19世紀}

\item 電磁波の発見はいつごろですか。

\sentakusi{18世紀後半}{19世紀前半}{19世紀後半}{20世紀前半}

\item 最初の動力飛行機の実験はいつごろですか。

\sentakusi{1900年代}{1910年代}{1920年代}{1930年代}

\item 最初の人工衛星はいつ成功しましたか。

\sentakusi{1950年代}{1960年代}{1970年代}{1980年代}



\end{enumerate}







%\bibliographystyle{eguchi}
%\bibliography{bib}


\end{document} %----------------------------------------------------------------







%%% Local Variables:
%%% mode: japanese-latex
%%% TeX-master: t
%%% coding: utf-8
%%% End:
