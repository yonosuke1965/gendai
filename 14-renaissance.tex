\ifx\mybook\undefined
\documentclass[uplatex,dvipdfmx]{jsarticle} \usepackage{mystyle}%\author{} %date{}
\usepackage{tabularx}
\title{ルネサンス〜宗教改革}
\if0 %----------------------------------------------------------------

\fi  %----------------------------------------------------------------
\begin{document}
\maketitle
\else\chapter{ルネサンス}
\fi


\section{ルネッサンス}

\begin{itemize}
\item 14〜16世紀のイタリアRenaissance\footnote{この言葉自体は19世紀の歴史家ミシュレやブルクハルトの用語。} (再生、文芸復興)。
\item ギリシア古典がアラビア語に翻訳されたものが保存されており、十字軍によって再発見される。
\item 14世紀のペスト流行。
\item ギリシア・ローマ文化の再生。
\item 人文主義 humanism, humanisme, Humanismus。人間や人間に関することがらを最重視する態度。狭い意味での人文主義(ユマニスム)。
\item → 学問の発展。
\item 現世肯定的。
\item ピコ・デラ・ミランドラ(1463)。アウグスティヌスに影響を受ける。「人間の尊厳」。人間は自由意志をもっており、神のようにも獣のようにもなることができる。
\item エラスムス(1466-1536)。『痴愚神礼賛』。貪欲な王や諸侯、嘘つきな商人、生徒を虐待する教師、戦争に明け暮れるローマ教皇、学派争いにあけくれる学者たちなどを嘲笑・風刺。プラトンの哲人王の理想を復活。、イエスは平和と愛と望んでいたのであり、平和と相互の献身がキリスト教の真のメッセージのはずである。→ 戦争を肯定する教会批判。
\item トマス・モア(1478-1535)。『ユートピア』など。どこにもない理想の国を描く。私有財産や貨幣は存しない。人は1日6時間だけ労働し、あとは学問をして過ごす。選挙で役人を選出する。宗教的寛容。
\item ラブレー。
\item ボッカチオ。
\item 美術は写実的、肉感的に。
\end{itemize}



\section{宗教改革}
\begin{itemize}

\item 15世紀、\emph{グーテンベルク}が活版印刷術を発明→印刷革命。『グーテンベルク聖書』(ラテン語版)。

  
\item 16世紀には、ローマ教会は堕落していると見られていた。特に問題とされたのが、金銭によって罪が許されるとする免罪符(贖宥状)の発行など。

\item \emph{ルター} (1483--1546)が1517年に「95か条の論題」でローマ教会を批判。ローマ教会に経済的に搾取されていると感がていたドイツ諸侯、領主、農民らがルターを支持。教会はルターを破門。
  
\item 『ルター聖書』(ドイツ語ヘブライ語・古典ギリシア語からの翻訳。)

\item 宗教改革の要求から対立が深まり宗教戦争。 → プロテスタント(新教)。労働の重視。あらゆる職業は天職。

\item スイスのツヴィングリがルターの影響を受け宗教改革運動を開始、ルター派とも対立する。 →
フランス人の\emph{カルヴァン}(1509--64)も宗教改革運動に参加、ジュネーブの政権を掌握し神権政治をおこなう。

\item 16世紀後半にはカルヴァン派信仰が西ヨーロッパに広がり、フランスではユグノー、ネーデルランドではゴイセン、スコットランドでは長老派、イングランドでは\emph{ピューリタン}(清教徒)と呼ばれる。

\item ローマ・カトリック教会はプロテスタント運動に対抗するために各種の内部改革をおこなう。イエズス会は厳格な規律のもとで布教と教育に励み、アジア、アフリカ、ラテンアメリカなどにも積極的に宣教師を派遣。ポルトガルやスペインの植民活動と協力関係を結ぶ。

\end{itemize}

\vspace{1zw}

\begin{tabularx}{1.0\linewidth}{X X X}

  ローマカトリック  & ルター派 & カルヴァン派 \\ \hline

  教会中心。& 信仰至上主義。精神面を強調。 & 蓄財を肯定。 \\
  普遍教会\footnote{ローマ教皇をトップとするピラミッド組織。} & 領邦教会\footnote{領主が教会のトップとなる。} & 長老主義\footnote{教会員のなかから信仰のあつい人物を長老に選び、牧師を補佐させる。} \\
  政教一致 & 政教分離 & 政教一致 \\
  世俗職業を蔑視 & 世俗の職業も尊重 & 禁欲と勤勉を奨励  \\

\end{tabularx}
\vspace{1zw}



\section{マキャベリの現実主義}

16世紀イタリアは北部の都市国家、中部の教皇領、南部のナポリ王国に分かれていた。フランスや神聖ローマ帝国、オスマン帝国などがが関わって長い戦争状態になる。
% ハプスブルグ家の神聖ローマ帝国とヴァロワ朝のフランス

フィレンツェのニコロ・マキアヴェリ (Niccol{\`o} di Bernardo Machiavelli, 1469-1527)は、『君主論』で君主はあらゆる手段を使ってよいと主張した。


\begin{quote}
  わたくしのねらいは、読むひとに直接役に立つものを書くことであって、物事についての想像の世界のことより、生々しい真実を追うほうがふさわしいと、わたしは思う。これまで多くの人は、現実のさまを見もせず知りもせずに、共和国や君主国を想像で論じてきた。しかし、いかに生きているかということと、いかに生きるべきかということは、はなはだかけ離れている。だから、人間いかに生きるべきかを見て、現に生きている現実の姿を見逃す人間は、自立するどころか、破滅を思い知らされるのが落ちである。なぜなら、何ごとについても善い行いをすると広言する人間は、よからぬ多数の人々の中にあって、破滅せざるをえない。したがって、自分の身を守ろうとする君主は、よくない人間にもなることを習い覚える必要がある。そして、この態度を必要に応じて使ったり使わなかったりしなければならない。(『君主論』)
\end{quote}


人々が善き共同体を営む場としてのポリスといった考え方は甘い。人々はみな利己的で貪欲で恩知らず。共通の善などには関心をもたない。支配者の権力どうしのぶつかりあいとしての国家と国家関係を考える。君主はあらゆる手段を使って利己的な人々をコントロールし、国家を管理することを考えるべき。「君主は人々に愛されるよりは恐られるように、しかし憎まれないようにふるまうべきだ」「不正義はあっても秩序ある国家と、正義はあっても無秩序な国家のどちらかを選べと言われたら、私は前者を選ぶであろう。」



\begin{itemize}
\item 架神恭介『よい子の君主論』(ちくま文庫、2009)がおもしろいので読もう。「小学生の権力闘争を舞台に楽しく学べる『君主論』。クラスを牛耳りたい良い子のみんなも、お子様に帝王学を学ばせたい保護者の方も、国家元首を目指す不敵なあなたも必読の一冊。」
\end{itemize}




% \section{さらに学習するために}


\ifx\mybook\undefined
\bibliographystyle{eguchi}
\bibliography{bib}


\end{document} %----------------------------------------------------------------
\fi






%%% Local Variables:
%%% mode: japanese-latex
%%% TeX-master: t
%%% coding: utf-8
%%% End:
