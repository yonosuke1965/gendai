\documentclass[]{jsarticle}
\usepackage{okumacro,plext}
\usepackage{natbib}
\usepackage{url}
\usepackage{txfonts}
\usepackage[utf8]{inputenc}
\usepackage[T1]{fontenc}
%\usepackage{my_resume}
%\usepackage{graphicx,wrapfig}
%\usepackage[greek,english]{babel}
%\usepackage{teubner}
%\usepackage[dvipdfm,bookmarkstype=toc=true,pdfauthor={江口聡, EGUCHI Satoshi}, pdftitle={}, pdfsubject={},pdfkeywords={},bookmarks=false, bookmarksopen=false,colorlinks=true,urlcolor=blue,linkcolor=black,citecolor=black,linktocpage=true]{hyperref}
  \AtBeginDvi{\special{pdf:tounicode EUC-UCS2}}% platex-utf8 でも OK
\author{江口聡}
%\date{}
\title{20世紀の倫理学理論}
\if0 %----------------------------------------------------------------

\fi  %----------------------------------------------------------------
\begin{document}
\maketitle

\section{ロールズの正義論}

\begin{itemize}
\item リベラル派のロールズの正義の二原理。

  \begin{enumerate}
  \item 各人は、すべての人びとにとっての同様な自由な体系と両立しうる最大限の自由への平等な権利を持たなければならない。
  \item 社会的・経済的不平等は、次の二つの条件を満たすように配置されなければならない。
    \begin{enumerate}
    \item 最も恵まれない人びとの最大限の利益となるように
    \item 公正な機会の均等という条件の下ですべての人に開かれた職務や地位にのみともなうように。
    \end{enumerate}
  \end{enumerate}

  \begin{quote}
    福祉国家的資本主義は、財産や技能の当初の分配を所与の実質的な不平等として容認し、その上
    で事後的に所得を再配分しようとする。これにたいして財産所有の民主主義は、財産や技能の当
    初の分配における平等を求めるが、所得の再配分措置にたいしては、それほど重きをおかない。
  \end{quote}

\item 功利主義者は所有権に二次的な価値しか認めない。あくまで効用(人びとの幸福)の最大化が目標。

\end{itemize}
\section{ノージックのリバタリアニズム}

\begin{itemize}

\item ノージックのような自由優先主義(リバタリアン)によれば、暴力や詐欺のような不正な手段によらずに、正当に獲得した財産であれば、自由に使ってもよいとされる。私的所有権はほぼ絶対。

  \begin{enumerate}
  \item 人びとは自分自身を所有する。
  \item 世界は原初的には誰のものでもない。
  \item 世界にはたらきかけ、自分自身の労働によって獲得したものはその人のものである。
  \item 他者の状態を悪くしないかぎり、世界の均等ではない取り分にたいする絶対的権利の獲得が可能である。
%   \item 世界の均等ではない取り分にたいする絶対的権利の獲得は比較的容易である。
   \item 一度、人びとが私有財産を占有すれば、資本や労働の自由市場が道徳的に要請される。
  \end{enumerate}


\end{itemize}

\section{ヘアの普遍的指令主義}
\begin{itemize}
\item 道徳判断は普遍的な指令。
\item 二層理論。理想的な道徳的思考は、完全な論理的な明晰さと、事実に関する充分な知識を要求するものである。したがって、生身の人間が簡単に行なえるようなものではない。自分の目先の利益を優先してしまうわれわれの傾向性や、実際の場面での思考の時間不足などを考慮するならば、そのつどそのつど上のような道徳的思考を行なうことは不可能である。むしろ、一般の生活でしばしば起こるような事例のほとんどでよい結果につながるような、おおまかな原則をあらかじめ選択し、通常はそれにしたがうようにした方が、全体としてよい結果になるはずである。
\end{itemize}






\nocite{rawls71:_theor_of_justic}
\nocite{nozick74:_anarc_state_and_utopia}
\nocite{hare81:_moral_think}



\end{document} %----------------------------------------------------------------







%%% Local Variables:
%%% mode: japanese-latex
%%% TeX-master: t
%%% coding: utf-8
%%% End:
